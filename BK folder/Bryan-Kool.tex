\documentclass{amsart}

\title{Donaldson-Thomas invariants of local elliptic surfaces via the topological vertex}
\author{Jim Bryan and Martijn Kool}
\date{\today}
\address{
Department of Mathematics\\
University of British Columbia \\
Room 121, 1984 Mathematics Road  \\
Vancouver, B.C., Canada V6T 1Z2  
}


%\usepackage{diagrams}
%\usepackage{eepic,epic}

\usepackage{amsmath}
\usepackage{amsmath,amsthm,amsfonts}
\usepackage{times}
\usepackage[all]{xy}
%\usepackage{amstex}
\usepackage[colorinlistoftodos]{todonotes}



%\newtheorem{thm}{Theorem}%[section]
\newtheorem{theorem}{Theorem}%[section]
\newtheorem{proposition}[theorem]{Proposition}
\newtheorem{conjecture}[theorem]{Conjecture}
\newtheorem{lemma}[theorem]{Lemma}
\newtheorem{corollary}[theorem]{Corollary}
\theoremstyle{definition}

\newtheorem{def-theorem}[theorem]{Definition-Theorem}
\newtheorem{remark}[theorem]{Remark}
\newtheorem{definition}[theorem]{Definition}
\newtheorem{example}[theorem]{Example}


\newcommand{\CC} {\mathbb{C}}          % complex numbers
\newcommand{\NN} {\mathbb{N}}		% natural numbers
\newcommand{\RR} {\mathbb{R}}		% real numbers
\newcommand{\ZZ} {\mathbb{Z}}		% integers
\newcommand{\QQ} {\mathbb{Q}}		% rationals
\newcommand{\PP} {\mathbb{P}}
\renewcommand{\AA} {\mathbb{A}}
\newcommand{\LL} {\mathbb{L}}
\newcommand{\FF} {\mathbb{F}}
\renewcommand{\O}{\mathcal{O}}


\newcommand{\rt}[1]{\stackrel{#1\,}{\rightarrow}}
\newcommand{\Rt}[1]{\stackrel{#1\,}{\longrightarrow}}
\newcommand\To{\longrightarrow}
\newcommand\into{\hookrightarrow}
\newcommand\Into{\ensuremath{\lhook\joinrel\relbar\joinrel\rightarrow}}
\newcommand\INTO{\ar@{^{(}->}[r]}
\newcommand\acts{\curvearrowright}


\newcommand{\Hom}{\operatorname{Hom}}
\newcommand{\Ker}{\operatorname{Ker}}
\newcommand{\End}{\operatorname{End}}
\newcommand{\GL}{\operatorname{GL}}
\newcommand{\Tr}{\operatorname{tr}}
\newcommand{\tr}{\operatorname{tr}}
\newcommand{\Coker}{\operatorname{Coker}}
\newcommand{\im}{\operatorname{Im}}
\newcommand{\M}{\overline{\mathcal{M}}}
\newcommand{\smargin}[1]{\marginpar{\tiny{#1}}}
\newcommand{\Sym}{\operatorname{Sym}}
\newcommand{\Coh}{\operatorname{Coh}}
\newcommand{\Hilb}{\operatorname{Hilb}}
\newcommand{\DT}{\operatorname{DT}}
\newcommand{\Var}{\operatorname{Var}}
\newcommand{\supp}{\operatorname{supp}}
\newcommand{\Spec}{\operatorname{Spec}}
\newcommand{\sm}{\operatorname{sm}}
\newcommand{\sing}{\operatorname{sing}}
\newcommand{\F}{\mathcal{F}}
\newcommand{\G}{\mathcal{G}}

\begin{document}

\begin{abstract}
%Jim's original:

We compute the Donaldson-Thomas invariants of a local elliptic surface
with section. We introduce a new computational technique which is a
mixture of motivic and toric methods. This allows us to write the
partition function for the invariants in terms of the topological
vertex. Utilizing identities for the topological vertex (some
previously known, some new), we derive product formulas for the
partition functions. In the special case where the elliptic surface is
a K3 surface, we get a new proof of the Katz-Klemm-Vafa formula.
\end{abstract}

\maketitle 

%\markboth{???}  {???}
%\renewcommand{\sectionmark}[1]{}


%\tableofcontents
%\pagebreak


\section{Introduction}

\section{Definitions}

Let $p : S \rightarrow B$ be an elliptic surface over a smooth projective curve $B$ of genus $g$. We make two assumptions:
\begin{itemize}
\item $p$ has a \emph{unique} section $B \hookrightarrow S$,\todo{Drop uniqueness... generating function becomes product with contribution from multiple sections.}
\item all singular fibres of $\pi$ are of Kodaira type $I_1$, i.e.~rational nodal fibres. 
\end{itemize}
We write $F_x$ for the fibre $p^{-1}(\{x\})$, for all closed points $x \in B$. We denote the classes of the fibre and section by $B, F \in H^2(S,\ZZ)$. Interesting examples are the elliptic surfaces $E(n)$, where $B = \PP^1$ and $S$ has $12n$ nodal fibres. Then $E(1)$ is the rational elliptic surface and $E(2)$ is the elliptic K3 surface.

Let $\beta \in H_2(S)$ be Poincar\'e dual to $B+dF$, where $d \geq 0$. 
%Then
%$$
%\beta^2 = 2h + B^2
%$$
Now let $X = \mathrm{Tot}(K_S)$ be the total space of the canonical bundle over $S$. Then $X$ is a non-compact Calabi-Yau 3-fold. Let
$$
\Hilb^{\beta,n}(X) = \{ Z \subset X \ : \ [Z] = \beta, \ \chi(\O_Z) = n\}
$$
denote the Hilbert scheme of proper subschemes $Z \subset X$ with fixed homology class and holomorphic Euler characteristics. K.~Behrend associates to any $\CC$-scheme of finite type $Y$ a constructible function $\nu : Y \rightarrow \ZZ$ \cite{Beh}. Applied to $\Hilb^{\beta,n}(X)$, the Donaldson-Thomas invariants of $X$ can be defined as\footnote{If $X$ is a compact Calabi-Yau 3-fold, R.P.~Thomas's original definition of DT invariants is by the degree of the virtual cycle of $\Hilb^{\beta,n}(X)$ \cite{Tho}. Behrend showed that this is the same as $e(\Hilb^{\beta,n}(X),\nu)$ \cite{Beh}. The advantage of the definition by means of virtual cycles is that the construction works relative to a base. This implies deformation invariance of the invariants.} 
$$
\DT_{\beta,n}(X) := \int_{\Hilb^{\beta,n}(X)} \nu \ de := \sum_{k \in \ZZ} k \ e(\nu^{-1}(\{k\})),
$$
where $e(\cdot)$ denotes topological Euler characteristic. Many of the key properties of DT invariants are already captured by the more classical Euler characteristic version\footnote{From the point of view of \cite{Joy, Bri}, there are two natural integration maps on the semi-classical Hall-algebra. One corresponds to weighing by the Behrend function and the other to weighing by the ``trivial'' constructible function which is constant equal to 1.}
$$
\widehat{\DT}_{\beta,n}(X) := \int_{\Hilb^{\beta,n}(X)} 1 \ de = e(\Hilb^{\beta,n}(X)).
$$
For brevity, we define
\begin{align*}
\Hilb^{d,n}(X) &:=\Hilb^{B+dF,n}(X), \\
\DT_{\beta,n}(X) &:= \DT_{B+dF,n}(X), \\
\widehat{\DT}_{d,n}(X) &:= \DT_{B+dF,n}(X).
\end{align*}
The generating functions of interest are
\begin{align*}
\DT(X) &:= \sum_{d \geq 0} \DT_d(X) q^d := \sum_{d \geq 0} \sum_{n \in \ZZ} \DT_{d,n}(X) p^n q^d, \\
\widehat{\DT}(X) &:= \sum_{d \geq 0} \widehat{\DT}_d(X) q^d := \sum_{d \geq 0} \sum_{n \in \ZZ} \widehat{\DT}_{d,n}(X) p^n q^d.
\end{align*}

Since we are dealing with generating functions and our calculations involve cut-paste methods on the moduli space, it is useful to introduce the following notation. We define
\begin{align*}
[\Hilb^{d,\bullet}(X)] := \sum_{n \in \ZZ} [\Hilb^{d,n}(X)] p^n,
\end{align*}
which is an element of $K_0(\Var_{\CC})(\!(p)\!)$, i.e.~a Laurent series with coefficients in the Grothendieck group of varieties. We also write $\Hilb^{d,\bullet}(X)$ to denote the union of all $\Hilb^{d,n}(X)$. Therefore $\Hilb^{d,\bullet}(X)$ is a $\CC$-scheme which is locally of finite type.




\section{Push-forward to the symmetric product} \ref{sym}

The scaling action of $\CC^*$ on the fibres of $X$ lifts to the moduli space $\Hilb^{d,\bullet}(X)$. Therefore
$$
\int_{\Hilb^{d,\bullet}(X)} 1 \ de = \int_{\Hilb^{d,\bullet}(X)^{\CC^*}} 1 \ de.
$$
Recall that 
$$
\int_{\Hilb^{d,\bullet}(X)} 1 \ de = \sum_{n \in \ZZ} p^n \int_{\Hilb^{d,n}(X)} 1 \ de \in \ZZ(\!(p)\!).
$$
We revisit the $\CC^*$-fixed point locus in detail in the next section. 

Denote by $\Sym^d(B)$ the $d$th symmetric product of $B$. Recall that we have projections 
$$
X \stackrel{\pi}{\longrightarrow} S \stackrel{p}{\longrightarrow} B.
$$
A subscheme $Z$ of $ \Hilb^{d,\bullet}(X)^{\CC^*}$ always contains the zero section $B \subset S \subset X$. We can remove it and consider the scheme $\overline{Z \setminus B}$. There exists a morphism\todo{Can/do we want to write this easier?}
\begin{align*}
\rho_{d} : &\Hilb^{d,\bullet}(X)^{\CC^*} \longrightarrow \Sym^d(B), \\
&Z \mapsto \mathrm{supp}( p_* \pi_* \O_{\overline{Z \setminus B}} ),
\end{align*}
where $\supp(\cdot)$ denotes the scheme theoretic support, which is a divisor on $B$. 
%In terms of the stratification of Proposition \ref{C^*}, it is easy to describe this map: it sends $Z$ described as in the proposition  to
%$$
%\sum_{\alpha=1}^{n} |\lambda^{(\alpha)}| x_\alpha \in \Sym^h(B).
%$$
We obtain
$$
\int_{\Hilb^{d,\bullet}(X)^{\CC^*}} 1 \ de = \int_{\Sym^d(B)} \rho_{d*}(1) \ de,
$$
where $f_d := \rho_{d*}(1)$ is a constructible function on $\Sym^d(B)$. Its value at a closed point $\sum_i a_i x_i$ is 
$$
\int_{\rho_{d}^{-1}\big(\sum_i a_i x_i\big)} 1 \ de.
$$
We are interested in the calculation of
$$
\widehat{\DT}(X) = \sum_{d \geq 0} \widehat{\DT}_d(X) q^d =\sum_{d \geq 0} q^d \int_{\Sym^d(B)} f_d \ de.
$$
We prove that the constructible function $f_d : \Sym^d(B) \rightarrow \ZZ(\!(p)\!)$ has two multiplicative properties. The first one is described as follows. Denote by $B^{\sm} \subset B$ the open subset over which the fibres $F_x$ are smooth and by $B^{\sing}$ the $N$ points over which the fibres $F_x$ are singular. We can consider the restrictions of $f_d$ to $\Sym^d(B^{\sm}) \subset \Sym^d(B)$ and $\Sym^d(B^{\sing}) \subset \Sym^d(B)$. Denote by $M(q)$ the MacMahon function.
\begin{proposition} \label{mult1}
Let $d_1, d_2 \geq 0$ be such that $d_1+d_2 = d$. Then 
%there are constructible functions
%\begin{align*}
%g_{d_1} : \Sym^{d_1}(B^{\sm}) &\longrightarrow \ZZ(\!(p)\!) \\
%h_{d_2} : \Sym^{d_2}(B^{\sing}) &\longrightarrow \ZZ(\!(p)\!),
%\end{align*}
%such that for any $\sum_i a_i x_i \in \Sym^{d_1}(B^{\sm})$ and $\sum_j b_j y_j \in \Sym^{d_2}(B^{\sing})$ \todo{Forgot: points which can are off the curve.}
$$ 
f_d(\mathfrak{a} + \mathfrak{b}) =\frac{(1-q)^{e(B)}}{M(q)^{e(X\setminus B)}} \cdot f_{d_1}(\mathfrak{a}) \cdot f_{d_2}(\mathfrak{b}), 
$$
for any $\mathfrak{a} \in \Sym^{d_1}(B^{\sm})$ and $\mathfrak{b} \in \Sym^{d_2}(B^{\sing})$. 
\end{proposition}
We prove this proposition in Section \ref{formal}. The following product formula is an immediate consequence of this result
\begin{equation} \label{firstprod}
\sum_{d \geq 0} q^d \int_{\Sym^d(B)} f_d \ de = \frac{(1-q)^{e(B)}}{M(q)^{e(X\setminus B)}}  \Big( \sum_{d \geq 0} q^d \int_{\Sym^d(B^{\sm})} f_d \ de \Big) \cdot \Big( \sum_{d \geq 0} q^d \int_{\Sym^d(B^{\sing})} f_d \ de \Big). 
\end{equation}
The restricted constructible functions $f_d  : \Sym^d(B^{\sm}) \rightarrow \ZZ$ and $f_d  : \Sym^d(B^{\sing}) \rightarrow \ZZ$ satisfy further multiplicative properties:
\begin{proposition} \label{mult2}
There exist functions $g : \ZZ_{\geq 0} \rightarrow \ZZ(\!(p)\!)$ and $h : \ZZ_{\geq 0} \rightarrow \ZZ(\!(p)\!)$, such that $g(0)=1$, $h(0)=1$, and
\begin{align*}
f_{d}(\mathfrak{a}) &= \frac{M(q)^{e(X\setminus B)}}{(1-q)^{e(B)}} \cdot \prod_{i} g(a_i), \\
f_{d}(\mathfrak{b}) &= \frac{M(q)^{e(X\setminus B)}}{(1-q)^{e(B)}} \cdot \prod_{j} h(b_j), 
\end{align*}
for all $\mathfrak{a} = \sum_i a_i x_i \in \Sym^{d}(B^{\sm})$ and $\mathfrak{b} = \sum_j b_j y_j \in \Sym^{d}(B^{\sing})$.
\end{proposition}
We prove this proposition in \ref{}. Together with Lemma \ref{lem: formula for euler char of sym products} from the appendix, Proposition \ref{mult2} and \eqref{firstprod} gives\todo{Incorporate lemma appendix in main text?}
$$
\sum_{d \geq 0} q^d \int_{\Sym^d(B)} f_d \ de = \frac{M(q)^{e(X\setminus B)}}{(1-q)^{e(B)}} \cdot \Big( \sum_{a=0}^{\infty} g(a) q^a \Big)^{e(B) - N} \cdot \Big( \sum_{b=0}^{\infty} h(b) q^b \Big)^N.
$$
We want to prove Lemmas \ref{mult1}, \ref{mult2} and find formulae for $g(a)$, $h(b)$. This requires a better understanding of the strata
$$
\rho_{d}^{-1} (\mathfrak{a} + \mathfrak{b}) \subset \Hilb^{d, \bullet}(X)^{\CC^*},
$$
for all $\mathfrak{a} = \sum_i a_i x_i \in \Sym^{d_1}(B^{\sm})$ and $\mathfrak{b} = \sum_j b_j y_j \in \Sym^{d_2}(B^{\sing})$ with $d_1+d_2=d$. We start with a more careful study of the $\CC^*$-fixed locus.




\section{The $\CC^*$-fixed locus} \label{fixedlocus}

As we already noted, the scaling action of $\CC^*$ on the fibres of $X$ lifts to the moduli space $\Hilb^{d,\bullet}(X)$. 
%Therefore
%$$
%e(\Hilb^{d,\bullet}(X)) = e(\Hilb^{d,\bullet}(X)^{\CC^*}).
%$$
Therefore, we only need to restrict attention to $\Hilb^{d,\bullet}(X)^{\CC^*}$. 

Using the map $\pi : X \rightarrow S$, a quasi-coherent sheaf on $X$ can be viewed as a quasi-coherent sheaf $\F$ on $S$ together with a morphism $\F \otimes K_{S}^{-1} \rightarrow \F$. A $\CC^*$-equivariant structure on $\F$ translates into a $\ZZ$-grading
$$
\pi_* \F = \bigoplus_{k \in \ZZ} \F_k,
$$
such that $\F \otimes K_{S}^{-1} \rightarrow \F$ is graded, i.e.
$$
\F_k \otimes K_{S}^{-1} \longrightarrow \F_{k-1}.
$$
The structure sheaf $\O_X$ corresponds to 
$$
\pi_* \O_X = \bigoplus_{k=0}^{\infty} K_{S}^{-k}.
$$
Therefore a $\CC^*$-fixed morphism $\F \rightarrow \O_X$ corresponds to a graded sheaf $\F$ as above together with maps
\begin{displaymath}
\xymatrix
{
\cdots & \F_1 & \oplus & \F_0 \ar[d] & \oplus & \F_{-1} \ar[d] & \oplus & \F_{-2} \ar[d] & \cdots \\
& & & \O_S & \oplus & K_{S}^{-1} & \oplus & K_{S}^{-2} & \cdots 
}
\end{displaymath}
It is useful to re-define $\G_k := \F_{-k} \otimes K_{S}^{k}$. Then the data of a $\CC^*$-fixed morphism $\F \rightarrow \O_X$ is equivalent to the following data:
\begin{itemize}
\item coherent sheaves $\{\G_k\}_{k \in \ZZ}$ on $S$,
\item morphisms $\{\G_k \rightarrow \G_{k+1}\}_{k \in \ZZ}$,
\item morphisms $\G_k \rightarrow \O_S$ such that the following diagram commutes:
\end{itemize}
\begin{displaymath}
\xymatrix
{
\cdots & \G_{-1} \ar[r] & \G_0 \ar[r] \ar[d] & \G_{1} \ar[r] \ar[d] & \G_{2} \ar[r] \ar[d] & \cdots \\
& & \O_S \ar@{=}[r]& \O_S \ar@{=}[r] & \O_S \ar@{=}[r] & \cdots 
}
\end{displaymath}
In the case of interest to us $\G \rightarrow \O_X$ is an ideal sheaf $I_Z \hookrightarrow \O_X$ cutting out $Z \subset X$. In the above language, this means $\G_k = 0$ for $k<0$, the morphisms $\G_k \rightarrow \O_S$ are injective (hence $\G_k = I_{Z_k}$ is an ideal sheaf cutting out $Z_k \subset S$), and the morphisms $\G_k \rightarrow \G_{k+1}$ are injective (hence $I_{Z_k} \subset I_{Z_{k+1}}$, i.e.~$Z_{k} \supset Z_{k+1}$). We conclude:
\begin{proposition}
A closed point $Z$ of $\Hilb^{d,\bullet}(X)^{\CC^*}$ corresponds to a finite nesting of closed subschemes of $S$
$$
Z_{0} \supset Z_{1} \supset \cdots \supset Z_{l},
$$
for some $l \geq 0$, such such that
$$
\sum_{k=0}^{l} [Z_k] = B + dF \in H_2(S).
$$
\end{proposition}

In the above proposition, each $Z_k$ contains a maximal Cohen-Macaulay subcurve $D_k$ such that $Z_k \setminus D_k$ is 0-dimensional. For $k=0$, $D_0$ is the scheme-theoretic union of the section $B$ and thickenings of certain distinct fibres $F_{x_1}$, $\ldots$, $F_{x_n}$. Denoting the orders of thickenings by $\lambda_{1}^{(1)}, \ldots, \lambda_{1}^{(n)} > 0$, we obtain\footnote{For any reduced curve $C$ on a surface $S$ with ideal sheaf $I_C \subset \O_S$ and $d>0$, we denote by $d C$ the scheme defined by the ideal sheaf $I_{C}^{d} \subset \O_S$.}
$$
D_0 = B \cup \lambda_{0}^{(1)} F_{x_1} \cup \cdots \cup \lambda_{0}^{(n)} F_{x_n}.
$$
This statement follows from Corollary \ref{cor: chow(beta) = sym (B)} of the appendix. Next, for all $i = 1, \ldots, n$ and $k \geq 1$, there are $\lambda_{k}^{(i)} \leq \lambda_{k-1}^{(i)}$ such that
$$
D_k = \lambda_{k}^{(1)} F_{x_1} \cup \cdots \cup \lambda_{k}^{(n)} F_{x_n}.
$$
We conclude:
\begin{proposition}
To each closed point $Z$ of $\Hilb^{d,\bullet}(X)^{\CC^*}$ correspond distinct closed points $x_1, \ldots, x_n \in B$ for some $n$ and (finite) 2D partitions $\lambda^{(1)}, \ldots, \lambda^{(n)}$ such that
$$
\sum_{i=1}^{n} |\lambda^{(i)}| = d.
$$
The maximal Cohen-Macaulay subcurve of $Z$ is given by the scheme-theoretic union of the zero section $B$ and the schemes with ideal sheaves
\begin{equation} \label{CMcurve}
\bigoplus_{k=0}^{\infty} \O_{S}(-\lambda_{k}^{(i)} F_{x_i}) \otimes K_{S}^{-k},
\end{equation}
for all $i = 1, \ldots, n$.
\end{proposition}
%This leads to the following proposition:
%\begin{proposition} \label{C^*}
%For each $h>0$, there exists a stratification
%$$
%\Hilb^{h,\bullet}(X)^{\CC^*} = \coprod_{n=1}^{\infty} \coprod_{{\scriptsize{\begin{array}{c} \lambda^{(1)}, \ldots, \lambda^{(n)} \mathrm{s.t.} \\ \sum_{\alpha=1}^{n} |\lambda^{(\alpha)}| = h \end{array}}}} \Hilb^{h,\bullet}_{\lambda^{(1)}, \ldots, \lambda^{(n)}}(X)^{\CC^*},
%$$
%where $\Hilb^{h,\bullet}_{\lambda^{(1)}, \ldots, \lambda^{(n)}}(X)^{\CC^*}$ is the locally closed subset of subschemes $Z \subset X$ with maximal Cohen-Macaulay curve defined by the scheme-theoretic union of $B$ and schemes with ideal sheaves of the form \eqref{CMcurve} for some distinct fibre $F_{x_1}, \ldots, F_{x_n} \subset S$.
%\end{proposition}

%In this proposition, the number of points $n$ and the position of the points $x_1, \ldots, x_n \in B$ can still vary freely. We now want to refine the stratification by fixing the position of these points.

In the previous section, we considered the stratum
$$
\rho_{d}^{-1} ( \mathfrak{a} + \mathfrak{b} ) \subset \Hilb^{d, \bullet}(X)^{\CC^*},
$$
for any $\mathfrak{a} = \sum_{i=1}^{m} a_i x_i \in \Sym^{d_1}(B^{\sm})$ and $\mathfrak{b} = \sum_{j=1}^{n} b_j y_j \in \Sym^{d_2}(B^{\sing})$ satisfying $d_1+d_2=d$. By the previous lemma, it has a further stratification into locally closed subsets
\begin{equation} \label{comps}
\coprod_{{\scriptsize{\begin{array}{c} \lambda^{(1)} \vdash a_1 \\ \cdots \\ \lambda^{(m)} \vdash a_m \end{array}}}} \coprod_{\scriptsize{\begin{array}{c} \mu^{(1)} \vdash b_1 \\ \cdots \\ \mu^{(m)} \vdash b_m \end{array}}} \Sigma(x_1, \ldots, x_m, y_1, \ldots, y_n, \lambda^{(1)}, \ldots, \lambda^{(m)}, \mu^{(1)}, \ldots, \mu^{(n)}).
\end{equation}
We abbreviate a stratum on the RHS by $\Sigma(\boldsymbol{x};\boldsymbol{y};\boldsymbol{\lambda};\boldsymbol{\mu})$. It should be viewed as the stratum of $Z \in \Hilb^{d,\bullet}(X)^{\CC^*}$ for which the maximal CM subcurve $C \subset Z$ has been fixed by the data $\boldsymbol{x}, \boldsymbol{y}, \boldsymbol{\lambda}, \boldsymbol{\mu}$. We are after the Euler characteristics of these strata. We will see this Euler characteristic does \emph{not} depend on the exact location of the points $x_i$ and $y_j$, but only on their number $m$ and $n$ and the partitions $\lambda^{(i)}$ and $\mu^{(j)}$.


\section{Restriction to formal neighbourhoods} \label{formal}

In the previous two sections, we reduced our consideration to the stratum $\Sigma(\boldsymbol{x};\boldsymbol{y};\boldsymbol{\lambda};\boldsymbol{\mu})$ of $Z \in \Hilb^{d,\bullet}(X)^{\CC^*}$ for which the maximal CM subcurve $C \subset Z$ is fixed by the data $\boldsymbol{x}, \boldsymbol{y}, \boldsymbol{\lambda}, \boldsymbol{\mu}$. In this section, we further break down this stratum by cutting it up in pieces covered by formal neighbourhoods. For notational simplicity, we first consider the case where the base point is 
$$
a x + b y \in \Sym^2(B),
$$
where $x \in B^{\sm}$ and $y \in B^{\sing}$ and compute the Euler characteristic of $\Sigma(x,y,\lambda,\mu)$. Once this is eastablished, it is not hard to calculate the Euler characteristic of any stratum  $\Sigma(\boldsymbol{x};\boldsymbol{y};\boldsymbol{\lambda};\boldsymbol{\mu})$. This leads to a proof of Propositions \ref{mult1}, \ref{mult2}, and a geometric characterization of the functions $g(a)$, $h(b)$ of Section \ref{sym}.


\subsection{Fpqc cover}

The idea is to use an appropriate cover of $X$ and calculate on pieces of the cover. We first give a complex analytic definition of the cover to aid the intuition and then give the actual ``algebro-geometric cover'': 
\begin{itemize}
\item The reduced support $B \cup F_x \cup F_y$ has three singular points: $x,y, F_{y}^{\sing}$. We take small balls $U_1, U_2, U_3$ around these points.
\item Consider the punctured space $X \setminus \{x,y, F_{y}^{\sing}\}$ and the following three disjoint closed curves in it $B \setminus \{x,y\}$, $F_x \setminus \{x\}$, $F_y \setminus \{y,F_{y}^{\sing}\}$. We take small tubular neighbourhoods $V_1, V_2, V_3$ of each of these curves. Note how the $U_i$ and $V_i$ together cover the reduced support  $B \cup F_x \cup F_y$.
\item Finally we take $W = X \setminus (B \cup F_x \cup F_y)$. 
\end{itemize}

In order to work in algebraic geometry, we take $U_1$ to be the formal neighbourhood of $\{x\}$ in $X$. If $R$ is the local ring at $x$, then we mean by $U_1$ the actual (non-noetherian) scheme
$$
\Spec( \varprojlim R / \mathfrak{m}^n )
$$
and \emph{not} the formal scheme $\mathrm{Spf} (\varprojlim R / \mathfrak{m}^n)$. Similarly, we take $U_2$ to be the formal neighbourhood of $\{y\}$ on $X$ and $U_3$ the formal neighbourhood of $F_{y}^{\sing}$ in $X$. Note that as formal schemes
$$
U_i \cong\Spec( \CC[\![x_1,x_2,x_3]\!]).
$$
The point of taking spec is that the maps $U_i \rightarrow X$ is an fpqc morphism \cite{Vis}. Flatness follows from the fact that localization and formal completion are flat operations \cite[Cor.~3.6, Prop.~10.12, Prop.~10.13]{AM}. 

Next, in the punctured space $X \setminus \{x,y,F_{y}^{\sing}\}$, we take $V_1$ to be the formal neighbourhood of $B \setminus \{x,y\}$. Similarly, we take $V_2$ to be the formal neighbourhood of $F_x \setminus \{x\}$ in $X \setminus \{x,y,F_{y}^{\sing}\}$ and $V_3$ the formal neighbourhood of $F_y \setminus \{x, F_{y}^{\sing}\}$ in $X \setminus \{x,y,F_{y}^{\sing}\}$. Again, formal neighbourhoods are meant in the sense of taking spec of of inverse limits as above. Finally, $W = X \setminus (B \cup F_x \cup F_y)$. Now
$$
\mathfrak{U} = \{U_1 \rightarrow X, U_2 \rightarrow X, U_3 \rightarrow X, V_1 \rightarrow X, V_2 \rightarrow X, V_3 \rightarrow X, W \subset X\}
$$
forms an fpqc cover of $X$. This means the data of a coherent sheaf on $X$ is equivalent to the data of coherent sheaves on each of the open of $\mathfrak{U}$ together gluing isomorphisms on overlaps. Technically: quasi-coherent sheaves on $X$ form a stack with respect to the fpqc topology \cite[Thm.~4.23]{Vis}.


\subsection{Local moduli spaces} \label{localmod}

We now introduce moduli spaces of closed subschemes of dimension $\leq 1$ on each of the pieces of the cover $\mathfrak{U}$. Consider $U_1 \cong \Spec(\CC[\![ x_1,x_2,x_3]\!])$ and assume the coordinates are chosen such that $x_2=x_3=0$ corresponds to the intersection of $U_1 \times_X B$ and $x_1=x_3=0$ corresponds to $U_1 \times_X F_x$. Define
$$
\Hilb^{(1,d),n}(U_1) = \big\{ I_Z \subset \O_{U_1} \ : \ [Z] = [U_1 \times_X B] + d [U_1 \times_X F_x] \ {\rm{and}} \ h^0(I_{Z_{CM}} / I_Z) = n \big\}.
$$
Here the equation
$$
[Z] = [U_1 \times_X B] + d [U_1 \times_X F_x]
$$
means $Z$ is supported along $(U_1 \times_X B) \cup (U_1 \times_X F_x)$ with multiplicity 1 along $(U_1 \times_X B)$ and multiplicity $d$ along $(U_1 \times_X F_x)$. Furthermore, $Z_{CM}$ denotes the maximal Cohen-Macaulay subcurve of $Z$ which fits in a short exact sequence
$$
0 \longrightarrow I_{Z} \longrightarrow I_{Z_{CM}} \longrightarrow Q \longrightarrow 0, 
$$
where $Q$ is a 0-dimensional scheme. The Hilbert scheme $\Hilb^{(1,d),n}(U_2)$ is defined likewise replacing the point $x$ by $y$. Recall that $x,y$ are intersections of $B$ with the fibres $F_x$, $F_y$. For the point $F_{y}^{\sing}$ and its formal neighbourhood $U_3$ we define
$$
\Hilb^{d,n}(U_3) = \big\{ I_Z \subset \O_{U_3} \ : \ [Z] = d [U_3 \times_X F_y] \ {\rm{and}} \ h^0(I_{Z_{CM}} / I_Z) = n \big\}.
$$
Each $U_i$ has an action of $\CC^*$ compatible with the fibre scaling on $X$ and this action lifts to the moduli space. Moreover, since $U_i \cong \Spec(\CC[\![x_1,x_2,x_3]\!])$ the bigger torus $\CC^{*3}$ acts on $U_i$ and lifts to the moduli space. The existence of these ``extra actions'' are one of the main tricks in our paper and will be used in Section \ref{}.

Next consider $V_1$, the formal neighbourhood of the punctured zero section $B \subset X$. Define
$$
\Hilb^{1,n}(V_1) = \big\{ I_Z \subset \O_{V_1} \ : \ [Z] = [V_1 \times_X B] \ {\rm{and}} \ h^0(I_{Z_{CM}} / I_Z) = n \big\}.
$$
For $V_2$, $V_3$ we define
$$
\Hilb^{d,n}(V_i) = \big\{ I_Z \subset \O_{V_i} \ : \ [Z] = d [V_i \times_X B] \ {\rm{and}} \ h^0(I_{Z_{CM}} / I_Z) = n \big\}.
$$
Finally, for $W$ we define
$$
\Hilb^{0,n}(W) = \big\{ I_Z \subset \O_{V_1} \ : \ \dim(Z) = 0 \ {\rm{and}} \ h^0(\O_Z) = n \big\}.
$$
Each $V_i$, $W$ also has an action of $\CC^*$ compatible with the fibre scaling on $X$. These actions lift to the moduli space. However, \emph{unlike} the $U_i$, no additional tori act on $V_i$, $W$.

As before, we use the notation $\Hilb^{(1,d),\bullet}(U_1)$ for the union of all $\Hilb^{(1,d),n}(U_1)$ (and similar for all other moduli spaces of this section). Like in Section \ref{fixedlocus}, the components of the $\CC^*$-fixed locus of $\Hilb^{(1,d),\bullet}(U_1)$ are indexed by partitions $\lambda \vdash d$
$$
\Hilb^{(1,d),\bullet}(U_1)^{\CC^*} = \coprod_{\lambda \vdash d} \Hilb^{(1,d),\bullet}(U_1)_{\lambda}^{\CC^*}.
$$

\begin{proposition} \label{bij}
Consider the stratum $\Sigma(x,y,a,b)$, where $a = |\lambda|$ and $b = |\mu|$. Restriction from $X$ to the open subsets of the cover $\mathfrak{U}$ induces a morphism
\begin{align}
\begin{split} \label{restr}
\Sigma(x,y,\lambda,\mu) \longrightarrow &\Hilb^{(1,a),\bullet}(U_1)_{\lambda}^{\CC^*} \times \Hilb^{(1,b),\bullet}(U_2)_{\mu}^{\CC^*} \times \Hilb^{b,\bullet}(U_3)_{\mu}^{\CC^*} \times \\
&\Hilb^{1,\bullet}(V_1)^{\CC^*} \times \Hilb^{a,\bullet}(V_2)_{\lambda}^{\CC^*} \times \Hilb^{b,\bullet}(V_3)_{\mu}^{\CC^*} \times \\
&\Hilb^{0,\bullet}(W)^{\CC^*},
\end{split}
\end{align}
which is a bijection on closed points.
\end{proposition}
\begin{proof}
Throughout this proof, we work on closed points only.

Restriction along $U_i \rightarrow X$, $V_i \longrightarrow X$, $W \subset X$ gives the map set-theoretically. For any reduced base $B$, restriction along $U_i \times B \rightarrow X \times B$, $V_i \times B \longrightarrow X \times B$, $W \times B \subset X \times B$ gives a map between moduli functors and hence induces the above morphism.

Since $\mathfrak{U}$ is an fpqc cover, fpqc descent implies that any ideal sheaf $I_Z \subset \O_X$ is entirely determined by its restriction along $U_i \rightarrow X$, $V_i \longrightarrow X$, $W \subset X$ proving injectivity.

Conversely, given local ideal sheaves in the image of \eqref{restr}, their restrictions to overlaps only depend on the underlying Cohen-Macaulay curve (and not on the embedded points). Since we chose the strata such that the underlying Cohen-Macauly curve automatically glues, there are no further gluing conditions and fpqc descent implies the morphism is surjective.
\end{proof}
   
\begin{remark}
Note that the argument of Proposition \ref{bij} is purely set-theoretic in nature and no statements about the scheme structure are made. 
\end{remark}

Proposition \ref{bij} allows us to calculate the Euler characteristic of the stratum $\Sigma(x,y,\lambda,\mu)$. Recall that $a = |\lambda|$, $b=|\mu|$ so $d=a+b$. Then by \eqref{comps} and Proposition \ref{bij}
\begin{align*}
&f_d(ax+by) = e(\rho_{d}^{-1}(ax+by)) = e(\Hilb^{1,\bullet}(V_1)^{\CC^*}) e(\Hilb^{0,\bullet}(W)^{\CC^*}) \times \\
&\sum_{\lambda \vdash a} \sum_{\mu \vdash b} e(\Hilb^{(1,a),\bullet}(U_1)_{\lambda}^{\CC^*}) e(\Hilb^{(1,b),\bullet}(U_2)_{\mu}^{\CC^*}) e(\Hilb^{b,\bullet}(U_3)_{\mu}^{\CC^*}) e(\Hilb^{a,\bullet}(V_2)_{\lambda}^{\CC^*}) e(\Hilb^{b,\bullet}(V_3)_{\mu}^{\CC^*}).
\end{align*}
For any smooth quasi-projective 3-fold $W$, J.~Cheah proves \ref{}
$$
e(\Hilb^{\bullet}(X)) = M(p)^{e(X)},
$$
where $\Hilb^{n}(X)$ denotes the Hilbert schemes of $n$ points on $X$. Moreover, for any smooth quasi-projective curve $C$, we have (Remark \ref{MacD})
$$
e(\Hilb^{\bullet}(C)) = \frac{1}{(1-q)^{e(C)}},
$$
where $\Hilb^{n}(C)$ denotes the Hilbert schemes of $n$ points on $C$. Rewriting gives
\begin{align*}
f_d(ax+by) = \frac{M(q)^{e(X \setminus B)}}{(1-q)^{e(B)}} &\Big( (1-q) M(q) \sum_{\lambda \vdash a} e(\Hilb^{(1,a),\bullet}(U_1)_{\lambda}^{\CC^*}) e(\Hilb^{a,\bullet}(V_2)_{\lambda}^{\CC^*}) \Big) \times \\
&\Big( (1-q) \sum_{\mu \vdash b} e(\Hilb^{(1,b),\bullet}(U_2)_{\mu}^{\CC^*}) e(\Hilb^{b,\bullet}(U_3)_{\mu}^{\CC^*}) e(\Hilb^{b,\bullet}(V_3)_{\mu}^{\CC^*}) \Big).
\end{align*}

   
\subsection{Geometric characterization of $g(a)$ and $h(b)$}

The arguments of the preceding two sections are immediately modified to any stratum $\Sigma(\boldsymbol{x};\boldsymbol{y};\boldsymbol{\lambda};\boldsymbol{\mu})$. Let $U$ be the formal neighbourhood of any point on $B \subset S \subset X$ and define $\Hilb^{(1,a),\bullet}(U)$ as in Section \ref{localmod}. Let $U'$ be the formal neighbourhood of the singular point of any singular fibre $F \subset S \subset X$ and define $\Hilb^{b,\bullet}(U')$ as in Section \ref{localmod}. Let $V$ be the formal neighbourhood of any smooth fibre $F \setminus B$ in $X \setminus B$ and define $\Hilb^{a,\bullet}(V)$ as in Section \ref{localmod}. Let $V'$ be the formal neighbourhood of any singular fibre $F \setminus (B \cup F^{\sing})$ in $X \setminus (B \setminus F^{\sing})$ and define $\Hilb^{b,\bullet}(V')$ as in Section \ref{localmod}. 

\begin{proposition}
For any $a,b>0$ define
\begin{align}
\begin{split} \label{gh}
g(a) &:= (1-p) M(p) \sum_{\lambda \vdash a} e(\Hilb^{(1,a),\bullet}(U)_{\lambda}^{\CC^*}) e(\Hilb^{a,\bullet}(V)_{\lambda}^{\CC^*}) \in \ZZ[\![p]\!], \\
h(b) &:= (1-p) \sum_{\mu \vdash b} e(\Hilb^{(1,b),\bullet}(U)_{\mu}^{\CC^*}) e(\Hilb^{b,\bullet}(U')_{\lambda}^{\CC^*}) e(\Hilb^{b,\bullet}(V')_{\lambda}^{\CC^*})  \in \ZZ[\![p]\!],
\end{split}
\end{align}
and let $g(0) := 1$, $h(0) :=1$. Then
\begin{align*}
f_{d}(\mathfrak{a} + \mathfrak{b}) &= \frac{M(q)^{e(X\setminus B)}}{(1-q)^{e(B)}} \cdot \prod_{i} g(a_i) \cdot \prod_{j} h(b_j), 
\end{align*}
for any $\mathfrak{a} = \sum_i a_i x_i \in \Sym^{d}(B^{\sm})$ and $\mathfrak{b} = \sum_j b_j y_j \in \Sym^{d}(B^{\sing})$.
\end{proposition}
   
We immediately deduce:  
\begin{corollary}
Propositions \ref{mult1}, \ref{mult2} are true for $g(a)$ and $h(b)$ defined by \eqref{gh}.
\end{corollary}   
   
   
\section{Reduction to the topological vertex}   

In this section, we prove Theorem \ref{main} of the introduction by expressing $g(a)$ and $h(b)$ in terms of the topological vertex. 
   
   

\appendix
\section{Odds and Ends}\label{appendix: odds and ends}


\subsection{Weighted Euler characteristics of symmetric products}

In this section we prove the following formula for the weighted Euler
characteristic of symmetric products.

\begin{lemma}\label{lem: formula for euler char of sym products}
Let $S$ be a scheme of finite type over $\CC $ and let $e (S)$ be its
topological Euler characteristic. Let $g:\ZZ _{\geq 0}\to \ZZ ((Q))$
be any function with $g (0)=1$. Let $f_{d}:\Sym ^{d} (S)\to \ZZ ((Q))$
be the constructible function defined by $f_{d} (\sum_{i}
a_{i}x_{i})=\prod _{i}g (a_{i})$. Then
\[
\sum _{d=0}^{\infty } u^{d} \int _{\Sym ^{d} (S)} f_{d} de =
\left(\sum _{a=0}^{\infty }g (a) u^{a} \right)^{e (S)}.
\]
\end{lemma}

\begin{remark} \label{MacD}
In the special case where $g=f_{d}\equiv  1$, the lemma recovers
MacDonald's formula: $\sum _{d=1}^{\infty }e (\Sym ^{d} (S)) u^{d} =
(1-u)^{-e (S)}$. 

The lemma is essentially a consequence of the existence of a power
structure on the Grothendieck group of varieties definited by
symmetric products and the compatibility of the Euler characteristic
homomorphism with that power structure \cite{}. For convenience's
sake, we provide a direct proof here.
\end{remark}
\begin{proof}
The $d$th symmetric product admits a stratification with strata
labelled by partitions of $d$. Associated to any partition of $d$ is a
unique tuple $(m_{1},m_{2},\dots )$ of non-negative integers with
$\sum _{j=1}^{\infty }j m_{j}=d$. The stratum labelled by
$(m_{1},m_{2},\dots )$ parameterizes collections of points where there
are $m_{j}$ points of multiplicity $j$. The full stratification is
given by:
\[
\Sym ^{d} (S) = \bigsqcup_{\begin{smallmatrix} (m_{1},m_{2},\dots )\\
\sum _{j=1}^{\infty }j m_{j}=d  \end{smallmatrix}} \left\{\left(\prod _{j=1}^{\infty }S^{m_{j}} \right) -\Delta  \right\}/ \prod _{j=1}^{\infty }\sigma _{m_{j}} 
\]
where by convention, $S^{0}$ is a point, $\Delta $ is the large
diagonal, and $\sigma _{m}$ is the $m$th symmetric group. Note that
the function $f_{d}$ is constant on each stratum and has value $\prod
_{j=1}^{\infty }g (j)^{m_{j}}$. Note also that the action of $\prod
_{j=1}^{\infty }\sigma _{m_{j}}$ on each stratum is free. 

For schemes over $\CC $, topological Euler characteristic is additive
under stratification and multiplicative under maps which are
(topological) fibrations. Thus
\[
\int _{\Sym ^{d} (S)} f_{d}\,\, de = \sum _{\begin{smallmatrix}(m_{1},m_{2},\dots )\\
\sum _{j=1}^{\infty }j m_{j}=d   \end{smallmatrix}} \left(\prod _{j=1}^{\infty } g (j)^{m_{j}} \right) \frac{e (S^{\sum _{j}m_{j}}-\Delta )}{m_{1}!\, m_{2}!\, m_{3}!\dots }.
\]

For any natural number $N$, the projection $S^{N}-\Delta \to
S^{N-1}-\Delta $ has fibers of the form $S-\{N-1\text{ points}
\}$. The fibers have constant Euler characteristic given by $e (S)-
(N-1)$ and consequently, $e (S^{N}-\Delta )= (e (S)- (N-1))\cdot e
(S^{N-1}-\Delta )$. Thus by induction, we find $e (S^{N}-\Delta ) = e
(S)\cdot (e (S)-1)\cdots (e (S)- (N-1))$ and so 
\[
\frac{e (S^{\sum _{j}m_{j}}-\Delta )}{m_{1}!\,m_{2}!\,m_{3}!\cdots } = \binom{e (S)}{m_{1},m_{2},m_{3},\cdots }
\]
where the right hand side is the generalized multinomial coefficient.

Putting it together and applying the generalized multinomial theorem,
we find
\begin{align*}
\sum _{d=0}^{\infty } \int _{\Sym ^{d} (S)}f_{d}\,\,de & = \sum _{(m_{1},m_{2},\dots )} \prod _{j=1}^{\infty } \left(g (j) u^{j} \right)^{m_{j}} \binom{e (S)}{m_{1},m_{2},m_{3},\dots }\\
&=\left(1+\sum _{j=1}^{\infty }g (j) u^{j} \right)^{e (S)}
\end{align*}
which proves the lemma.   
\end{proof}

\subsection{Some geometry of curves on elliptic surfaces}\label{subsec: geometry of curves on elliptic surfaces}

In this subsection we prove the following lemma and corollary, which will tell us what is the reduced support of all curves in the class $\beta =B+dF$.

\begin{lemma}\label{lem: H0 (pi* (D) (B))=H0 (pi* (D))}
For any line bundle $\epsilon $ on $B$, multiplication by the
canonical section of $\O (B)$ induces an isomorphism
\[
H^{0} (S,\pi ^{*} (\epsilon ) (B)) \cong H^{0} (S,\pi ^{*} (\epsilon )).
\]
\end{lemma}

\begin{corollary} \label{cor: chow(beta) = sym (B)}
Let $\beta = B+dF \in H_{2} (S)$. Then the Chow variety of curves in
the class $\beta $ is isomorphic to $\Sym ^{d} (B)$ where a point
$\sum _{i}d_{i} x_{i}\in \Sym ^{d} (B)$ corresponds to the curve
$B+\sum _{i}d_{i} F_{x_{i}}$.
\end{corollary}

\begin{proof}
The corollary follows immediately from the lemma since the Chow
variety is the space of effective divisors and the lemma implies that
any effective divisor in the class $\beta $ is a union of the section
$B$ with an effective divisor pulled by from the base.

To prove lemma~\ref{lem: H0 (pi* (D) (B))=H0 (pi* (D))} we proceed as
follows. For any line bundle $\delta $ on $B$, the Leray spectral
sequence yields the short exact sequence:
\[
0\to H^{0} (B,\delta \otimes R^{1}\pi _{*}\O )\to H^{1} (S,\pi ^{*}\delta )\Rt{\alpha } H^{1} (B,\delta )\to 0,
\]
in particular, $\alpha $ is a surjection.

Then the long exact cohomology sequence associated to 
\[
0\to \pi ^{*}\delta \otimes \O (-B)\to \pi ^{*}\delta \to \O _{B}\otimes \pi ^{*}\delta \to 0
\]
is
\[
\dotsb \to H^{1} (S,\pi ^{*} (\delta ))\rt{\alpha }H^{1} (B,\delta )\to H^{2} (S,\pi ^{*}\delta \otimes \O (-B))\to H^{2} (S,\pi ^{*}\delta )\to 0,
\]
and since $\alpha $ is a surjection, we get an isomorphism of the last
two terms. We apply Serre duality to that isomorphism and we use the
fact that $K_{S} = \pi ^{*} (K_{B}\otimes L)$ where $L = \left(R\pi
_{*}\O _{S} \right)^{\vee }$ \cite[prop?]{Fr-Mo} to obtain
\[
H^{0} (S,\pi ^{*} (\delta ^{-1}\otimes K_{B}\otimes L) (B)) \cong H^{0}(S,\pi ^{*} (\delta ^{-1}\otimes K_{B}\otimes L)).
\]
Letting $\delta =K_{B}\otimes L\otimes \epsilon ^{-1}$, the lemma is
proved.
\end{proof}


     
\bibliography{mainbiblio}
\bibliographystyle{plain}

\end{document}

