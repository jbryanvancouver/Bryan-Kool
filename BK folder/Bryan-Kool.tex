\documentclass{amsart}

\title[DT invariants of local elliptic surfaces]{Donaldson-Thomas invariants of local elliptic surfaces via the topological vertex}
\author{Jim Bryan and Martijn Kool}
\date{\today}
\address{
Department of Mathematics\\
University of British Columbia \\
Room 121, 1984 Mathematics Road  \\
Vancouver, B.C., Canada V6T 1Z2  
}

\address{
Mathematical Institute \\
Utrecht University \\
Room 502, Budapestlaan 6  \\
3584 CD Utrecht, The Netherlands  
}


%\usepackage{diagrams}
%\usepackage{eepic,epic}

\usepackage{amsmath}
\usepackage{amsmath,amsthm,amsfonts}
\usepackage{amssymb}
\usepackage{times}
\usepackage[all]{xy}
%\usepackage{amstex}
\usepackage[colorinlistoftodos]{todonotes}



%\newtheorem{thm}{Theorem}%[section]
\newtheorem{theorem}{Theorem}%[section]
\newtheorem{proposition}[theorem]{Proposition}
\newtheorem{conjecture}[theorem]{Conjecture}
\newtheorem{lemma}[theorem]{Lemma}
\newtheorem{corollary}[theorem]{Corollary}
\theoremstyle{definition}

\newtheorem{def-theorem}[theorem]{Definition-Theorem}
\newtheorem{remark}[theorem]{Remark}
\newtheorem{definition}[theorem]{Definition}
\newtheorem{example}[theorem]{Example}


\newcommand{\CC} {\mathbb{C}}          % complex numbers
\newcommand{\NN} {\mathbb{N}}		% natural numbers
\newcommand{\RR} {\mathbb{R}}		% real numbers
\newcommand{\ZZ} {\mathbb{Z}}		% integers
\newcommand{\QQ} {\mathbb{Q}}		% rationals
\newcommand{\PP} {\mathbb{P}}
\renewcommand{\AA} {\mathbb{A}}
\newcommand{\LL} {\mathbb{L}}
\newcommand{\FF} {\mathbb{F}}
\renewcommand{\O}{\mathcal{O}}
\newcommand{\Y}{\mathcal{Y}}
\newcommand{\sfV}{\mathsf{V}}
\newcommand{\Pic}{\mathrm{Pic}}


\newcommand{\rt}[1]{\stackrel{#1\,}{\rightarrow}}
\newcommand{\Rt}[1]{\stackrel{#1\,}{\longrightarrow}}
\newcommand\To{\longrightarrow}
\newcommand\into{\hookrightarrow}
\newcommand\Into{\ensuremath{\lhook\joinrel\relbar\joinrel\rightarrow}}
\newcommand\INTO{\ar@{^{(}->}[r]}
\newcommand\acts{\curvearrowright}


\newcommand{\Hom}{\operatorname{Hom}}
\newcommand{\Ker}{\operatorname{Ker}}
\newcommand{\End}{\operatorname{End}}
\newcommand{\GL}{\operatorname{GL}}
\newcommand{\Tr}{\operatorname{tr}}
\newcommand{\tr}{\operatorname{tr}}
\newcommand{\Coker}{\operatorname{Coker}}
\newcommand{\im}{\operatorname{Im}}
\newcommand{\M}{\overline{\mathcal{M}}}
\newcommand{\smargin}[1]{\marginpar{\tiny{#1}}}
\newcommand{\Sym}{\operatorname{Sym}}
\newcommand{\Coh}{\operatorname{Coh}}
\newcommand{\Hilb}{\operatorname{Hilb}}
\newcommand{\DT}{\operatorname{DT}}
\newcommand{\CM}{\operatorname{CM}}
\newcommand{\Var}{\operatorname{Var}}
\newcommand{\supp}{\operatorname{supp}}
\newcommand{\Spec}{\operatorname{Spec}}
\newcommand{\sm}{\operatorname{sm}}
\newcommand{\sing}{\operatorname{sing}}
\newcommand{\conn}{\operatorname{conn}}
\newcommand{\F}{\mathcal{F}}
\newcommand{\G}{\mathcal{G}}
\newcommand{\cP}{\mathcal{P}}
\newcommand{\cQ}{\mathcal{Q}}
\newcommand{\cR}{\mathcal{R}}

\begin{document}

\begin{abstract}
%Jim's original:

We compute the Donaldson-Thomas invariants of a local elliptic surface
with section. We introduce a new computational technique which is a
mixture of motivic and toric methods. This allows us to write the
partition function for the invariants in terms of the topological
vertex. Utilizing identities for the topological vertex (some
previously known, some new), we derive product formulas for the
partition functions in a companion paper with B.~Young. In the special case where the elliptic surface is
a K3 surface, we get a new proof of the Katz-Klemm-Vafa formula.
\end{abstract}

\maketitle 

%\markboth{???}  {???}
%\renewcommand{\sectionmark}[1]{}


%\tableofcontents
%\pagebreak


\section{Introduction}

Let $p : S \rightarrow B$ be an elliptic surface over a complex smooth projective curve $B$. We assume $p$ has a section and all singular fibres are rational nodal curves. Let $N$ be the number of singular fibres. Important examples are the rational elliptic surface or elliptic K3 surface for which $B = \PP^1$ and $N = 12$, $N=24$ respectively.

We are interested in the Donaldson-Thomas invariants of $X = \mathrm{Tot}(K_S)$, i.e.~the total space of the canonical bundle $K_S$. This is a non-compact Calabi-Yau 3-fold. Let $\beta$ be an effective curve class on $S$. Consider the Hilbert scheme 
$$
\Hilb^{\beta,n}(X) = \{ Z \subset X \ : \ [Z] = \beta, \ \chi(\O_Z) = n\}
$$
of proper subschemes $Z \subset X$ with homology class $\beta$ and holomorphic Euler characteristics $n$. The DT invariants of $X$ can be defined as
$$
\DT_{\beta,n}(X) := e(\Hilb^{\beta,n}(X), \nu) := \sum_{k \in \ZZ} k \ e(\nu^{-1}(\{k\})),
$$
where $e(\cdot)$ denotes topological Euler characteristic and $\nu : \Hilb^{\beta,n}(X) \rightarrow \ZZ$ is Behrend's constructible function \cite{Beh}. We consider an Euler characteristic version of these invariants
$$
\widehat{\DT}_{\beta,n}(X) := e(\Hilb^{\beta,n}(X)).
$$
We choose a section $B \subset S$ and focus on the primitive classes $\beta = B + dF$, where $B$ is the class of the chosen section and $F$ the class of the fibre. Let 
\begin{align*}
\widehat{\DT}(X) := \sum_{d \geq 0} \widehat{\DT}_d(X) q^d := \sum_{d \geq 0} \sum_{n \in \ZZ} \widehat{\DT}_{B+dF,n}(X) p^n q^d.
\end{align*}
The main result of this paper is a formula for this generating function, and its connected analog $\widehat{\DT}^{\conn}(X)$, in terms of the topological vertex $\sfV_{\lambda,\mu,\nu}(p)$, $e(B)$, and $N$. This formula can be found in Theorem \ref{main} of Section \ref{vertex}.

It turns out the formula of Theorem \ref{main} can be expressed in terms of traces of certain natural operators on Fock space. Some of these traces are known and others are new. The way to compute these traces is of independent interest and forms the subject of a companion paper with B.~Young \cite{BKY}. The outcome is the following. Consider the Dedekind eta function and the following Jacobi eta function \cite{Cha}
\begin{align*}
\eta(\tau) &= e^{\frac{\pi i \tau}{12}}\prod_{k=1}^{\infty}(1-q^k) \\
K(z,\tau) &= \frac{i \vartheta_1(z,\tau)}{\eta(\tau)^3} = (p^{\frac{1}{2}} - p^{-\frac{1}{2}}) \prod_{k=1}^{\infty} \frac{(1-p q^k) (1-p^{-1} q^k)}{(1-q^k)^2}
\end{align*}
expanded in the region $|p|, |q| <1$ where 
$$
p = e^{2 \pi i z}, \ q = e^{2 \pi i \tau}.
$$
\begin{theorem}[Bryan-Kool-Young] 
Let $g$ be the genus of $B$. Then for all $d > 2g-2$ the coefficient of $q^d$ of $\widehat{\DT}^{\conn}(X)$ is given by the coefficient of $q^d$ of \todo{Can we drop assumption on $d$?}
$$
\big(q^{-\frac{1}{24}}\eta(\tau) \big)^{-e(S)} K(z,\tau)^{-e(B)}.
$$
\end{theorem}
In the case $S \rightarrow \PP^1$ is the elliptically fibred K3 surface, this provides a new derivation of the famous Katz-Klemm-Vafa formula. The KKV formula was recently proved in \emph{all} curve classes in \cite{PT}. This is the first derivation of the KKV formula, which does \emph{not} depend on the Kawai-Yoshioka formula \cite{KY}.

The most important result of this paper is perhaps not the formula, but rather the method of calculation. This method has found further applications to the calculation of DT generating functions on $K3 \times E$, where $E$ is an elliptic curve \cite{Bry}, and abelian 3-folds \cite{BOPY}. Even though the geometry under consideration is not toric, we combine $\CC^*$-localization, motivic methods, formal methods, and $\CC^{*3}$-localization to end up with expressions that only depend on $\sfV_{\lambda,\mu,\nu}(p)$, $e(B)$, and $N$. Here is a rough sketch of our method:
\begin{enumerate}
\item[(A)] The action of $\CC^*$ on the fibres of $X$ lifts to the moduli space\footnote{The bullet indicates that we take the union of $\Hilb^{B+dF,n}(X)$ over all $n$.} $\Hilb^{B+dF,\bullet}(X)$. Therefore, we only have to understand $e(\Hilb^{B+dF,\bullet}(X))^{\CC^*}$. Push-forward along $X \rightarrow S \rightarrow B$ induces a morphism
\begin{equation} \label{intromap}
\Hilb^{B+dF,\bullet}(X)^{\CC^*} \rightarrow \Sym^d(B).
\end{equation}
This map is constructed in Section \ref{fixedlocus}. The fibres of \eqref{intromap} decompose into components according to the shape of the underlying Cohen-Macaulay curve. This leads to a decomposition over 2D partitions $\lambda = (\lambda_0 \geq \lambda_1 \geq \cdots)$.
\item[(B)] The Euler characteristics of the fibres of \eqref{intromap} give a constructible function $f_d$ on $\Sym^d(B)$. In Section \ref{sym}, we show that if $f_d$ satisfies a certain product formula, then $\widehat{\DT}(X)$ satisfies a corresponding product formula. This follows from general power structure arguments reviewed in Appendix \ref{power}.
\item[(C)] A component $\Sigma$ of a fibre of \eqref{intromap} indexed by $\lambda$ can be further broken down by taking a certain fpqc cover of the underlying (now fixed) Cohen-Macaulay curve $Z_{\CM}$ determined by $\lambda$. This cover consists of formal neighbourhoods $\widehat{X}_x$ around the singular points $x$ of the reduced support of $Z_{\CM}$ and ``tubular neighbourhoods'' along the reduced support of $Z_{\CM}$ after removing the singularities. Since $Z_{\CM}$ is already fixed, gluing is automatic. Hence restriction to the elements of the cover gives a set theoretic bijection of $\Sigma$ with Hilbert schemes on the elements of the cover. In Section \ref{formal}, we show these moves lead to the product formula for $f_d$ mentioned in (A).
\item[(D)] On the formal neighbourhoods $\widehat{X}_x$, we have an action of $\CC^{*3}$. This allows us to express their contributions to the generating function in terms of the topological vertex. The contributions of the tubular neighbourhoods along the \emph{punctured} section and fibres can also be expressed in terms of the topological vertex (roughly speaking, by utilizing actions of the elliptic curve $F$ or $\CC^*$ on itself). This is worked out in Section \ref{vertex}.
\end{enumerate}

Many of the methods of this paper work well with the Behrend function. In particular, steps (A), (B), and (D) do not provide any problems. In Section \ref{Behrend}... \todo{TBC: section on Behrend function.} \\

\noindent \textbf{Acknowledgements.} We would like to thank... J.B.~was supported by... M.K.~was supported by a PIMS postdoctoral fellowship (CRG geometry and physics), while at UBC where most of this work was done. M.K.~was supported by NWO-GQT and Marie-Sk{\l}odowska Curie IF 656898 while at Utrecht. \todo{Add people and funding.}


\section{Definitions}

Let $p : S \rightarrow B$ be an elliptic surface over a smooth projective curve $B$. Besides assuming $S$ is not a product, we require:
\begin{enumerate}
\item $p$ has a section $B \subset S$,
\item all singular fibres of $p$ are of Kodaira type $I_1$, i.e.~rational nodal curves. 
\end{enumerate}
We write $F_x$ for the fibre $p^{-1}(\{x\})$, for all closed points $x \in B$. We choose a section $B \subset S$ and denote its class in $H_2(S)$ by $B$ as well. We denote the class of the fibre by $F \in H_2(S)$. Interesting examples are the elliptic surfaces $E(n)$. Then $B = \PP^1$ and the number of nodal fibres is $12n$. For example, $E(1)$ is the rational elliptic surface and $E(2)$ is the elliptic K3 surface.

Let $\beta = B+dF \in H_2(S)$ for arbitrary $d \geq 0$. Denote by $X = \mathrm{Tot}(K_S)$ the total space of the canonical bundle over $S$. Then $X$ is a non-compact Calabi-Yau 3-fold. Consider the Hilbert scheme
$$
\Hilb^{\beta,n}(X) = \{ Z \subset X \ : \ [Z] = \beta, \ \chi(\O_Z) = n\}
$$
of proper subschemes $Z \subset X$ with fixed homology class and holomorphic Euler characteristic. K.~Behrend associates to any $\CC$-scheme of finite type $Y$ a constructible function $\nu : Y \rightarrow \ZZ$ \cite{Beh}. Applied to $\Hilb^{\beta,n}(X)$, the Donaldson-Thomas invariants of $X$ can be defined as\footnote{If $X$ is a \emph{compact} Calabi-Yau 3-fold, R.P.~Thomas's original definition of the DT invariant is the degree of the virtual cycle of $\Hilb^{\beta,n}(X)$ \cite{Tho}. Behrend showed this is the same as $e(\Hilb^{\beta,n}(X),\nu)$ \cite{Beh}. The advantage of the definition by means of virtual cycles is that the construction works relative to a base. This implies deformation invariance of the invariants.} 
$$
\DT_{\beta,n}(X) := \int_{\Hilb^{\beta,n}(X)} \nu \ de := \sum_{k \in \ZZ} k \ e(\nu^{-1}(\{k\})),
$$
where $e(\cdot)$ denotes topological Euler characteristic. Many of the key properties of DT invariants are already captured by the more classical Euler characteristic version\footnote{From the point of view of \cite{JS, Bri}, there are two natural integration maps on the semi-classical Hall-algebra. One corresponds to weighing by the Behrend function. The other corresponds to weighing by the ``trivial'' constant constructible function 1. The former gives rise to $\DT(X)$ and the latter to $\widehat{\DT}(X)$.}
$$
\widehat{\DT}_{\beta,n}(X) := \int_{\Hilb^{\beta,n}(X)} 1 \ de = e(\Hilb^{\beta,n}(X)).
$$
For brevity, we define
\begin{align*}
\Hilb^{d,n}(X) &:=\Hilb^{B+dF,n}(X), \\
\DT_{d,n}(X) &:= \DT_{B+dF,n}(X), \\
\widehat{\DT}_{d,n}(X) &:= \DT_{B+dF,n}(X).
\end{align*}
The generating functions of interest are
\begin{align*}
\DT(X) &:= \sum_{d \geq 0} \DT_d(X) q^d := \sum_{d \geq 0} \sum_{n \in \ZZ} \DT_{d,n}(X) p^n q^d, \\
\widehat{\DT}(X) &:= \sum_{d \geq 0} \widehat{\DT}_d(X) q^d := \sum_{d \geq 0} \sum_{n \in \ZZ} \widehat{\DT}_{d,n}(X) p^n q^d.
\end{align*}
Note that the corresponding connected series $\DT^{\conn}(X)$, $\widehat{\DT}^{\conn}(X)$ are obtained after dividing by
\begin{align*}
&\sum_{d \geq 0} \sum_{n \in \ZZ} e(\Hilb^{dF,n}(X),\nu) p^n q^d \\
&\sum_{d \geq 0} \sum_{n \in \ZZ} e(\Hilb^{dF,n}(X)) p^n q^d
\end{align*}
respectively.

Since we are dealing with generating functions and our calculations involve cut-paste methods on the moduli space, it is useful to introduce the following notation. We define
\begin{align*}
[\Hilb^{d,\bullet}(X)] := \sum_{n \in \ZZ} [\Hilb^{d,n}(X)] p^n,
\end{align*}
which is an element of $K_0(\Var_{\CC})(\!(p)\!)$, i.e.~a Laurent series with coefficients in the Grothendieck group of varieties. We also write $\Hilb^{d,\bullet}(X)$ to denote the union of all $\Hilb^{d,n}(X)$. Therefore $\Hilb^{d,\bullet}(X)$ is a $\CC$-scheme locally of finite type.


\section{The $\CC^*$-fixed locus} \label{fixedlocus}

The action of $\CC^*$ on the fibres of $X$ lifts to the moduli space $\Hilb^{d,\bullet}(X)$. Therefore
$$
\int_{\Hilb^{d,\bullet}(X)} 1 \ de = \int_{\Hilb^{d,\bullet}(X)^{\CC^*}} 1 \ de.
$$
Recall that 
$$
\int_{\Hilb^{d,\bullet}(X)} 1 \ de = \sum_{n \in \ZZ} p^n \int_{\Hilb^{d,n}(X)} 1 \ de \in \ZZ(\!(p)\!).
$$
Using the map $\pi : X \rightarrow S$, a quasi-coherent sheaf on $X$ can be viewed as a quasi-coherent sheaf $\F$ on $S$ together with a morphism $\F \otimes K_{S}^{-1} \rightarrow \F$. A $\CC^*$-equivariant structure on $\F$ translates into a $\ZZ$-grading
$$
\pi_* \F = \bigoplus_{k \in \ZZ} \F_k,
$$
such that $\F \otimes K_{S}^{-1} \rightarrow \F$ is graded, i.e.
$$
\F_k \otimes K_{S}^{-1} \longrightarrow \F_{k-1}.
$$
Here $\F_k$ has weight $k$ and $K_{S}$ weight $-1$ under the $\CC^*$-action. The structure sheaf $\O_X$ corresponds to 
$$
\pi_* \O_X = \bigoplus_{k=0}^{\infty} K_{S}^{-k}.
$$
Therefore a $\CC^*$-equivariant morphism $\F \rightarrow \O_X$ corresponds to a graded sheaf $\F$ as above together with maps
\begin{displaymath}
\xymatrix
{
\cdots & \F_1 \ar[d] & \oplus & \F_0 \ar[d] & \oplus & \F_{-1} \ar[d] & \oplus & \F_{-2} \ar[d] & \cdots \\
\cdots &  0 & \oplus & \O_S & \oplus & K_{S}^{-1} & \oplus & K_{S}^{-2} & \cdots, 
}
\end{displaymath}
where 
\begin{displaymath}
\xymatrix
{
\F_k \otimes K_{S}^{-1} \ar[r] \ar[d] & \F_{k-1} \ar[d] \\
K_{S}^{k} \otimes K_{S}^{-1} \ar@{=}[r] & K_{S}^{k-1}
}
\end{displaymath}
commute for all $k\leq 0$ and the composition $\F_1 \otimes K_{S}^{-1} \rightarrow \F_0 \rightarrow \O_S$ is to zero. 

It is useful to re-define $\G_k := \F_{-k} \otimes K_{S}^{k}$. Then a $\CC^*$-equivariant morphism $\F \rightarrow \O_X$ is equivalent to the following data:
\begin{itemize}
\item quasi-coherent sheaves $\{\G_k\}_{k \in \ZZ}$ on $S$,
\item morphisms $\{\G_k \rightarrow \G_{k+1}\}_{k \in \ZZ}$,
\item morphisms $\G_k \rightarrow \O_S$ such that the following diagram commutes:
\end{itemize}
\begin{displaymath}
\xymatrix
{
\cdots & \G_{-1} \ar[d] \ar[r] & \G_0 \ar[r] \ar[d] & \G_{1} \ar[r] \ar[d] & \G_{2} \ar[r] \ar[d] & \cdots \\
\cdots & 0 \ar[r] & \O_S \ar@{=}[r]& \O_S \ar@{=}[r] & \O_S \ar@{=}[r] & \cdots 
}
\end{displaymath}
In the case of interest to us $\G \rightarrow \O_X$ is an ideal sheaf $I_Z \hookrightarrow \O_X$ cutting out $Z \subset X$. In the above language, this means $\G_k = 0$ for $k<0$, the morphisms $\G_k \rightarrow \O_S$ are injective (hence $\G_k = I_{Z_k \subset S}$ is an ideal sheaf cutting out $Z_k \subset S$), and the morphisms $\G_k \rightarrow \G_{k+1}$ are injective (hence $I_{Z_k \subset S} \subset I_{Z_{k+1} \subset S}$, i.e.~$Z_{k} \supset Z_{k+1}$). We conclude:
\begin{proposition} \label{nest}
A closed point $Z$ of $\Hilb^{d,\bullet}(X)^{\CC^*}$ corresponds to a finite nesting of closed subschemes of $S$
$$
Z_{0} \supset Z_{1} \supset \cdots \supset Z_{l},
$$
for some $l \geq 0$, such such that
$$
\sum_{k=0}^{l} [Z_k] = B + dF \in H_2(S).
$$
\end{proposition}
Let $\Hilb^{B+dF}(S)$ be the Hilbert scheme of effective divisors on $S$ with class $$B+dF \in H_2(S).$$ Let $g$ be the genus of $B$. In Lemma \ref{Hilbcvs} of the Appendix \ref{appHilb}, we prove that pull-back along $p$ and adding the section $B$ induces an isomorphism
$$
\Sym^d(B) \cong \Hilb^{B+dF}(S),
$$
for all $d > 2g-2$. This allows us to \emph{see} the curves on $S$. 

For any reduced curve $C \subset S$ defined by ideal sheaf $I_{C \subset S}$ and $d >0$, we denote by $dC$ the Gorenstein curve defined by the ideal sheaf $I_{C \subset S}^d$ (the $d$th power of $I_{C \subset S}$). We combine Lemma \ref{Hilbcvs} with a (family version of) Proposition \ref{nest} to prove the following:
\begin{proposition} \label{proprho}
Let $n \in \ZZ$ and $d > 2g-2$. Then there exists a morphism
$$
\rho_d : \Hilb^{d,n}(X)^{\CC^*} \longrightarrow \Sym^d(B). 
$$
At the level of closed points this morphism has the following description. Let $Z \in \Hilb^{d,n}(X)^{\CC^*}$ and let $Z_{\CM} \subset Z$ be the maximal Cohen-Macaulay subcurve of $Z$. Since $Z_{\CM}$ is $\CC^*$-fixed, its ideal sheaf decomposes
$$
I_{Z_{\CM}} = \bigoplus_{k=0}^{\infty} I_{Y_k \subset S} \otimes K_{S}^{-k},
$$
where 
$$
Y_0 = B \cup \lambda_{0}^{(1)} F_{x_1} \cup \cdots \cup \lambda_{0}^{(n)} F_{x_n}
$$
for some distinct closed points $x_i \in B$ and $\lambda_{0}^{(i)} > 0$ and
$$
Y_k = \lambda_{k}^{(1)} F_{x_1} \cup \cdots \cup \lambda_{k}^{(n)} F_{x_n}.
$$
for some $\lambda_{k-1}^{(i)} \geq \lambda_{k}^{(i)}$. Here $\lambda^{(i)} = (\lambda^{(i)}_{0} \geq \lambda^{(i)}_{1} \geq \cdots)$ define 2D partitions satisfying 
$$
\sum_{i=1}^{n} |\lambda^{(i)}| = d.
$$
The map $\rho_d$ sends $Z$ to 
$$
\sum_{i=1}^{n} |\lambda^{(i)}| x_i \in \Sym^d(B).
$$
\end{proposition}

\begin{remark}
The morphism of this proposition is perhaps somewhat surprising. Since we are on a 3-fold, the map which sends a closed subscheme of $Z \in \Hilb^{d,n}(X)$ to its underlying Cohen-Macaulay curve $Z_{\CM}$ is \emph{not} a morphism. Nevertheless, the map $\rho_d$ which records the location of the fibres in $Z_{\CM}$ and their multiplicities, is a morphism.
\end{remark}

\begin{proof}
The description of $\rho_d$ at the level of closed points is clear. We must construct $\rho_d$ as a morphism from Proposition \ref{nest} and Lemma \ref{Hilbcvs} of Appendix \ref{appHilb}.

Let $T$ be an arbitrary base scheme of finite type and let 
$$
\Y \subset X \times T
$$
be a $\CC^*$-fixed and $T$-flat closed subscheme. Assume for each $t \in T$ the fibre $\Y_t$ has class $B+dF \in H_2(S)$ and $\chi(\O_{\Y_t}) = n$. Since $\Y$ is $\CC^*$-fixed, Proposition \ref{nest} implies that its ideal sheaf decomposes\footnote{The arguments leading to Proposition \ref{nest} hold equally well for $T$-flat families over a base $T$.}
$$
I_{\Y} = \bigoplus_{k = 0}^{\infty} I_{\Y_k \subset S \times T} \otimes K_{S}^{-k},
$$
where $K_{S}$ is pulled-back along $S \times T \rightarrow S$ and 
$$
\Y_0 \supset \Y_1 \supset \cdots.
$$
Then each $\Y_k \subset S \times T$ is $T$-flat as well. The maximal CM subschemes $\Y_{k, \CM} \subset \Y_k \subset S \times T$ are $T$-flat as well \todo{Provide explanation why $\Y_{\CM,k}$ also $T$-flat?} and induces morphisms
\begin{align*}
T &\longrightarrow \Hilb^{B + d_0 F}(S), \\ 
T &\longrightarrow \Hilb^{d_k F}(S),  \textrm{ \ for \ } k>0
\end{align*}
where $\sum_k d_k = d$. Adding divisors gives a morphism $T \longrightarrow \Hilb^{B+dF}(S)$. By Lemma \ref{Hilbcvs}, we obtain a morphism $T \rightarrow \Sym^d(B)$. This morphism corresponds to a $T$-flat family for $\Sym^d(B)$. We have defined $\rho_d$ as a morphism. 
\end{proof}

%In the above proposition, each $Z_k \subset S$ contains a maximal Cohen-Macaulay (in fact, Gorenstein) subcurve $D_k$ such that $Z_k \setminus D_k$ is 0-dimensional. For $k=0$, $D_0$ is the scheme-theoretic union of the section $B$ and thickenings of certain distinct fibres $F_{x_1}$, $\ldots$, $F_{x_n}$. Denoting the orders of thickenings by $\lambda_{0}^{(1)}, \ldots, \lambda_{0}^{(n)} > 0$, we obtain\footnote{For any reduced curve $C$ on a surface $S$ with ideal sheaf $I_C \subset \O_S$ and $d>0$, we denote by $d C$ the scheme defined by the ideal sheaf $I_{C}^{d} \subset \O_S$.}
%$$
%D_0 = B \cup \lambda_{0}^{(1)} F_{x_1} \cup \cdots \cup \lambda_{0}^{(n)} F_{x_n}.
%$$
%This statement follows from Corollary \ref{cor: chow(beta) = sym (B)} of the appendix. Next, for all $i = 1, \ldots, n$ and $k \geq 1$, there are $\lambda_{k}^{(i)} \leq \lambda_{k-1}^{(i)}$ such that
%$$
%D_k = \lambda_{k}^{(1)} F_{x_1} \cup \cdots \cup \lambda_{k}^{(n)} F_{x_n}.
%$$
%We conclude:
%\begin{proposition} \label{ZCM}
%To each closed point $Z$ of $\Hilb^{d,\bullet}(X)^{\CC^*}$ correspond distinct closed points $x_1, \ldots, x_n \in B$ for some $n$ and (finite) 2D partitions $\lambda^{(1)}, \ldots, \lambda^{(n)}$ such that
%$$
%\sum_{i=1}^{n} |\lambda^{(i)}| = d.
%$$
%The maximal Cohen-Macaulay subcurve of $Z$ is given by the scheme-theoretic union of the zero section $B$ and the schemes with ideal sheaves
%$$
%\bigoplus_{k=0}^{\infty} \O_{S}(-\lambda_{k}^{(i)} F_{x_i}) \otimes K_{S}^{-k},
%$$
%for all $i = 1, \ldots, n$.
%\end{proposition}

%Note that in the notation of this proposition, the morphism $\rho_d$ in \eqref{rho} maps $Z$ to
%$$
%\sum_{i=1}^{n} |\lambda^{(i)}| x_i \in \Sym^d(B),
%$$
%where $|\lambda^{(i)}|$ denotes the size of the 2D partition $\lambda^{(i)}$.
%This leads to the following proposition:
%\begin{proposition} \label{C^*}
%For each $h>0$, there exists a stratification
%$$
%\Hilb^{h,\bullet}(X)^{\CC^*} = \coprod_{n=1}^{\infty} \coprod_{{\scriptsize{\begin{array}{c} \lambda^{(1)}, \ldots, \lambda^{(n)} \mathrm{s.t.} \\ \sum_{\alpha=1}^{n} |\lambda^{(\alpha)}| = h \end{array}}}} \Hilb^{h,\bullet}_{\lambda^{(1)}, \ldots, \lambda^{(n)}}(X)^{\CC^*},
%$$
%where $\Hilb^{h,\bullet}_{\lambda^{(1)}, \ldots, \lambda^{(n)}}(X)^{\CC^*}$ is the locally closed subset of subschemes $Z \subset X$ with maximal Cohen-Macaulay curve defined by the scheme-theoretic union of $B$ and schemes with ideal sheaves of the form \eqref{CMcurve} for some distinct fibre $F_{x_1}, \ldots, F_{x_n} \subset S$.
%\end{proposition}

%In this proposition, the number of points $n$ and the position of the points $x_1, \ldots, x_n \in B$ can still vary freely. We now want to refine the stratification by fixing the position of these points.


\section{Push-forward to the symmetric product} \label{sym}

In the previous section we constructed a morphism (Proposition \ref{proprho}) \todo{Add: this map and everything below needs $d> 2g-2$!!! Or maybe we can do better...}
\begin{equation} \label{rho}
\rho_{d} : \Hilb^{d,\bullet}(X)^{\CC^*} \longrightarrow \Sym^d(B).
\end{equation}
We obtain
$$
\int_{\Hilb^{d,\bullet}(X)^{\CC^*}} 1 \ de = \int_{\Sym^d(B)} \rho_{d*}(1) \ de,
$$
where $f_d := \rho_{d*}(1)$ is a constructible function on $\Sym^d(B)$. Its value at a closed point $\mathfrak{a} \in \Sym^d(B)$ is 
$$
f_d(\mathfrak{a}) = \int_{\rho_{d}^{-1}(\mathfrak{a})} 1 \ de.
$$
We are interested in the calculation of
$$
\widehat{\DT}(X) = \sum_{d \geq 0} \widehat{\DT}_d(X) q^d =\sum_{d \geq 0} q^d \int_{\Sym^d(B)} f_d \ de.
$$
It turns out that the constructible function $f_d : \Sym^d(B) \rightarrow \ZZ(\!(p)\!)$ has two multiplicative properties. The first one is described as follows. Denote by $B^{\sm} \subset B$ the open subset over which the fibres are smooth and by $B^{\sing}$ the $N$ points over which the fibres are singular. We can consider the restrictions of $f_d$ to $\Sym^d(B^{\sm}) \subset \Sym^d(B)$ and $\Sym^d(B^{\sing}) \subset \Sym^d(B)$. Denote by $M(p)$ the MacMahon function.
\begin{proposition} \label{mult1}
Let $d_1, d_2 \geq 0$ be such that $d_1+d_2 = d$. Then 
%there are constructible functions
%\begin{align*}
%g_{d_1} : \Sym^{d_1}(B^{\sm}) &\longrightarrow \ZZ(\!(p)\!) \\
%h_{d_2} : \Sym^{d_2}(B^{\sing}) &\longrightarrow \ZZ(\!(p)\!),
%\end{align*}
%such that for any $\sum_i a_i x_i \in \Sym^{d_1}(B^{\sm})$ and $\sum_j b_j y_j \in \Sym^{d_2}(B^{\sing})$ \todo{Forgot: points which can are off the curve.}
$$ 
f_d(\mathfrak{a} + \mathfrak{b}) =\frac{(p^{\frac{1}{2}} - p^{-\frac{1}{2}})^{e(B)}}{M(p)^{e(X)}} \cdot f_{d_1}(\mathfrak{a}) \cdot f_{d_2}(\mathfrak{b}), 
$$
for any $\mathfrak{a} \in \Sym^{d_1}(B^{\sm})$ and $\mathfrak{b} \in \Sym^{d_2}(B^{\sing})$. 
\end{proposition}
We prove this proposition in Section \ref{chargh}. The following product formula is an immediate consequence of this result
\begin{align}
\begin{split} \label{firstprod}
&\sum_{d \geq 0} q^d \int_{\Sym^d(B)} f_d \ de = \\
&\frac{(p^{\frac{1}{2}} - p^{-\frac{1}{2}})^{e(B)}}{M(p)^{e(X)}}  \Bigg( \sum_{d \geq 0} q^d \int_{\Sym^d(B^{\sm})} f_d \ de \Bigg) \cdot \Bigg( \sum_{d \geq 0} q^d \int_{\Sym^d(B^{\sing})} f_d \ de \Bigg). 
\end{split}
\end{align}
The restricted constructible functions $f_d  : \Sym^d(B^{\sm}) \rightarrow \ZZ(\!(p)\!)$ and $f_d  : \Sym^d(B^{\sing}) \rightarrow \ZZ(\!(p)\!)$ satisfy further multiplicative properties:
\begin{proposition} \label{mult2}
There exist functions $g : \ZZ_{\geq 0} \rightarrow \ZZ(\!(p)\!)$ and $h : \ZZ_{\geq 0} \rightarrow \ZZ(\!(p)\!)$ taking values in formal Laurent series $\ZZ(\!(p)\!)$, such that $g(0)=1$, $h(0)=1$, and
\begin{align*}
f_{d}(\mathfrak{a}) &= \frac{M(p)^{e(X)}}{(p^{\frac{1}{2}} - p^{-\frac{1}{2}})^{e(B)}} \cdot \prod_{i=1}^m g(a_i), \\
f_{d}(\mathfrak{b}) &= \frac{M(p)^{e(X)}}{(p^{\frac{1}{2}} - p^{-\frac{1}{2}})^{e(B)}} \cdot \prod_{j=1}^n h(b_j), 
\end{align*}
for all $d \geq 0$, $\mathfrak{a} = \sum_{i=1}^m a_i x_i \in \Sym^{d}(B^{\sm})$, and $\mathfrak{b} = \sum_{j=1}^n b_j y_j \in \Sym^{d}(B^{\sing})$, where $x_i \in B^{\sm}$ and $y_j \in B^{\sing}$ are collections of distinct closed points.
\end{proposition}
We prove this proposition in Section \ref{chargh}. Together with Lemma \ref{lem: formula for euler char of sym products} from the appendix, Proposition \ref{mult2} and equation \eqref{firstprod} imply
\begin{equation} \label{initialprod}
\sum_{d \geq 0} q^d \int_{\Sym^d(B)} f_d \ de = \frac{M(p)^{N}}{(p^{\frac{1}{2}} - p^{-\frac{1}{2}})^{e(B)}} \cdot \Bigg( \sum_{a=0}^{\infty} g(a) q^a \Bigg)^{e(B) - N} \cdot \Bigg( \sum_{b=0}^{\infty} h(b) q^b \Bigg)^N.
\end{equation}
Our goal is to prove Propositions \ref{mult1}, \ref{mult2}, and find formulae for $g(a)$, $h(b)$. This requires a better understanding of the strata
$$
\rho_{d}^{-1} (\mathfrak{a} + \mathfrak{b}) \subset \Hilb^{d, \bullet}(X)^{\CC^*},
$$
for all $\mathfrak{a} \in \Sym^{d_1}(B^{\sm})$ and $\mathfrak{b} \in \Sym^{d_2}(B^{\sing})$ with $d_1+d_2=d$. Now suppose 
\begin{align*}
\mathfrak{a} &= \sum_{i=1}^{m} a_i x_i \in \Sym^{d_1}(B^{\sm}), \\
\mathfrak{b} &= \sum_{j=1}^{n} b_j y_j \in \Sym^{d_2}(B^{\sing}),
\end{align*}
where  $x_i \in B^{\sm}$ and $y_j \in B^{\sing}$ are collections of distinct closed points and $d_1+d_2=d$. Proposition \ref{proprho} gives a decomposition of $\rho_{d}^{-1} ( \mathfrak{a} + \mathfrak{b} )$ into components\footnote{We use the term component somewhat loose: it means a subset which is both open and closed. We do not care whether it is connected.}
\begin{equation} \label{comps}
\bigsqcup_{{\scriptsize{\begin{array}{c} \lambda^{(1)} \vdash a_1 \\ \cdots \\ \lambda^{(m)} \vdash a_m \end{array}}}} \bigsqcup_{\scriptsize{\begin{array}{c} \mu^{(1)} \vdash b_1 \\ \cdots \\ \mu^{(m)} \vdash b_m \end{array}}} \Sigma(x_1, \ldots, x_m, y_1, \ldots, y_n, \lambda^{(1)}, \ldots, \lambda^{(m)}, \mu^{(1)}, \ldots, \mu^{(n)}).
\end{equation}
We abbreviate these components by $\Sigma(\boldsymbol{x};\boldsymbol{y};\boldsymbol{\lambda};\boldsymbol{\mu})$. We see that $\Sigma(\boldsymbol{x};\boldsymbol{y};\boldsymbol{\lambda};\boldsymbol{\mu})$ can be characterized as the stratum of points $Z \in \Hilb^{d,\bullet}(X)^{\CC^*}$, for which the maximal Cohen-Macaulay subcurve $Z_{\CM} \subset Z$ is determined by the data $\boldsymbol{x}, \boldsymbol{y}, \boldsymbol{\lambda}, \boldsymbol{\mu}$ as in Proposition \ref{proprho}. Note that these strata have a natural scheme structure: the fibres of $\rho_d$ are closed subschemes of $\Hilb^{d,\bullet}(X)^{\CC^*}$ and these strata are components of them. We are interested in the Euler characteristics of these strata. In the next section, we will see the Euler characteristic of $\Sigma(\boldsymbol{x};\boldsymbol{y};\boldsymbol{\lambda};\boldsymbol{\mu})$ does \emph{not} depend on the exact location of the points $x_i \in B^{\sm}$ and $y_j \in B^{\sing}$, but only on their number $m$ and $n$ and the partitions $\lambda^{(i)}$ and $\mu^{(j)}$.


\section{Restriction to formal neighbourhoods} \label{formal}

In the previous two sections we reduced our consideration to the strata $\Sigma(\boldsymbol{x};\boldsymbol{y};\boldsymbol{\lambda};\boldsymbol{\mu})$ of $Z \in \Hilb^{d,\bullet}(X)^{\CC^*}$ for which the maximal Cohen-Macaulay subcurve $Z_{\CM} \subset Z$ is determined by the data $\boldsymbol{x}, \boldsymbol{y}, \boldsymbol{\lambda}, \boldsymbol{\mu}$. In this section we further break down this stratum by cutting it up in pieces covered by formal neighbourhoods. For notational simplicity, we first consider the case where the base point is 
$$
a x + b y \in \Sym^d(B),
$$
with $x \in B^{\sm}$, $y \in B^{\sing}$, and $d=a+b$. We show how to compute $e(\Sigma(x,y,\lambda,\mu))$. Once this case is established, it is not hard to generalize to arbitrary $e(\Sigma(\boldsymbol{x};\boldsymbol{y};\boldsymbol{\lambda};\boldsymbol{\mu}))$. This leads to a proof of Propositions \ref{mult1}, \ref{mult2}, and a geometric characterization of the functions $g(a)$, $h(b)$ of Section \ref{sym}.


\subsection{Fpqc cover}

The idea is to use an appropriate cover of $X$ and calculate on pieces of the cover. We first give a complex analytic definition of the cover to aid the intuition and then give the actual ``algebro-geometric cover'': 
\begin{enumerate}
\item The reduced support $B \cup F_x \cup F_y$ has three singular points\footnote{Recall that $x,y \in B$ in the base can be viewed as points on $S$ and $X$ via the sections $B \subset S \subset X$.}: $x,y \in B$ and $z \in F^{\sing}_{y}$. We take small open balls around these points.
\item Consider the punctured curve $B^\circ := B \setminus \{x,y\}$ and let $X^\circ := X \setminus (F_x  \cup F_y)$. We take a tubular neighbourhood of $B^\circ \subset X^\circ$.
\item Consider the punctured curve $F_{x}^{\circ} := F_x \setminus \{x\}$ and let $X^\circ := X \setminus B$. We take a tubular neighbourhood of $F_{x}^\circ \subset X^\circ$.
\item Consider the punctured curve $F_{y}^{\circ} := F_y \setminus \{y,z\}$ and let $X^\circ := X \setminus (B \cup \{z\})$. We take a tubular neighbourhood of $F_{y}^{\circ} \subset X^\circ$.
\item Finally, we take $W = X \setminus (B \cup F_x \cup F_y)$. 
\end{enumerate}

In order to work in algebraic geometry, in (1) we take the formal neighbourhood $\widehat{X}_x$ of $\{x\}$ in $X$. Denote the local ring at $x$ by $(R,\mathfrak{m})$. Then by $\widehat{X}_x$ we mean the (non-noetherian) scheme
$$
\Spec \varprojlim R / \mathfrak{m}^n 
$$
and \emph{not} the formal scheme $$\mathrm{Spf} \varprojlim R / \mathfrak{m}^n.$$ Similarly in (2) and (3), let $\widehat{X}_y$ be the formal neighbourhood of $\{y\}$ in $X$ and $\widehat{X}_z$ the formal neighbourhood of $\{z\}$ in $X$. Note that 
$$
\widehat{X}_x \cong \widehat{X}_y \cong \widehat{X}_z \cong\Spec \CC[\![x_1,x_2,x_3]\!].
$$
Even though $\widehat{X}_x$ is non-noetherian, the morphism $\widehat{X}_x \rightarrow X$ has a good property: it is fpqc so can be used as part of a cover \cite[Sect.~2.3.2]{Vis}. Flatness of this map follows from the fact that formal completion is an exact operation \cite[Prop.~10.12, 10.13]{AM}. 

In (2) we consider $B^\circ := B \setminus \{x,y\}$, $X^\circ := X \setminus (F_x  \cup F_y)$ and let $\widehat{X}^{\circ}_{B^\circ}$ be the formal neighbourhood of $F_{x}^{\circ}$ in $X^\circ$. For (3) and (4) the formal neighbourhoods $\widehat{X}^{\circ}_{F_{x}^{\circ}}$ and $\widehat{X}^{\circ}_{F_{y}^{\circ}}$ are defined analogously. Note that the definition of $X^\circ$ in (2)--(4) varies slightly. Finally, in (5) we take $W = X \setminus (B \cup F_x \cup F_y)$. Then
$$
\mathfrak{U} = \{\widehat{X}_x \rightarrow X, \widehat{X}_y \rightarrow X, \widehat{X}_z \rightarrow X, \widehat{X}^{\circ}_{B^\circ} \rightarrow X, \widehat{X}^{\circ}_{F_{x}^{\circ}} \rightarrow X, \widehat{X}^{\circ}_{F_{y}^{\circ}} \rightarrow X, W \subset X\}
$$
forms an fpqc cover of $X$. Consequently the data of a quasi-coherent sheaf on $X$ is equivalent to the data of quasi-coherent sheaves on each of the opens of $\mathfrak{U}$ and gluing isomorphisms between the restrictions on the overlaps. Technically: quasi-coherent sheaves on $X$ form a stack with respect to the fpqc topology \cite[Thm.~4.23]{Vis}.


\subsection{Local moduli spaces} \label{localmod}

We now introduce moduli spaces of closed subschemes of dimension $\leq 1$ on each of the pieces of the cover $\mathfrak{U}$. Assume the coordinates on $$\widehat{X}_x \cong \Spec \CC[\![ x_1,x_2,x_3]\!]$$ are chosen such that $x_2=x_3=0$ corresponds to the intersection $\widehat{X}_x \times_X B$ and $x_1=x_3=0$ corresponds to $\widehat{X}_x \times_X F_x$. Define
\begin{align*}
&\Hilb^{(1,d),n}(\widehat{X}_x) := \\
&\big\{ I_Z \subset \O_{\widehat{X}_x} \ : \ [Z] = [\widehat{X}_x \times_X B] + d [\widehat{X}_x \times_X F_x] \ {\rm{and}} \ h^0(I_{Z_{\CM}} / I_Z) = n \big\}.
\end{align*}
Here the equation
$$
[Z] = [\widehat{X}_x \times_X B] + d [\widehat{X}_x \times_X F_x]
$$
means $Z$ is supported along $$(\widehat{X}_x \times_X B) \cup (\widehat{X}_x \times_X F_x)$$ with multiplicity 1 along $(\widehat{X}_x \times_X B)$ and multiplicity $d$ along $(\widehat{X}_x \times_X F_x)$. Furthermore, $Z_{\CM}$ denotes the maximal Cohen-Macaulay subcurve of $Z$. The ideal sheaves fit into a short exact sequence
$$
0 \longrightarrow I_{Z} \longrightarrow I_{Z_{\CM}} \longrightarrow Q \longrightarrow 0, 
$$
where $Q$ is a 0-dimensional. The Hilbert scheme $\Hilb^{(1,d),n}(\widehat{X}_y)$ is defined likewise replacing the point $x$ by $y$. For $\widehat{X}_z$, we define
$$
\Hilb^{d,n}(\widehat{X}_z) := \big\{ I_Z \subset \O_{\widehat{X}_z} \ : \ [Z] = d [\widehat{X}_z \times_X F_y] \ {\rm{and}} \ h^0(I_{Z_{\CM}} / I_Z) = n \big\}.
$$
Each of $\widehat{X}_x, \widehat{X}_y, \widehat{X}_z$ has an action of $\CC^*$ compatible with the fibre scaling on $X$. This action lifts to the moduli space. Moreover, since each of these formal neighbourhoods is isomorphic to $\Spec \CC[\![x_1,x_2,x_3]\!]$, the bigger torus $\CC^{*3}$ acts on it and this action lifts to the moduli space. The existence of these ``extra actions'' will be used in Section \ref{vertex}.

Next consider $\widehat{X}^{\circ}_{B^\circ}$, the formal neighbourhood of the punctured zero section $B^\circ \subset X^\circ$. Define
$$
\Hilb^{1,n}(\widehat{X}^{\circ}_{B^\circ}) := \big\{ I_Z \subset \O_{\widehat{X}^{\circ}_{B^\circ}} \ : \ [Z] = [\widehat{X}^{\circ}_{B^\circ} \times_X B] \ {\rm{and}} \ h^0(I_{Z_{\CM}} / I_Z) = n \big\}.
$$
For $\widehat{X}^{\circ}_{F_{x}^{\circ}}$, $\widehat{X}^{\circ}_{F_{y}^{\circ}}$ we define
\begin{align*}
\Hilb^{d,n}(\widehat{X}^{\circ}_{F_{x}^{\circ}}) &:= \big\{ I_Z \subset \O_{\widehat{X}^{\circ}_{F_{x}^{\circ}}} \ : \ [Z] = d [\widehat{X}^{\circ}_{F_{x}^{\circ}} \times_X B] \ {\rm{and}} \ h^0(I_{Z_{\CM}} / I_Z) = n \big\}, \\
\Hilb^{d,n}(\widehat{X}^{\circ}_{F_{y}^{\circ}}) &:= \big\{ I_Z \subset \O_{\widehat{X}^{\circ}_{F_{y}^{\circ}}} \ : \ [Z] = d [\widehat{X}^{\circ}_{F_{y}^{\circ}} \times_X B] \ {\rm{and}} \ h^0(I_{Z_{\CM}} / I_Z) = n \big\}.
\end{align*}
Finally, for $W$ we define
$$
\Hilb^{0,n}(W) := \big\{ I_Z \subset \O_{W} \ : \ \dim(Z) = 0 \ {\rm{and}} \ h^0(\O_Z) = n \big\}.
$$
On $\widehat{X}^{\circ}_{B^\circ}$, $\widehat{X}^{\circ}_{F_{x}^{\circ}}$, $\widehat{X}^{\circ}_{F_{y}^{\circ}}$, and $W$ we have an action of $\CC^*$ compatible with the fibre scaling on $X$. These actions lift to the moduli space. However, \emph{unlike} for $\widehat{X}_x$, $\widehat{X}_y$, $\widehat{X}_z$, no additional tori act.

As before, we use the notation $\Hilb^{(1,d),\bullet}(\widehat{X}_x)$ for the union of all $\Hilb^{(1,d),n}(\widehat{X}_x)$ (and similarly for all other moduli spaces of this section). Like in Section \ref{fixedlocus}, the components of the $\CC^*$-fixed locus of $\Hilb^{(1,d),\bullet}(\widehat{X}_x)$ are indexed by 2D partitions 
$$
\Hilb^{(1,d),\bullet}(\widehat{X}_x)^{\CC^*} = \bigsqcup_{\lambda \vdash d} \Hilb^{(1,d),\bullet}(\widehat{X}_x)_{\lambda}^{\CC^*}.
$$

\begin{proposition} \label{bij}
Consider the stratum $\Sigma(x,y,a,b)$, where $a = |\lambda|$ and $b = |\mu|$. Restriction from $X$ to the elements of the cover $\mathfrak{U}$ induces a morphism
\begin{align}
\begin{split} \label{restr}
\Sigma(x,y,\lambda,\mu) \longrightarrow &\Hilb^{(1,a),\bullet}(\widehat{X}_x)_{\lambda}^{\CC^*} \times \Hilb^{(1,b),\bullet}(\widehat{X}_y)_{\mu}^{\CC^*} \times \Hilb^{b,\bullet}(\widehat{X}_z)_{\mu}^{\CC^*} \times \\
&\Hilb^{1,\bullet}(\widehat{X}^{\circ}_{B^\circ})^{\CC^*} \times \Hilb^{a,\bullet}(\widehat{X}^{\circ}_{F_{x}^{\circ}})_{\lambda}^{\CC^*} \times \Hilb^{b,\bullet}(\widehat{X}^{\circ}_{F_{y}^{\circ}})_{\mu}^{\CC^*} \times \\
&\Hilb^{0,\bullet}(W)^{\CC^*},
\end{split}
\end{align}
which is a bijection on closed points.
\end{proposition}
\begin{proof}
Since pull-back works in families, restriction defines a morphism between LHS and RHS. For the rest of the proof, we work on closed points only.

Since $\mathfrak{U}$ is an fpqc cover, fpqc descent implies that any ideal sheaf $I_Z \subset \O_X$ is entirely determined by its restriction along the morphisms of the elements of $\mathfrak{U}$. This proves injectivity.

Conversely, given local ideal sheaves in the image of \eqref{restr}, their restrictions to overlaps only depend on the underlying Cohen-Macaulay curve (and not on the embedded points). Since we chose the strata such that the underlying Cohen-Macaulay curve automatically glues, there are no further gluing conditions and fpqc descent implies surjectivity.
\end{proof}
   
\begin{remark}
Note that the argument of Proposition \ref{bij} is purely set-theoretic in nature. We do \emph{not} claim \eqref{restr} is an isomorphism of schemes. 
\end{remark}

\begin{remark}
It is important to relate holomorphic Euler characteristic of domain and target in \eqref{restr}. For any subscheme $Z$ in the domain $\Sigma(x,y,\lambda,\mu)$, denote the corresponding maximal Cohen-Macaulay curve of its elements by $Z_{\CM}$ (Proposition \ref{proprho}). Then
$$
\chi(\O_Z) = \chi(\O_{Z_{\CM}}) + \chi(I_{Z_{\CM}} / I_{Z}).
$$ 
Recall that $Z_{\CM}$ is entirely determined by the data $x,y, \lambda, \mu$, where $\lambda = (\lambda_0 \geq \lambda_1 \geq \cdots)$ and $\mu = (\mu_0 \geq \mu_1 \geq \cdots)$ are 2D partitions (equation \eqref{comps}). An easy calculation shows 
$$
\chi(\O_{Z_{\CM}}) = \chi(\O_B) - \lambda_0 - \mu_0.
$$
We conclude
\begin{equation} \label{relchi}
\chi(\O_Z) = \frac{e(B)}{2} - \lambda_0 - \mu_0 + \chi(I_{Z_{\CM}} / I_{Z}).
\end{equation}
\end{remark}

Proposition \ref{bij} allows us to calculate 
$$
f_d(ax+by) = e(\rho_{d}^{-1}(ax+by)) = \sum_{\lambda \vdash a} \sum_{\mu \vdash b} e(\Sigma(x,y,a,b)).
$$
By Proposition \ref{bij} and \eqref{relchi} this equals
\begin{align}
\begin{split} \label{fdintermediate}
f_d(ax+by) = \ &p^{\frac{e(B)}{2}} e(\Hilb^{1,\bullet}(\widehat{X}^{\circ}_{B^{\circ}})^{\CC^*}) e(\Hilb^{0,\bullet}(W)^{\CC^*}) \times \\
&\sum_{\lambda \vdash a} \sum_{\mu \vdash b} p^{- \lambda_0 - \mu_0 } e(\Hilb^{(1,a),\bullet}(\widehat{X}_{x})_{\lambda}^{\CC^*}) e(\Hilb^{(1,b),\bullet}(\widehat{X}_{y})_{\mu}^{\CC^*}) \times \\
&e(\Hilb^{b,\bullet}(\widehat{X}_{z})_{\mu}^{\CC^*}) e(\Hilb^{a,\bullet}(\widehat{X}^{\circ}_{F_{x}^{\circ}})_{\lambda}^{\CC^*}) e(\Hilb^{b,\bullet}(\widehat{X}^{\circ}_{F_{y}^{\circ}})_{\mu}^{\CC^*}).
\end{split}
\end{align}

Before we proceed, we calculate $e(\Hilb^{0,\bullet}(W)^{\CC^*})$ and $e(\Hilb^{1,\bullet}(\widehat{X}^{\circ}_{B^{\circ}})^{\CC^*})$. The first follows from a formula of J.~Cheah \cite{Che} 
\begin{equation} \label{Cheah}
e(\Hilb^{0,\bullet}(W)^{\CC^*}) = M(p)^{e(W)}.
\end{equation}
For the second we use the following proposition:
\begin{proposition} \label{section}
Let $x_1, \ldots, x_n \in B$ be any number of closed points. Define 
\begin{align*}
B^\circ &:= B \setminus \{x_1, \ldots, x_n\}, \\
X^\circ &:= X \setminus \bigsqcup_{i=1}^{n} F_{x_i}.
\end{align*} 
Let $\widehat{X}^{\circ}_{B^{\circ}}$ be the formal neighbourhood of $B^\circ$ in $X^\circ$. Define $\Hilb^{1,n}(\widehat{X}^{\circ}_{B^{\circ}})$ as the Hilbert scheme of subschemes $Z \subset \widehat{X}^{\circ}_{B^{\circ}}$, such that $Z_{\CM} = B^\circ$ and $\chi(I_{Z_{\CM}} / I_Z) = n$.
%, where $Z_{CM}$ denotes the maximal Cohen-Macaulay subscheme contained in $Z$ (Proposition \ref{ZCM}). 
Then
$$
e(\Hilb^{1,\bullet}(\widehat{X}^{\circ}_{B^{\circ}})) = \Bigg( \frac{M(p)}{(1-p)} \Bigg)^{e(B^\circ)}.
$$
\end{proposition}
\begin{proof}
Let $x \in B^\circ$ and let $\widehat{X}_{x} \cong \Spec \CC [\![x_1,x_2,x_3]\!]$ be the formal neighbourhood of $x$ in $X^\circ$. Denote by $$\Hilb^{1,n}(\widehat{X}^{\circ}_{x})$$ the Hilbert scheme of subschemes $Z \subset \widehat{X}_{x}$, such that $Z_{\CM} = \{x_2=x_3=0\}$ and $\chi(I_{Z_{\CM}} / I_Z) = n$. 

We have projections
$$
X^\circ \longrightarrow S^\circ \longrightarrow B^\circ.
$$
Similar to Proposition \ref{bij}, these map induces a morphism
$$
\Hilb^{1,n}(\widehat{X}^{\circ}_{B^{\circ}}) \longrightarrow \Sym^n(B^\circ).
$$
The fibre over a point $\mathfrak{a} = \sum_i a_i x_i$ equals
$$
\prod_{i} \Hilb^{1,a_i}(\widehat{X}_{x_i}).
$$
This follows by using an appropriate fpqc cover of $B^\circ$ similar to Proposition \ref{bij}. Therefore, Lemma \ref{lem: formula for euler char of sym products} of the appendix implies
$$
e(\Hilb^{1,\bullet}(\widehat{X}^{\circ}_{B^{\circ}})) = \Bigg( \sum_{a=0}^{\infty} e(\Hilb^{1,a}(\widehat{X}_{x})) p^a \Bigg)^{e(B^\circ)}.
$$
The formal neighbourhood $\widehat{X}_{x}$ has an action of $\CC^{*3}$ and this action lifts to $\Hilb^{1,a}(\widehat{X}_{x})$. The fixed locus consists of a finite number of points counted by the topological vertex\footnote{Discussed in general in Section \ref{vertex}.}
$$
\sfV_{\square,\varnothing,\varnothing}(p) = \frac{M(p)}{1-p}.
$$
The proof follows.
\end{proof}

Using \eqref{Cheah} and Proposition \ref{section}, equation \eqref{fdintermediate} becomes
\begin{align*}
f_d(ax+by) = \ &\frac{M(p)^{e(X)}}{(p^{\frac{1}{2}}-p^{-\frac{1}{2}})^{e(B)}} \times \\
&(1-p) \sum_{\lambda \vdash a} p^{-\lambda_0} e(\Hilb^{(1,a),\bullet}(\widehat{X}_x)_{\lambda}^{\CC^*}) e(\Hilb^{a,\bullet}(\widehat{X}^{\circ}_{F_{x}^{\circ}})_{\lambda}^{\CC^*}) \times \\
&\frac{1-p}{M(p)} \sum_{\mu \vdash b} p^{-\mu_0} e(\Hilb^{(1,b),\bullet}(\widehat{X}_y)_{\mu}^{\CC^*}) e(\Hilb^{b,\bullet}(\widehat{X}_z)_{\mu}^{\CC^*}) e(\Hilb^{b,\bullet}(\widehat{X}^{\circ}_{F_{y}^{\circ}})_{\mu}^{\CC^*}).
\end{align*}

   
\subsection{Geometric characterization of $g(a)$ and $h(b)$} \label{chargh}

The arguments of the preceding two sections are straightforwardly modified to any stratum $\Sigma(\boldsymbol{x};\boldsymbol{y};\boldsymbol{\lambda};\boldsymbol{\mu})$. Fix a smooth fibre $F_x$ and a singular fibre $F_y$. Denote the singular point of $F_y$ by $z$. Let $\widehat{X}_x$, $\widehat{X}_z$ be the formal neighbourhoods of $x$, $z$ in $X$. Define $\Hilb^{(1,a),\bullet}(\widehat{X}_x)$, $\Hilb^{b,\bullet}(\widehat{X}_z)$ as in Section \ref{localmod}. As in Section \ref{localmod}, we also consider the ``tubular'' formal neighbourhoods $\widehat{X}^{\circ}_{F_{x}^{\circ}}$, $\widehat{X}^{\circ}_{F_{y}^{\circ}}$ and corresponding Hilbert schemes $\Hilb^{a,\bullet}(\widehat{X}^{\circ}_{F_{x}^{\circ}})$, $\Hilb^{b,\bullet}(\widehat{X}^{\circ}_{F_{y}^{\circ}})$. The arguments of this section yield:
\begin{proposition} \label{geomgh}
For any $a,b>0$ define
\begin{align*}
g(a) &:= (1-p) \sum_{\lambda \vdash a} p^{-\lambda_0} e(\Hilb^{(1,a),\bullet}(\widehat{X}_x)_{\lambda}^{\CC^*}) e(\Hilb^{a,\bullet}(\widehat{X}^{\circ}_{F_{x}^{\circ}})_{\lambda}^{\CC^*}), \\
h(b) &:= \frac{1-p}{M(p)} \sum_{\mu \vdash b} p^{-\mu_0} e(\Hilb^{(1,b),\bullet}(\widehat{X}_y)_{\mu}^{\CC^*}) e(\Hilb^{b,\bullet}(\widehat{X}_z)_{\mu}^{\CC^*}) e(\Hilb^{b,\bullet}(\widehat{X}^{\circ}_{F_{y}^{\circ}})_{\mu}^{\CC^*}),
\end{align*}
and let $g(0) := 1$, $h(0) :=1$. Then
\begin{align*}
f_{d}(\mathfrak{a} + \mathfrak{b}) &= \frac{M(p)^{e(X)}}{(p^{\frac{1}{2}}-p^{-\frac{1}{2}})^{e(B)}} \cdot \prod_{i} g(a_i) \cdot \prod_{j} h(b_j), 
\end{align*}
for any $\mathfrak{a} = \sum_i a_i x_i \in \Sym^{d}(B^{\sm})$ and $\mathfrak{b} = \sum_j b_j y_j \in \Sym^{d}(B^{\sing})$, where $x_i \in B^{\sm}$ and $y_j \in B^{\sing}$ are collections of distinct closed points.
\end{proposition}
   
We immediately deduce:  
\begin{corollary} 
Propositions \ref{mult1} and \ref{mult2} are true for $g(a)$ and $h(b)$ defined in Proposition \ref{geomgh}.
\end{corollary}   
   
   
\section{Reduction to the topological vertex}  \label{vertex} 

In this section, we prove the main result of this paper: expressions for $\widehat{\DT}(X)$ and $\widehat{\DT}^{\conn}(X)$ in terms of the topological vertex $\sfV_{\lambda,\mu,\nu}(p)$, $e(B)$, and $N$ (the number of nodal fibres). This is Theorem \ref{main} below. The theorem follows by expressing $g(a)$ and $h(b)$ of Proposition \ref{geomgh} in terms of the topological vertex. 


\subsection{Point contributions}   

We denote by 
$$
\sfV_{\lambda,\mu,\nu}(p) = \sum_{\pi} p^{|\pi|}, 
$$
the topological vertex of DT theory\footnote{In general this depends on an equivariant measure $\mathsf{w}(\pi)$ depending on equivariant parameters $s_1, s_2, s_3$. If $s_1+s_2+s_3=0$, then $\mathsf{w}(\pi) = \pm 1$. Since we are working with an Euler characteristic version of DT invariants, we take all signs to be $+1$.} \cite{MNOP1, MNOP2}. Here the sum is over all 3D partitions $\pi$ with outgoing legs $\lambda, \mu, \nu$ and $|\pi|$ denotes renormalized volume. For a 2D partition $\lambda = (\lambda_0 \geq \lambda_1 \geq \cdots)$, we write $\lambda'$ for the corresponding transposed partition and 
\begin{align*}
|\lambda| &:= \sum_{k=0}^{\infty} \lambda_k, \\
|\!|\lambda|\!| &:= \sum_{k=0}^{\infty} \lambda_{k}^{2}.
\end{align*}

\begin{proposition} \label{vertex1}
Let $F_x$ be a smooth fibre and $F_y$ a singular fibre with singularity $z$. Then for any $\lambda \vdash a$, $\mu \vdash b$
\begin{align*}
p^{-\lambda_0} e(\Hilb^{(1,a),\bullet}(\widehat{X}_x)_{\lambda}^{\CC^*}) &= \sfV_{\lambda,\square,\varnothing}(p), \\
p^{-\mu_0} e(\Hilb^{(1,b),\bullet}(\widehat{X}_y)_{\mu}^{\CC^*}) &= \sfV_{\mu,\square,\varnothing}(p), \\
p^{-|\!|\mu|\!|} e(\Hilb^{b,\bullet}(\widehat{X}_z)_{\mu}^{\CC^*}) &= \sfV_{\mu,\mu',\varnothing}(p).
\end{align*}
\end{proposition}
\begin{proof}
Recall that $$\widehat{X}_x \cong \widehat{X}_y \cong \widehat{X}_z \cong \Spec \CC[\![x_1,x_2,x_3]\!].$$ Therefore, $\CC^{*3}$ acts on each of these schemes and their moduli spaces 
$$
\Hilb^{(1,a),\bullet}(\widehat{X}_x)_{\lambda}^{\CC^*}, \ \Hilb^{(1,b),\bullet}(\widehat{X}_y)_{\mu}^{\CC^*}, \ \Hilb^{b,\bullet}(\widehat{X}_z)_{\mu}^{\CC^*}.
$$
The coordinates can be chosen such that the action of the last factor of $\CC^{*3}$ corresponds to $x_3 \mapsto t_3 x_3$. This component acts trivially since we are already on the $\CC^*$-fixed locus. The $\CC^{*3}$-fixed locus consists of isolated reduced points corresponding to monomial ideals with asymptotics $(\lambda,\varnothing,\varnothing)$, $(\mu,\varnothing,\varnothing)$, $(\mu,\mu',\varnothing)$ respectively\footnote{The transpose in $\mu'$ occurs, because we follow the convention of \cite{ORV}.}. These monomial ideals are exactly what the topological vertex counts. 

Finally, note that the generating functions $e(\Hilb^{(1,a),\bullet}(\widehat{X}_x)_{\lambda}^{\CC^*})$, $e(\Hilb^{(1,b),\bullet}(\widehat{X}_y)_{\mu}^{\CC^*})$, $e(\Hilb^{b,\bullet}(\widehat{X}_z)_{\mu}^{\CC^*})$ all start with $1$. On the other hand, from the definition
\begin{align*}
\sfV_{\lambda,\square,\varnothing}(p) &= p^{-\lambda_0} + \cdots, \\
\sfV_{\mu,\square,\varnothing}(p) &= p^{-\mu_0} + \cdots, \\
\sfV_{\mu,\mu',\varnothing}(p) &= p^{-\sum_{k=0}^{\infty} \mu_{k}^{2}} + \cdots,
\end{align*}
where $\cdots$ stands for higher order terms in $p$. The proposition follows.
\end{proof}


\subsection{Fibre contribution}

Let $F_x$ be a smooth fibre and $F_y$ a singular fibre. Recall the formal neighbourhoods $\widehat{X}^{\circ}_{F_{x}^{\circ}}$,  $\widehat{X}^{\circ}_{F_{y}^{\circ}}$ of Section \ref{formal}.
\begin{proposition} \label{vertex2}
For any $\lambda \vdash a$ and $\mu \vdash b$, we have
\begin{align*}
e(\Hilb^{a,\bullet}(\widehat{X}^{\circ}_{F_{x}^{\circ}})_{\lambda}^{\CC^*}) &= \frac{1}{\sfV_{\lambda,\varnothing,\varnothing}(p)}, \\
e(\Hilb^{b,\bullet}(\widehat{X}^{\circ}_{F_{y}^{\circ}})_{\mu}^{\CC^*}) &= \frac{1}{\sfV_{\mu,\varnothing,\varnothing}(p)}.
\end{align*}
\end{proposition}
\begin{proof}
We start with the first equation. Let $F := F_x \subset S$ be a smooth fibre. Consider the auxiliary surface 
$$
\tilde{S} = B \times F
$$ 
and let $Y = \mathrm{Tot}(K_{\tilde{S}})$. 
%let $e \in F$ be any closed point and consider the embeddings 
%\begin{align*}
%B &\hookrightarrow \tilde{S}, \ b \mapsto (b,e) \\
%\tilde{S} &\hookrightarrow \tilde{X}, \ p \mapsto (p,0).
%\end{align*}
Denote by $\widehat{X}_F$ the formal neighbourhood of $F$ in $X$ and by $\widehat{Y}_{F}$ the formal neighbourhood of
$$
F \cong \{x\} \times F \subset \tilde{S} \subset Y,
$$
where $\tilde{S} \subset Y$ denotes the zero section. Certainly $\widehat{X}_F$ and $\widehat{Y}_{F}$ are \emph{not} isomorphic. 

Next, denote by $\widehat{X}^{\circ}_{F^\circ}$ the formal neighbourhood of $F \setminus B$ in $X \setminus B$. Moreover, we denote by 
$$
\widehat{Y}_{F^{\circ}}^{\circ}
$$
the formal neighbourhood of $F \setminus (B \times \{x\})$ inside $Y \setminus (B \times \{x\})$. Recall that $x \in F \cap B$ and $B \times \{x\} \subset \tilde{S} \subset Y$. After removing the section $B$, there exists an isomorphism \todo{Provide more argument here?} 
\begin{equation} \label{isopuncturednghs}
\widehat{X}^{\circ}_{F^\circ} \cong \widehat{Y}_{F^{\circ}}^{\circ}.
\end{equation}
We are interested in the moduli space $\Hilb^{a,\bullet}(\widehat{X}^{\circ}_{F^\circ})$ and the correspondingly defined moduli space $\Hilb^{a,\bullet}(\widehat{Y}^{\circ}_{F^\circ})$. Since $\widehat{X}^{\circ}_{F^\circ}$, $\widehat{Y}^{\circ}_{F^\circ}$ have (compatible) $\CC^*$-actions coming from scaling the fibres of $X$, $Y$, we can consider their $\CC^*$-fixed loci and stratify them according to 2D partitions as in \eqref{comps}. By \eqref{isopuncturednghs}, we have obtain an isomorphism   
$$
\Hilb^{a,\bullet}(\widehat{X}^{\circ}_{F^\circ})_{\lambda}^{\CC^*} \cong \Hilb^{a,\bullet}(\widehat{Y}^{\circ}_{F^\circ})_{\lambda}^{\CC^*}.
$$
This observation allows us to work in the much simpler geometry of $Y$. Note that this is only possible because we removed the section $B$ from $X$ and $Y$.

Let $F \subset \tilde{S} \subset Y$ be as above. Denote by $\widehat{Y}_{y}$ the formal neighbourhood $y \in Y$, where $y$ is the intersection of $F \cong \{x\} \times F$ and $B \times \{x\}$ inside the zero section $\tilde{S} \subset Y$. Let $\widehat{Y}_F$, $\widehat{Y}^{\circ}_{F^\circ}$ be the formal neighbourhoods introduced above. Then we have an fpqc cover
$$
\{\widehat{Y}_y \rightarrow \widehat{Y}_F, \widehat{Y}^{\circ}_{F^\circ} \rightarrow \widehat{Y}_F\}. 
$$
On these pieces, we introduce moduli spaces as in Section \ref{formal}
\begin{align*}
\Hilb^{a,\bullet}(\widehat{Y}_y), \ \Hilb^{a,\bullet}(\widehat{Y}_F), \ \Hilb^{a,\bullet}(\widehat{Y}^{\circ}_{F^\circ}).
\end{align*}
Similar to Proposition \ref{bij}, restriction gives a bijective morphism on closed points
$$
\Hilb^{a,\bullet}(\widehat{Y}_F)^{\CC^*}_{\lambda} \rightarrow \Hilb^{a,\bullet}(\widehat{Y}_y)^{\CC^*}_{\lambda} \times \Hilb^{a,\bullet}(\widehat{Y}^{\circ}_{F^\circ})^{\CC^*}_{\lambda}.
$$

Recall that $\tilde{S} = B \times F$. Therefore $F$ does not only act on $F \subset \tilde{S}$, but on any thickening $d F \subset \tilde{S}$. This is because
$$
\O_{dF} = \O_{d x} \otimes \O_F,
$$
where $dx \subset B$ denotes the $d$ times thickening of the point $x \in B$. Moreover, $F$ acts on the thickened curve defined by the ideal sheaf
$$
\bigoplus_{k=0}^{\infty} \O_{\tilde{S}}(-\lambda_k F) \otimes K_{\tilde{S}}^{-k}.
$$
The action of the elliptic curve $F$ on itself is fixed-point-free, so it lifts to a free action on $\Hilb^{a,\bullet}(\widehat{Y}_F)^{\CC^*}_{\lambda}$. Since $e(F) = 0$, we deduce
\begin{equation} \label{e=1}
e(\Hilb^{a,\bullet}(\widehat{Y}_F)^{\CC^*}_{\lambda}) = 1.
\end{equation}
Finally, since $\widehat{Y}_y \cong \Spec \CC[\![x_1,x_2,x_3]\!]$, we have an action of $\CC^{*3}$ and 
\begin{equation} \label{001}
e(\Hilb^{a,\bullet}(\widehat{Y}_y)^{\CC^*}_{\lambda}) = e(\Hilb^{a,\bullet}(\widehat{Y}_y)^{\CC^{*3}}_{\lambda}) = \sfV_{\lambda,\varnothing,\varnothing}(p).
\end{equation}
We conclude from equations \eqref{isopuncturednghs}, \eqref{e=1}, \eqref{001} that
\begin{align*}
1=e(\Hilb^{a,\bullet}(\widehat{Y}_F)^{\CC^*}_{\lambda}) &= e( \Hilb^{a,\bullet}(\widehat{Y}_y)^{\CC^*}_{\lambda}) e(\Hilb^{a,\bullet}(\widehat{Y}^{\circ}_{F^\circ})^{\CC^*}_{\lambda}) \\
&=  \sfV_{\lambda,\varnothing,\varnothing}(p) \cdot e(\Hilb^{a,\bullet}(\widehat{Y}^{\circ}_{F^\circ})^{\CC^*}_{\lambda}) \\
&=  \sfV_{\lambda,\varnothing,\varnothing}(p) \cdot e(\Hilb^{a,\bullet}(\widehat{X}^{\circ}_{F^\circ})^{\CC^*}_{\lambda}).
\end{align*}

The equation for $e(\Hilb^{b,\bullet}(\widehat{X}^{\circ}_{F_{y}^{\circ}})_{\mu}^{\CC^*})$ can be deduced similarly. This time, the smooth fibre $F = F_x \subset S \subset X$ is replaced by the \emph{smooth locus} of the singular fibre, i.e.~ 
$$
F' := F_{y}^{\sm} = F_{y} \setminus \{z\},
$$
where $z$ denotes the singularity of $F_y$. Note that
$$
F' \cong \PP^1 \setminus \{2 \ \rm{points}\} \cong \CC^*.
$$
Therefore, we again have a free action of $F'$ on itself and $e(F') = 0$. The rest of the proof follows the same steps.
\end{proof}   


\subsection{Main theorem}

Combining Proposition \ref{geomgh} with Propositions \ref{vertex1}, \ref{vertex2} immediately gives:
\begin{proposition} \label{combgh}
For any $a,b>0$ 
\begin{align}
\begin{split} \label{gh}
g(a) &= (1-p) \sum_{\lambda \vdash a} \frac{\sfV_{\lambda,\square,\varnothing}(p)}{\sfV_{\lambda,\varnothing,\varnothing}(p)}, \\
h(b) &= \frac{1-p}{M(p)} \sum_{\mu \vdash b} \frac{\sfV_{\mu,\square,\varnothing}(p) \sfV_{\mu,\mu',\varnothing}(p) p^{|\!|\mu|\!|}}{\sfV_{\mu,\varnothing,\varnothing}(p)}.
\end{split}
\end{align}
\end{proposition}

Putting all our results together, we obtain our main theorem:
\begin{theorem} \label{main}
%For any elliptic surface $p : S \rightarrow B$ over a smooth projective curve $B$ with unique section, $N$ rational nodal fibres and no further singular fibres, we have
\begin{align*}
&\widehat{\DT}(X) = \frac{1}{(p^{\frac{1}{2}} - p^{-\frac{1}{2}})^{e(B)}} \Bigg( \sum_{\lambda} \frac{\sfV_{\lambda,\square,\varnothing}(p)}{\sfV_{\lambda,\varnothing,\varnothing}(p)} q^{|\lambda|} \Bigg)^{e(B) - N}  \Bigg( \sum_{\mu} \frac{\sfV_{\mu,\square,\varnothing}(p) \sfV_{\mu,\mu',\varnothing}(p) p^{|\!|\mu|\!|}}{\sfV_{\mu,\varnothing,\varnothing}(p)} q^{|\mu|} \Bigg)^{N} \\
&\widehat{\DT}^{\conn}(X) = \frac{1}{(p^{\frac{1}{2}} - p^{-\frac{1}{2}})^{e(B)}} \Bigg( \frac{\sum_{\lambda} \frac{\sfV_{\lambda,\square,\varnothing}(p)}{\sfV_{\lambda,\varnothing,\varnothing}(p)} q^{|\lambda|}}{\sum_{\lambda} q^{|\lambda|}} \Bigg)^{e(B) - N}  \Bigg( \frac{\sum_{\mu} \frac{\sfV_{\mu,\square,\varnothing}(p) \sfV_{\mu,\mu',\varnothing}(p) p^{|\!|\mu|\!|}}{\sfV_{\mu,\varnothing,\varnothing}(p)} q^{|\mu|}}{\sum_{\mu} \sfV_{\mu, \mu', \varnothing}(p) p^{|\!|\mu|\!|} q^{|\mu|}} \Bigg)^{N}.
\end{align*}
%where the sums are over all 2D partitions and where $\sfV_{\lambda,\mu,\nu}(p)$ denotes the topological vertex.
\end{theorem}
\begin{proof}
Inserting the equations for $g(a)$, $h(b)$ of Proposition \ref{combgh} into \eqref{initialprod} gives the formula for $\widehat{\DT}(X)$. Following the exact same line of reasoning of Sections \ref{sym}--\ref{vertex}, it is easy to see that
\begin{align*}
\sum_{d \geq 0} \sum_{n \in \ZZ} e(\Hilb^{dF,n}(X)) p^n q^d = \Bigg( \sum_{\lambda} q^{|\lambda|} \Bigg)^{e(B) - N} \Bigg( \sum_{\mu} \sfV_{\mu,\mu',\varnothing}(p) p^{|\!|\mu|\!|} q^{|\lambda|} \Bigg)^{N}.
\end{align*}
The formula for $\widehat{\DT}^{\conn}(X)$ follows after dividing $\widehat{\DT}(X)$ by this expression.
\end{proof}


\section{Introducing the Behrend function} \label{Behrend}


\appendix
\section{Odds and Ends}\label{appendix: odds and ends}

\subsection{Curves on elliptic surfaces}\label{appHilb}

Let $p : S \rightarrow B$ be an elliptic surface with section $B \subset S$. In this section we allow any type of singular fibres. We assume $S$ is not a product, which implies
$$
p^* : \Pic^0(B) \stackrel{\cong}{\longrightarrow} \Pic^0(S)
$$
is an isomorphism \cite[VII.1.1]{Mir}. For any $\beta \in H_2(S)$, we denote by $\Hilb^\beta(S)$ the Hilbert scheme of effective divisors on $S$ in class $\beta$. 

Denote by $B \in H_2(S)$ the class of the section and by $F \in H_2(S)$ the class of the fibre. Then we have the following commutative diagram 
\begin{displaymath}
\xymatrix
{
\Sym^d(B) \ar[d]^{p^*} \ar[r] & \Pic^d(B) \ar[d]^{p^*}_{\cong} \\
\Hilb^{dF}(S) \ar[d]^{+B} \ar[r] & \Pic^{dF}(S) \ar[d]^{\otimes \O_S(B)}_{\cong} \\
\Hilb^{B+dF}(S) \ar[r] & \Pic^{B+dF}(S). 
}
\end{displaymath}
Here the horizontal arrows are Abel-Jacobi maps and the vertical arrows are induced by pull-back and multiplication by the section defining $B \subset S$. The following is the main result of this appendix:
\begin{lemma} \label{Hilbcvs}
Let $g$ be the genus of $B$. If $d > 2g - 2$, then the above maps induces an isomorphism
$$
\Sym^d(B) \stackrel{\cong}{\longrightarrow} \Hilb^{B+dF}(S).
$$
\end{lemma}

\begin{remark}
The upshot of this proposition is that we can now \emph{see} the elements of $\Hilb_{B+dF}(S)$ lying in our surface $S$.
\end{remark}

\begin{proof}
Since $d > 2g-2$ the Abel-Jacobi map 
$$
\Sym^d(B) \rightarrow \Pic^d(B)
$$
is a projective bundle with fibres linear systems of dimension $d-g$. We immediately deduce that all horizontal arrows of the commutative diagram are surjective. At the level of fibres, we have inclusions of linear systems
\begin{equation} \label{epsilon}
|\epsilon| \hookrightarrow |p^* \epsilon| \hookrightarrow |p^* \epsilon (B)|,
\end{equation}
where $\epsilon \in \Pic^d(B)$. We want to prove that all these linear systems have the same dimension (Claim). The first map is clearly an isomorphism since
$$
H^0(S,p^* \epsilon) \cong H^0(B, \epsilon \otimes p_* \O_S) = H^0(B,\epsilon)
$$
where we used the projection formula and $p_* \O_S \cong \O_B$. Surjectivity of the second map of \eqref{epsilon} requires more work and is proved below. Before we prove Claim, we show it implies the proposition. Choose Poincar\'e line bundles 
\begin{align*}
\cP &\textrm{ \ on \ } \Pic^d(B) \times B, \\
\cQ &\textrm{ \ on \ } \Pic^{dF}(S) \times S, \\
\cR &\textrm{ \ on \ } \Pic^{B+dF}(S) \times S.
\end{align*}
In each case, denote projection to the base Picard by $\pi_{\Pic}$. Since $d > 2g-2$, the sheaf $\pi_* \mathcal{P}$ is locally free because the dimensions of $|\epsilon|$, $\epsilon \in \Pic^d(B)$ do not jump. Using the isomorphism $\Pic^d(B) \cong \Pic^{dF}(S)$, we see that pull-back of $\mathcal{P}$ along 
$$
\Pic^{dF}(S) \times S \rightarrow \Pic^d(B) \times B
$$
equals $\mathcal{Q}$ (up to tensoring by a line bundle pulled-back from $S$, which we get rid of by redefining $\mathcal{Q}$). We write this as $p^* \mathcal{P} \cong \mathcal{Q}$. Pushing down to $\Pic^{dF}(S)$, using the projection formula and $p_* \O_S \cong \O_B$, we obtain
$$
\pi_{\Pic *} \cP \cong \pi_{\Pic *} \cQ
$$
on $\Pic^{dF}(S) \cong \Pic^{d}(B)$. On $\Pic^{dF}(S) \times S$ we can form $\cQ(B)$ by pulling back $\O_S(B)$ along projection to the second factor. Then
$$
\cR \cong \cQ(B),
$$
on $\Pic^{B+dF}(S) \times S \cong \Pic^{dF}(S) \times S$ (again, after possibly redefining $\cR$). Multiplying by the section defining $B$ and pushing down gives a morphism
\begin{equation} \label{vbmap}
\pi_{\Pic *} \cQ \hookrightarrow (\pi_{\Pic *} \cQ)(B) \cong \pi_{\Pic *} \cR.
\end{equation}
Claim implies that the dimension of the fibres of $|p^* \epsilon(B)|$, $\epsilon \in \Pic^d(B)$ do not jump. Therefore $(\pi_{\Pic *} \cQ)(B) \cong \pi_{\Pic *} \cR$ is locally free. Since \eqref{vbmap} is a morphism of locally free sheaves, it suffices to check we have induced isomorphisms on the fibres. This is exactly the content of Claim. We conclude all Abel-Jacobi maps are projective bundles 
\begin{align*}
\PP(\pi_{\Pic *} \mathcal{P}) &\rightarrow \Pic^d(B), \\
\PP(\pi_{\Pic *} \mathcal{Q}) &\rightarrow \Pic^{dF}(S), \\
\PP(\pi_{\Pic *} \mathcal{R}) &\rightarrow \Pic^{B+dF}(S).
\end{align*}
and 
$$
\PP(\pi_{\Pic *} \mathcal{P}) \cong \PP(\pi_{\Pic *} \mathcal{Q}) \cong \PP\big((\pi_{\Pic *} \mathcal{Q})(B)\big) \cong \PP(\pi_{\Pic *} \mathcal{R}).
$$

What is left is the Claim: the second map of \eqref{epsilon} is an isomorphism for all $\epsilon \in \Pic^d(C)$. For any line bundle $\delta$ on $B$, the Leray spectral sequence yields the following short exact sequence
\[
0\to H^{1} (B,\delta ) \stackrel{\alpha}{\longrightarrow} H^{1} (S,p^{*}\delta ) \longrightarrow H^{0} (B,\delta \otimes R^{1}p _{*}\O_S ) \to 0.
\]
Therefore, if
\begin{equation} \label{ineq}
\deg( \delta) - \deg (R^1 p_* \O_S)^\vee < 0,
\end{equation}
then $\alpha$ is an isomorphism. Assume this is the case. The long exact cohomology sequence associated to 
\[
0\to p^{*}\delta (-B)\to p^{*}\delta \to \O _{B}\otimes p^{*}\delta \to 0
\]
is
\[
\dotsb \to H^{1} (S,p ^{*} \delta )\rt{\alpha }H^{1} (B,\delta )\to H^{2} (S,p ^{*}\delta (-B))\to H^{2} (S,p ^{*}\delta )\to 0,
\]
and since $\alpha $ is a surjection, we get an isomorphism of the last
two terms. We apply Serre duality to that isomorphism and we use the
fact that $K_{S} = p ^{*} (K_{B}\otimes L)$ where $L = \left(R^1 p
_{*}\O _{S} \right)^{\vee }$ \cite[Thm.~12.1]{BPV} to obtain
\begin{equation} \label{SDiso}
H^{0} (S,p ^{*} (\delta ^{-1}\otimes K_{B}\otimes L) (B)) \cong H^{0}(S,p ^{*} (\delta ^{-1}\otimes K_{B}\otimes L)).
\end{equation}
Let $\epsilon \in \Pic^d(B)$ and take
$$
\delta =K_{B}\otimes L\otimes \epsilon ^{-1}.
$$
Then $\delta$ satisfies \eqref{ineq} if and only if $d>2g-2$, which we assumed. The lemma follows from \eqref{SDiso}.
\end{proof}

%Jim's original:
%In this subsection we prove the following lemma and corollary, which will tell us what is the reduced support of all curves in the class $\beta =B+dF$.

%\begin{lemma}\label{lem: H0 (pi* (D) (B))=H0 (pi* (D))}
%For any line bundle $\epsilon $ on $B$, multiplication by the
%canonical section of $\O_S (B)$ induces an isomorphism
%\[
%H^{0} (S,p^{*} (\epsilon ) (B)) \cong H^{0} (S,p^{*} (\epsilon )).
%\]
%\end{lemma}

%\begin{corollary} \label{cor: chow(beta) = sym (B)}
%Let $\beta = B+dF \in H_{2} (S)$. Then the Chow variety of curves in
%the class $\beta $ is isomorphic to $\Sym ^{d} (B)$ where a point
%$\sum _{i}d_{i} x_{i}\in \Sym ^{d} (B)$ corresponds to the curve
%$B+\sum _{i}d_{i} F_{x_{i}}$.
%\end{corollary}

%\begin{proof}
%The corollary follows immediately from the lemma since the Chow
%variety is the space of effective divisors and the lemma implies that
%any effective divisor in the class $\beta $ is a union of the section
%$B$ with an effective divisor pulled back from the base.

%To prove Lemma~\ref{lem: H0 (pi* (D) (B))=H0 (pi* (D))} we proceed as
%follows. For any line bundle $\delta $ on $B$, the Leray spectral
%sequence yields the short exact sequence:\todo{I think arrows Leray go other way around. Anyway... last two terms iso when $2g-2<d$.}
%\[
%0\to H^{0} (B,\delta \otimes R^{1}p _{*}\O_S )\to H^{1} (S,p^{*}\delta )\Rt{\alpha } H^{1} (B,\delta )\to 0,
%\]
%in particular, $\alpha $ is a surjection.

%Then the long exact cohomology sequence associated to 
%\[
%0\to p^{*}\delta \otimes \O_S (-B)\to p^{*}\delta \to \O _{B}\otimes p^{*}\delta \to 0
%\]
%is
%\[
%\dotsb \to H^{1} (S,p ^{*} \delta )\rt{\alpha }H^{1} (B,\delta )\to H^{2} (S,p ^{*}\delta \otimes \O_S (-B))\to H^{2} (S,p ^{*}\delta )\to 0,
%\]
%and since $\alpha $ is a surjection, we get an isomorphism of the last
%two terms. We apply Serre duality to that isomorphism and we use the
%fact that $K_{S} = p ^{*} (K_{B}\otimes L)$ where $L = \left(R^1 p
%_{*}\O _{S} \right)^{\vee }$ \cite[Thm.~12.1]{BPV} to obtain
%\[
%H^{0} (S,p ^{*} (\delta ^{-1}\otimes K_{B}\otimes L) (B)) \cong H^{0}(S,p ^{*} (\delta ^{-1}\otimes K_{B}\otimes L)).
%\]
%Letting $\delta =K_{B}\otimes L\otimes \epsilon ^{-1}$, the lemma is
%proved. 
%\end{proof}

\subsection{Weighted Euler characteristics of symmetric products} \label{power}

In this section we prove the following formula for the weighted Euler
characteristic of symmetric products.

\begin{lemma}\label{lem: formula for euler char of sym products}
Let $B$ be a scheme of finite type over $\CC $ and let $e (B)$ be its
topological Euler characteristic. Let $g:\ZZ _{\geq 0}\to \ZZ (\!(p)\!)$
be any function with $g (0)=1$. Let $f_{d}:\Sym ^{d} (B)\to \ZZ (\!(p)\!)$
be the constructible function defined by $$f_{d} (\mathfrak{a})=\prod _{i}g (a_{i}),$$ for all $\mathfrak{a} = \sum_{i}
a_{i}x_{i} \in \Sym^d(B)$ with $x_i \in B$ distinct closed points. Then
\[
\sum _{d=0}^{\infty } q^{d} \int _{\Sym ^{d} (B)} f_{d} de =
\left(\sum _{a=0}^{\infty }g (a) q^{a} \right)^{e (B)}.
\]
\end{lemma}

\begin{remark} \label{MacD}
In the special case where $g=f_{d}\equiv  1$, the lemma recovers
MacDonald's formula: $$\sum _{d=0}^{\infty }e (\Sym ^{d} (B)) q^{d} =
\frac{1}{(1-q)^{e (B)}}.$$ 

The lemma is essentially a consequence of the existence of a power
structure on the Grothendieck group of varieties definited by
symmetric products and the compatibility of the Euler characteristic
homomorphism with that power structure \todo{Ref? Bryan-Young?} \cite{}. For convenience's
sake, we provide a direct proof here.
\end{remark}
\begin{proof}
The $d$th symmetric product admits a stratification with strata
labelled by partitions of $d$. Associated to any partition of $d$ is a
unique tuple $(m_{1},m_{2},\dots )$ of non-negative integers with
$\sum _{j=1}^{\infty }j m_{j}=d$. The stratum labelled by
$(m_{1},m_{2},\dots )$ parameterizes collections of points where there
are $m_{j}$ points of multiplicity $j$. The full stratification is
given by:
\[
\Sym ^{d} (B) = \bigsqcup_{\begin{smallmatrix} (m_{1},m_{2},\dots )\\
\sum _{j=1}^{\infty }j m_{j}=d  \end{smallmatrix}} \left\{\left(\prod _{j=1}^{\infty }B^{m_{j}} \right) -\Delta  \right\}/ \prod _{j=1}^{\infty }\sigma _{m_{j}} 
\]
where by convention, $B^{0}$ is a point, $\Delta $ is the large
diagonal, and $\sigma _{m}$ is the $m$th symmetric group. Note that
the function $f_{d}$ is constant on each stratum and has value $\prod
_{j=1}^{\infty }g (j)^{m_{j}}$. Note also that the action of $\prod
_{j=1}^{\infty }\sigma _{m_{j}}$ on each stratum is free. 

For schemes over $\CC $, topological Euler characteristic is additive
under stratification and multiplicative under maps which are
(topological) fibrations. Thus
\[
\int _{\Sym ^{d} (B)} f_{d}\,\, de = \sum _{\begin{smallmatrix}(m_{1},m_{2},\dots )\\
\sum _{j=1}^{\infty }j m_{j}=d   \end{smallmatrix}} \left(\prod _{j=1}^{\infty } g (j)^{m_{j}} \right) \frac{e (B^{\sum _{j}m_{j}}-\Delta )}{m_{1}!\, m_{2}!\, m_{3}!\dots }.
\]

For any natural number $N$, the projection $B^{N}-\Delta \to
B^{N-1}-\Delta $ has fibers of the form $B-\{N-1\text{ points}
\}$. The fibers have constant Euler characteristic given by $e (B)-
(N-1)$ and consequently, $e (B^{N}-\Delta )= (e (B)- (N-1))\cdot e
(B^{N-1}-\Delta )$. Thus by induction, we find $e (B^{N}-\Delta ) = e
(B)\cdot (e (B)-1)\cdots (e (B)- (N-1))$ and so 
\[
\frac{e (B^{\sum _{j}m_{j}}-\Delta )}{m_{1}!\,m_{2}!\,m_{3}!\cdots } = \binom{e (B)}{m_{1},m_{2},m_{3},\cdots }
\]
where the right hand side is the generalized multinomial coefficient.

Putting it together and applying the generalized multinomial theorem,
we find
\begin{align*}
\sum _{d=0}^{\infty } q^d \int _{\Sym ^{d} (B)}f_{d}\,\,de & = \sum _{(m_{1},m_{2},\dots )} \prod _{j=1}^{\infty } \left(g (j) q^{j} \right)^{m_{j}} \binom{e (B)}{m_{1},m_{2},m_{3},\dots }\\
&=\left(1+\sum _{j=1}^{\infty }g (j) q^{j} \right)^{e (B)}
\end{align*}
which proves the lemma.   
\end{proof}

     
\begin{thebibliography}{MNOP2}
\bibitem[AM]{AM} M.~F.~Atiyah and I.~G.~MacDonald, \textit{Introduction to commutative algebra}, Westview Press (1969).
%\bibitem[BCC]{BCC} J.~Bouttier, G.~Chapuy, and S.~Corteel, \textit{From Aztec diamonds to pyramids: steep tilings}, arXiv:1407.0665 [math.CO].
\bibitem[Beh]{Beh} K.~Behrend, \textit{Donaldson-Thomas type invariants via microlocal geometry}, Annals of Math.~\textbf{170} (2009), 1307--1338.
%\bibitem[BGK]{BGK} J.~Bryan, F.~Greer, and M.~Kool, \emph{in preparation}.
\bibitem[BKY]{BKY} J.~Bryan, M.~Kool, and B.~Young, \emph{in preparation}.
%\bibitem[BO]{BO} S.~Bloch and A.Okounkov, \textit{The character of the infinite wedge representation}, Adv.~Math.~\textbf{149} (2000), 1--60.
\bibitem[BOPY]{BOPY} J.~Bryan, G.~Oberdieck, R.~Pandharipande, and Q.~Yin, \emph{Curve counting on abelian surfaces and threefolds}, arXiv:1506.00841.
\bibitem[BPV]{BPV} W.~Barth, C.~Peters, and A.~van de Ven, \textit{Compact complex surfaces}, Springer-Verlag (1984).
\bibitem[Bri]{Bri} T.~Bridgeland, \textit{An introduction to motivic Hall algebras}, Adv.~Math.~\textbf{229} (2012), 102--138.
\bibitem[Bry]{Bry} J.~Bryan, \textit{The Donaldson-Thomas theory of $K3 \times E$ via the topological vertex}, arXiv:1504.02920.
\bibitem[Cha]{Cha} K.~Chandrasekharan, \emph{Elliptic functions}, Grundlehren Math.~Wiss.~\textbf{281} Springer-Verlag (1985).
\bibitem[Che]{Che} J.~Cheah, \textit{On the cohomology of Hilbert schemes of points}, J.~Alg.~Geom.~\textbf{5} (1996), 479--511.
%\bibitem[HKK]{HKK} M.-x.~Huang, S.~Katz, and A.~Klemm, \textit{Topological string on elliptic CY 3-folds and the ring of Jacobi forms}, arXiv:1501.04891 [hep-th].
\bibitem[JS]{JS} D.~Joyce and Y.~Song, \textit{A theory of generalized Donaldson-Thomas invariants}, Mem.~of the AMS \textbf{217} (2012), 1--216.
%\bibitem[Kac]{Kac} V.~Kac, \textit{Infinite dimensional Lie algebras}, Cambridge University Press (1990).
%\bibitem[KKV]{KKV} S.~Katz, A.~Klemm, and C.~Vafa, \textit{M-theory, topological strings, and spinning black holes}, Adv.~Theor.~Math.~Phys. \textbf{3} (1999), 1445--1537.
\bibitem[KY]{KY} T.~Kawai and K.~Yoshioka, \emph{String partition functions and infinite products}, Adv.~Theor.~Math.~Phys.~\textbf{4} (2000), 397--485.
\bibitem[Mir]{Mir} R.~Miranda, \textit{The basic theory of elliptic surfaces}, Dottorato di Ricerca in Math., ETS Editrice Pisa (1989).
\bibitem[MNOP1]{MNOP1} D.~Maulik, N.~Nekrasov, A.~Okounkov and R.~Pandharipande, \textit{Gromov-{W}itten theory and {D}onaldson-{T}homas theory, {I}}, Compos.~Math.~\textbf{142} (2006), 1263--1285. %math.AG/0312059.
\bibitem[MNOP2]{MNOP2} D.~Maulik, N.~Nekrasov, A.~Okounkov and R.~Pandharipande, \textit{Gromov-{W}itten theory and {D}onaldson-{T}homas theory, {II}}, Compos.~Math.~\textbf{142} (2006), 1286--1304. 
%\bibitem[MPT]{MPT} D.~Maulik, R.~Pandharipande and R.~P.~Thomas, \textit{Curves on K3 surfaces and modular forms}, J.~Topol.~\textbf{3}, 937--996 (2010). %arXiv:1001.2719.
%\bibitem[OP]{OP} A.~Okounkov and R.~Pandharipande, \textit{Gromov-Witten theory, Hurwitz theory, and completed cycles}, Annals of Math.~\textbf{163} (2006), 517?560.
%\bibitem[OPa]{OPa} G.~Oberdieck and R.~Pandharipande, \textit{Curve counting on $K3 \times E$, the Igusa cusp form $\chi_{10}$, and descendent integration}, arXiv:1411.1514 [math.AG].
%\bibitem[OR]{OR} A.~Okounkov and N.~Reshetikhin, \textit{Random skew plane partitions and the Pearcey process}, Comm.~in Math.~Phys.~\textbf{269} (2007), 571--609.
\bibitem[ORV]{ORV} A.~Okounkov, N.~Reshetikhin, and C.~Vafa, \textit{Quantum Calabi-Yau and classical crystals}, in: The unity of mathematics, editors: P.~Etingof, V.~Retakh, I.~M.~Singer, Progress in Math.~\textbf{244}, Birkh\"auser (2006).
\bibitem[PT]{PT} R.~Pandharipande and R.~P.~Thomas, \textit{Higher genus curves on K3 surfaces and the Katz-Klemm-Vafa formula}, preprint.
%\bibitem[Sta]{Sta} R.~P.~Stanley, \textit{Enumerative combinatorics}, vol.~2, Cambridge University Press (2001).
\bibitem[Tho]{Tho} R.~P.~Thomas, \textit{A Holomorphic Casson Invariant for Calabi--Yau 3-Folds, and Bundles on K3 Fibrations}, J.~Diff.~Geom.~\textbf{54} (2000) 367--438.
\bibitem[Vis]{Vis} A.~Vistoli, \textit{Notes on Grothendieck Topologies, Fibred Categories and Descent Theory}, in B.~Fantechi, L.~G\"ottsche, L.~Illusie, S.~Kleiman, N.~Nitsure, A.~Vistoli, \textit{Fundamental Algebraic Geometry: Grothendieck's FGA Explained}, Math.~Surveys and Monographs \textbf{123} (2006).
%\bibitem[Tod]{Tod} Y.~Toda, \textit{Stable pairs on local K3 surfaces}, J.~Diff.~Geom.~\textbf{92} (2012), 285--370.
%\bibitem[You]{You} B.~Young, \textit{Counting coloured boxes}, PhD thesis University of British Columbia (2008).
\end{thebibliography}

\end{document}

