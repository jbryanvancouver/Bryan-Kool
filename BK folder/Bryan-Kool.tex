\documentclass{amsart}

\title{Donaldson-Thomas invariants of local elliptic surfaces via the topological vertex}
\author{Jim Bryan and Martijn Kool}
\date{\today}
\address{
Department of Mathematics\\
University of British Columbia \\
Room 121, 1984 Mathematics Road  \\
Vancouver, B.C., Canada V6T 1Z2  
}


%\usepackage{diagrams}
%\usepackage{eepic,epic}

\usepackage{amsmath}
\usepackage{amsmath,amsthm,amsfonts}
\usepackage{times}
\usepackage[all]{xy}
%\usepackage{amstex}


%\newtheorem{thm}{Theorem}%[section]
\newtheorem{theorem}{Theorem}%[section]
\newtheorem{proposition}[theorem]{Proposition}
\newtheorem{conjecture}[theorem]{Conjecture}
\newtheorem{lemma}[theorem]{Lemma}
\newtheorem{corollary}[theorem]{Corollary}
\theoremstyle{definition}

\newtheorem{def-theorem}[theorem]{Definition-Theorem}
\newtheorem{remark}[theorem]{Remark}
\newtheorem{definition}[theorem]{Definition}
\newtheorem{example}[theorem]{Example}


\newcommand{\CC} {\mathbb{C}}          % complex numbers
\newcommand{\NN} {\mathbb{N}}		% natural numbers
\newcommand{\RR} {\mathbb{R}}		% real numbers
\newcommand{\ZZ} {\mathbb{Z}}		% integers
\newcommand{\QQ} {\mathbb{Q}}		% rationals
\newcommand{\PP} {\mathbb{P}}
\renewcommand{\AA} {\mathbb{A}}
\newcommand{\LL} {\mathbb{L}}
\newcommand{\FF} {\mathbb{F}}
\renewcommand{\O}{\mathcal{O}}


\newcommand{\rt}[1]{\stackrel{#1\,}{\rightarrow}}
\newcommand{\Rt}[1]{\stackrel{#1\,}{\longrightarrow}}
\newcommand\To{\longrightarrow}
\newcommand\into{\hookrightarrow}
\newcommand\Into{\ensuremath{\lhook\joinrel\relbar\joinrel\rightarrow}}
\newcommand\INTO{\ar@{^{(}->}[r]}
\newcommand\acts{\curvearrowright}


\newcommand{\Hom}{\operatorname{Hom}}
\newcommand{\Ker}{\operatorname{Ker}}
\newcommand{\End}{\operatorname{End}}
\newcommand{\GL}{\operatorname{GL}}
\newcommand{\Tr}{\operatorname{tr}}
\newcommand{\tr}{\operatorname{tr}}
\newcommand{\Coker}{\operatorname{Coker}}
\newcommand{\im}{\operatorname{Im}}
\newcommand{\M}{\overline{\mathcal{M}}}
\newcommand{\smargin}[1]{\marginpar{\tiny{#1}}}
\newcommand{\Sym}{\operatorname{Sym}}
\newcommand{\Coh}{\operatorname{Coh}}


\begin{document}

\begin{abstract}
We compute the Donaldson-Thomas invariants of a local elliptic surface
with section. We introduce a new computational technique which is a
mixture of motivic and toric methods. This allows us to write the
partition function for the invariants in terms of the topological
vertex. Utilizing identities for the topological vertex (some
previously known, some new), we derive product formulas for the
partition functions. In the special case where the elliptic surface is
a K3 surface, we get a new proof of the Katz-Klemm-Vafa formula.
\end{abstract}

\maketitle 

%\markboth{???}  {???}
%\renewcommand{\sectionmark}[1]{}


%\tableofcontents
%\pagebreak


\section{Introduction}

I really need to start working on this.



   

\appendix
\section{Odds and Ends}\label{appendix: odds and ends}


\subsection{Weighted Euler characteristics of symmetric products}

In this section we prove the following formula for the weighted Euler
characteristic of symmetric products.

\begin{lemma}\label{lem: formula for euler char of sym products}
Let $S$ be a scheme of finite type over $\CC $ and let $e (S)$ be its
topological Euler characteristic. Let $g:\ZZ _{\geq 0}\to \ZZ ((Q))$
be any function with $g (0)=1$. Let $f_{d}:\Sym ^{d} (S)\to \ZZ ((Q))$
be the constructible function defined by $f_{d} (\sum_{i}
a_{i}x_{i})=\prod _{i}g (a_{i})$. Then
\[
\sum _{d=0}^{\infty } u^{d} \int _{\Sym ^{d} (S)} f_{d} de =
\left(\sum _{a=0}^{\infty }g (a) u^{a} \right)^{e (S)}.
\]
\end{lemma}

\begin{remark}
In the special case where $g=f_{d}\equiv  1$, the lemma recovers
Macdonald's formula: $\sum _{d=1}^{\infty }e (\Sym ^{d} (S)) u^{d} =
(1-u)^{-e (S)}$. 

The lemma is essentially a consequence of the existence of a power
structure on the Grothendieck group of varieties definited by
symmetric products and the compatibility of the Euler characteristic
homomorphism with that power structure \cite{}. For convenience's
sake, we provide a direct proof here.
\end{remark}
\begin{proof}
The $d$th symmetric product admits a stratification with strata
labelled by partitions of $d$. Associated to any partition of $d$ is a
unique tuple $(m_{1},m_{2},\dots )$ of non-negative integers with
$\sum _{j=1}^{\infty }j m_{j}=d$. The stratum labelled by
$(m_{1},m_{2},\dots )$ parameterizes collections of points where there
are $m_{j}$ points of multiplicity $j$. The full stratification is
given by:
\[
\Sym ^{d} (S) = \bigsqcup_{\begin{smallmatrix} (m_{1},m_{2},\dots )\\
\sum _{j=1}^{\infty }j m_{j}=d  \end{smallmatrix}} \left\{\left(\prod _{j=1}^{\infty }S^{m_{j}} \right) -\Delta  \right\}/ \prod _{j=1}^{\infty }\sigma _{m_{j}} 
\]
where by convention, $S^{0}$ is a point, $\Delta $ is the large
diagonal, and $\sigma _{m}$ is the $m$th symmetric group. Note that
the function $f_{d}$ is constant on each stratum and has value $\prod
_{j=1}^{\infty }g (j)^{m_{j}}$. Note also that the action of $\prod
_{j=1}^{\infty }\sigma _{m_{j}}$ on each stratum is free. 

For schemes over $\CC $, topological Euler characteristic is additive
under stratification and multiplicative under maps which are
(topological) fibrations. Thus
\[
\int _{\Sym ^{d} (S)} f_{d}\,\, de = \sum _{\begin{smallmatrix}(m_{1},m_{2},\dots )\\
\sum _{j=1}^{\infty }j m_{j}=d   \end{smallmatrix}} \left(\prod _{j=1}^{\infty } g (j)^{m_{j}} \right) \frac{e (S^{\sum _{j}m_{j}}-\Delta )}{m_{1}!\, m_{2}!\, m_{3}!\dots }.
\]

For any natural number $N$, the projection $S^{N}-\Delta \to
S^{N-1}-\Delta $ has fibers of the form $S-\{N-1\text{ points}
\}$. The fibers have constant Euler characteristic given by $e (S)-
(N-1)$ and consequently, $e (S^{N}-\Delta )= (e (S)- (N-1))\cdot e
(S^{N-1}-\Delta )$. Thus by induction, we find $e (S^{N}-\Delta ) = e
(S)\cdot (e (S)-1)\cdots (e (S)- (N-1))$ and so 
\[
\frac{e (S^{\sum _{j}m_{j}}-\Delta )}{m_{1}!\,m_{2}!\,m_{3}!\cdots } = \binom{e (S)}{m_{1},m_{2},m_{3},\cdots }
\]
where the right hand side is the generalized multinomial coefficient.

Putting it together and applying the generalized multinomial theorem,
we find
\begin{align*}
\sum _{d=0}^{\infty } \int _{\Sym ^{d} (S)}f_{d}\,\,de & = \sum _{(m_{1},m_{2},\dots )} \prod _{j=1}^{\infty } \left(g (j) u^{j} \right)^{m_{j}} \binom{e (S)}{m_{1},m_{2},m_{3},\dots }\\
&=\left(1+\sum _{j=1}^{\infty }g (j) u^{j} \right)^{e (S)}
\end{align*}
which proves the lemma.   
\end{proof}

\subsection{Some geometry of curves on elliptic surfaces}\label{subsec: geometry of curves on elliptic surfaces}

In this subsection we prove the following lemma and corollary, which will tell us what is the reduced support of all curves in the class $\beta =B+dF$.

\begin{lemma}\label{lem: H0 (pi* (D) (B))=H0 (pi* (D))}
For any line bundle $\epsilon $ on $B$, multiplication by the
canonical section of $\O (B)$ induces an isomorphism
\[
H^{0} (S,\pi ^{*} (\epsilon ) (B)) \cong H^{0} (S,\pi ^{*} (\epsilon )).
\]
\end{lemma}

\begin{corollary}\label{cor: chow(beta) = sym (B)}
Let $\beta = B+dF \in H_{2} (S)$. Then the Chow variety of curves in
the class $\beta $ is isomorphic to $\Sym ^{d} (B)$ where a point
$\sum _{i}d_{i} x_{i}\in \Sym ^{d} (B)$ corresponds to the curve
$B+\sum _{i}d_{i}\pi ^{-1} (x_{i}) $.
\end{corollary}

\begin{proof}
The corollary follows immediately from the lemma since the Chow
variety is the space of effective divisors and the lemma implies that
any effective divisor in the class $\beta $ is a union of the section
$B$ with an effective divisor pulled by from the base.

To prove lemma~\ref{lem: H0 (pi* (D) (B))=H0 (pi* (D))} we proceed as
follows. For any line bundle $\delta $ on $B$, the Leray spectral
sequence yields the short exact sequence:
\[
0\to H^{0} (B,\delta \otimes R^{1}\pi _{*}\O )\to H^{1} (S,\pi ^{*}\delta )\Rt{\alpha } H^{1} (B,\delta )\to 0,
\]
in particular, $\alpha $ is a surjection.

Then the long exact cohomology sequence associated to 
\[
0\to \pi ^{*}\delta \otimes \O (-B)\to \pi ^{*}\delta \to \O _{B}\otimes \pi ^{*}\delta \to 0
\]
is
\[
\dotsb \to H^{1} (S,\pi ^{*} (\delta ))\rt{\alpha }H^{1} (B,\delta )\to H^{2} (S,\pi ^{*}\delta \otimes \O (-B))\to H^{2} (S,\pi ^{*}\delta )\to 0,
\]
and since $\alpha $ is a surjection, we get an isomorphism of the last
two terms. We apply Serre duality to that isomorphism and we use the
fact that $K_{S} = \pi ^{*} (K_{B}\otimes L)$ where $L = \left(R\pi
_{*}\O _{S} \right)^{\vee }$ \cite[prop?]{Fr-Mo} to obtain
\[
H^{0} (S,\pi ^{*} (\delta ^{-1}\otimes K_{B}\otimes L) (B)) \cong H^{0}(S,\pi ^{*} (\delta ^{-1}\otimes K_{B}\otimes L)).
\]
Letting $\delta =K_{B}\otimes L\otimes \epsilon ^{-1}$, the lemma is
proved.
\end{proof}


     
\bibliography{mainbiblio}
\bibliographystyle{plain}

\end{document}

