\documentclass[12pt]{amsart}
%\documentclass[12pt]{article}

\usepackage[english]{babel}

%\usepackage[pdftex,paper=a4paper,portrait=true,textwidth=450pt,textheight=675pt,tmargin=3cm,marginratio=1:1]{geometry}
\usepackage[pdftex,textwidth=400pt,marginratio=1:1]{geometry}

\usepackage{amsfonts}

\usepackage[dvips]{graphics}

\usepackage{amsmath}

\usepackage{amsthm}

\usepackage{amssymb}

\usepackage{eufrak}

\usepackage{cancel}

\usepackage{color}

\usepackage[curve]{xypic}

%\usepackage{mnsymbol}

%\usepackage{stmaryrd}

%\input xy

%\xyoption{all}

\newtheorem{theorem}{Theorem}[section]

\newtheorem{corollary}[theorem]{Corollary}

\newtheorem{proposition}[theorem]{Proposition}

\newtheorem{proposition-definition}[theorem]{Proposition-Definition}


\newtheorem{lemma}[theorem]{Lemma}

\newtheorem{conjecture}[theorem]{Conjecture}

\newtheorem{assumption}[theorem]{Assumption}

\newtheorem{remark}[theorem]{Remark}

\theoremstyle{definition}

\newtheorem{definition}[theorem]{Definition}

\theoremstyle{property}

\newtheorem{property}[theorem]{Property}

\makeatletter
\newcommand{\contraction}[5][1ex]{%
  \mathchoice
    {\contraction@\displaystyle{#2}{#3}{#4}{#5}{#1}}%
    {\contraction@\textstyle{#2}{#3}{#4}{#5}{#1}}%
    {\contraction@\scriptstyle{#2}{#3}{#4}{#5}{#1}}%
    {\contraction@\scriptscriptstyle{#2}{#3}{#4}{#5}{#1}}}%
\newcommand{\contraction@}[6]{%
  \setbox0=\hbox{$#1#2$}%
  \setbox2=\hbox{$#1#3$}%
  \setbox4=\hbox{$#1#4$}%
  \setbox6=\hbox{$#1#5$}%
  \dimen0=\wd2%
  \advance\dimen0 by \wd6%
  \divide\dimen0 by 2%
  \advance\dimen0 by \wd4%
  \vbox{%
    \hbox to 0pt{%
      \kern \wd0%
      \kern 0.5\wd2%
      \contraction@@{\dimen0}{#6}%
      \hss}%
    \vskip 0.5ex%  how far above the line starts
    \vskip\ht2}}
\newcommand{\contracted}[5][1ex]{%
  \contraction[#1]{#2}{#3}{#4}{#5}\ensuremath{#2#3#4#5}}
\newcommand{\contraction@@}[3][0.05em]{%
% the 1st parameter (explicitely inserted) is the width % of the contraction line
  \hbox{%
    \vrule width #1 height 0pt depth #3%
    \vrule width #2 height 0pt depth #1%
    \vrule width #1 height 0pt depth #3%
    \relax}}
\makeatother

%%Richard's macros%%
\DeclareFontFamily{OT1}{rsfs}{}
\DeclareFontShape{OT1}{rsfs}{n}{it}{<-> rsfs10}{}
\DeclareMathAlphabet{\curly}{OT1}{rsfs}{n}{it}

%\newcommand{\nocontentsline}[3]{}
%\newcommand{\tocless}[2]{\bgroup\let\addcontentsline=\nocontentsline#1{#2}\egroup}


%\newcommand\A{\mathcal A}
%\newcommand\I{\curly I}
\newcommand\I{\mathcal I}
\newcommand\J{\mathcal J}

\newcommand\K{K\"ahler~}
\renewcommand\L{\mathcal L}
\newcommand\LL{\mathbb L}
\renewcommand\S{\mathcal S}
\renewcommand\O{\mathcal O}
\newcommand\PP{\mathbb P}

\newcommand\cP{\curly P}
\newcommand\Pb{\mathcal P_\beta}

\newcommand\Db{\overline{\!D}}

\newcommand\D{\mathcal D}

\newcommand\cE{\mathcal E}

\newcommand\F{\mathcal F}

\newcommand\G{\mathcal G}

\newcommand\U{\mathfrak U}

\newcommand\X{\mathfrak X}

\newcommand\Mb{\,\overline{\!M}}
\newcommand\MMb{\,\overline{\!\mathcal M}}
\newcommand\Xb{\,\overline{\!X}}
\newcommand\XXb{\,\overline{\!\mathcal X}}
\newcommand\C{\mathbb C}

\newcommand\cB{\mathcal B}

\newcommand\cC{\mathcal C}

\newcommand\A{\mathbb A}

\newcommand\mdot{{\scriptscriptstyle\bullet}}

\newcommand\FF{\mathbb F}

\newcommand\II{\mathbb I}

\newcommand\HH{\mathbb H}

\newcommand\Q{\mathbb Q}

\newcommand\Pg{\mathcal P_\gamma}

\newcommand\R{\mathbb R}
\newcommand\Z{\mathbb Z}

\newcommand\cX{\mathcal X}

\newcommand\cZ{\mathcal Z}

\newcommand\m{\mathfrak m}

\newcommand{\rt}[1]{\stackrel{#1\,}{\rightarrow}}
\newcommand{\Rt}[1]{\stackrel{#1\,}{\longrightarrow}}
\newcommand\To{\longrightarrow}
\newcommand\into{\hookrightarrow}
\newcommand\Into{\ar@{^(->}[r]<-.3ex>}
\newcommand\acts{\curvearrowright}
\newcommand\so{\ \ \Longrightarrow\ }
\newcommand\res{\arrowvert^{}_}
\newcommand\ip{\lrcorner\,}
\newcommand\comp{{\,}_{{}^\circ}}
\renewcommand\_{^{}_}
\newcommand\take{\backslash}
\newcommand{\mat}[4]{\left(\begin{array}{cc} \!\!#1 & #2\!\! \\ \!\!#3 &
#4\!\!\end{array}\right)}
\newcommand\bull{{\scriptscriptstyle\bullet}}
\newcommand\udot{^\bull}
\newcommand\dbar{\overline\partial}
\newcommand\rk{\operatorname{rank}}
\newcommand\tr{\operatorname{tr}}
\newcommand\coker{\operatorname{coker}}
\newcommand\im{\operatorname{im}}
\newcommand\Td{\operatorname{Td}}
\newcommand\ev{\operatorname{ev}}
\renewcommand\div{\operatorname{div}}
\newcommand\id{\operatorname{id}}
\newcommand\vol{\operatorname{vol}}

\newcommand\Coh{\operatorname{Coh}}

\renewcommand\Im{\operatorname{Im}}
\newcommand\Hom{\operatorname{Hom}}
\renewcommand\hom{\curly H\!om}
\newcommand\Ext{\operatorname{Ext}}
\newcommand\ext{\curly Ext}
\newcommand\Aut{\operatorname{Aut}}
\newcommand\ob{\operatorname{ob}}
\newcommand\Pic{\operatorname{Pic}}
\newcommand\Proj{\operatorname{Proj}\,}
\newcommand\Spec{\operatorname{Spec}\,}
\newcommand\Hilb{\operatorname{Hilb}}
\renewcommand\Re{\operatorname{Re}}
\newcommand\Sym{\operatorname{Sym}}
\newcommand\beq[1]{\begin{equation}\label{#1}}
\newcommand\eeq{\end{equation}}
\newcommand\beqa{\begin{eqnarray*}}
\newcommand\eeqa{\end{eqnarray*}}

\setcounter{secnumdepth}{2}
\DeclareRobustCommand{\SkipTocEntry}[4]{}
% \makeatletter
% \newcommand\@dotsep{4.5}
% \def\@tocline#1#2#3#4#5#6#7{\relax
%   \ifnum #1>\c@tocdepth % then omit
%   \else
%     \par \addpenalty\@secpenalty\addvspace{#2}%
%     \begingroup \hyphenpenalty\@M
%     \@ifempty{#4}{%
%       \@tempdima\csname r@tocindent\number#1\endcsname\relax
%     }{%
%       \@tempdima#4\relax
%     }%
%     \parindent\z@ \leftskip#3\relax \advance\leftskip\@tempdima\relax
%     \rightskip\@pnumwidth plus1em \parfillskip-\@pnumwidth
%     #5\leavevmode\hskip-\@tempdima #6\relax
%     \leaders\hbox{$\m@th
%       \mkern \@dotsep mu\hbox{.}\mkern \@dotsep mu$}\hfill
%     \hbox to\@pnumwidth{\@tocpagenum{#7}}\par
%     \nobreak
%     \endgroup
%   \fi}
% \makeatother 

\begin{document}
\begin{proposition} \label{formaliso}
Let $Z \hookrightarrow X$ be a regular embedding of $\C$-schemes of finite type, and let $\I$ be the ideal sheaf of $Z$. Let $\widehat{X}_Z$ be the formal neighbourhood and $N_{Z/X}$ the normal bundle of $Z$ in $X$. Denote by $\widehat{N_{Z/X}}_Z$ the formal neighbourhood of the zero-section $Z$ in $N_{Z/X}$. Then there exists an isomorphism of sheaves of ring $\O_{\widehat{X}_{Z}} \cong \O_{\widehat{N_{Z/X}}_Z}$ commuting with the structure maps to $\O_Z$ if and only if for each $r \geq 1$ there exists a morphism of sheaves of abelian groups
$$
s : \O_X / \I^r \longrightarrow \O_X / \I^{r+1}
$$
with the following properties:
\begin{enumerate}
\item $s$ is a section of the quotient map $\pi : \O_X / \I^{r+1} \longrightarrow \O_X / \I^r$,
\item for each $p \in X$ at the level of stalks at $p$ we have $s(1)=1$ and for any  $i,j,a,b \geq 0$ such that $i+a = j+b$, $x \in \I_{p}^{i} / \I_{p}^{i+1}$, $y \in \I_{p}^{j} / \I_{p}^{j+1}$ we have
$$
s^a(x) \cdot s^b(y) = \left\{ \begin{array}{cc} s^{a-j}(x \cdot y) & \mathrm{if} \ a \geq j \\ 0 & \mathrm{otherwise} \end{array} \right.,
$$ 
where $s^a = s \circ \cdots \circ s$ denotes the $a$-fold composition of $s$, $s^a(x) \cdot s^b(y) \in \O_{X,p} / \I_{p}^{a+i}$, and $x \cdot y \in \I^{i+j} / \I^{i+j+1}$.
\end{enumerate}
\end{proposition}

\begin{proof}
In this proof we abbreviate $N:=N_{Z/X}$, $\widehat{X} := \widehat{X}_Z$, and $\widehat{N} := \widehat{N_{Z/X}}_Z$. By definition $N = \mathbf{Spec} \ \Sym^{\mdot} \I / \I^2$. Since the embedding is regular we have an isomorphism \cite[B.7.1]{Ful}
$$
\Sym^{\mdot} \I / \I^2 \cong \bigoplus_{i \geq 0} \I^i / \I^{i+1}, 
$$
where $\I^0 := \O_X$. We denote the ideal inside $\bigoplus_{i \geq 0} \I^i / \I^{i+1}$ generated by $\I / \I^2$ by $\langle \I \rangle$. Then $\langle \I \rangle$ is the ideal of the zero-section $Z \hookrightarrow N$.

We first prove the ``only if'' part. Throughout we often use the canonical isomorphism 
$$
\O_{\widehat{X}} / \widehat{\I}^r \cong \O_X / \I^r,
$$
for all $r \geq 1$ \cite[Cor.~II.9.8]{Har}. Since the isomorphism $\O_{\widehat{X}} \cong \O_{\widehat{N}}$ commutes with the structure maps to $\O_Z$ it induces isomorphisms
$$
\O_{\widehat{X}} / \widehat{\I}^r \cong \O_{\widehat{N} / \widehat{\langle \I \rangle}^r},
$$
for all $r \geq 1$. We obtain the following commutative diagram
\begin{displaymath}
\xymatrix
{
\O_X / \I^{r+1} \ar[r] \ar_{\cong}[d] & \O_X / \I^r \ar_{\cong}[d] \\
\O_{\widehat{X}} / \widehat{\I}^{r+1} \ar[r] \ar_{\cong}[d] & \O_{\widehat{X}} / \widehat{\I}^r \ar_{\cong}[d] \\
\O_{\widehat{N}} / \widehat{\langle \I \rangle}^{r+1} \ar[r] \ar_{\cong}[d] & \O_{\widehat{N}} / \widehat{\langle \I \rangle}^r \ar_{\cong}[d] \\
\bigoplus_{i=0}^{r} \I^i / \I^{i+1} \ar[r] & \bigoplus_{i=0}^{r-1} \I^i / \I^{i+1}.
}
\end{displaymath}
The bottom map clearly has a section. It induces the desired section
$$
s : \O_X / \I^r \longrightarrow \O_X / \I^{r+1}
$$
by the diagram. Property (2) is easily verified for these sections.

We now turn to the ``if'' part. We first construct an isomorphism of \emph{sheaves of abelian groups} $\O_{\widehat{N}} \cong \O_{\widehat{X}}$ commuting with the structure maps to $\O_Z$. This part does not use property (2). By definition 
\begin{align*}
\O_{\widehat{X}} &:= \varprojlim \O_X / \I^i \\ 
\O_{\widehat{N}} &:=\varprojlim \O_{N} / \langle \I \rangle^i \cong \prod_{i=0}^{\infty} \I^i / \I^{i+1},
\end{align*}
where $\I^0 := \O_X$. Define the following morphism of sheaves of abelian groups (e.g.~on the level of stalks)
\begin{align}
\begin{split} \label{keymap}
&\prod_{i=0}^{\infty} \I^i / \I^{i+1} \longrightarrow \varprojlim \O_X / \I^i \\
&\{x_i\}_{i=0}^{\infty} \mapsto \Big\{ \sum_{j=0}^{i} s^{i-j}(x_j) \Big\}_{i=0}^{\infty},
\end{split}
\end{align}
where $s^{i-j} = s \circ \cdots \circ s$ denotes the $(i-j)$-fold composition of $s$. This is well-defined because each $s$ is a section and for any $x \in \I^i / \I^{i+1} \subset \O_X / \I^{i+1}$ we have $\pi(x) = 0$. Injectivity of this map is immediate. For given $\{y_i\}_{i=0}^{\infty} \in \varprojlim \O_X / \I^i$ it is easy to solve
$$
\Big\{y_i = \sum_{j=0}^{i} s^{i-j}(x_j) \Big\}_{i=0}^{\infty},
$$
for $x_i \in \O_X / \I^{i+1}$. The projective limit property implies $x_i \in \I^i / \I^{i+1} \subset \O_X / \I^{i+1}$. This proves surjectivity.

It is not hard to check that condition (2) is equivalent to \eqref{keymap} being a \emph{ring homorphism}. 
\end{proof}

Let $\pi : S \rightarrow B$ be a flat morphism of smooth varieties with $\dim(B)=1$ (but $S$ of any dimension) and $F \subset S$ any fibre. There is no reason why $\C^*$ has to act on $F$. Surprisingly the previous proposition can be used to construct an action of $\C^*$ on the \emph{formal neighbourhood} $\widehat{S}_F$ of $F \subset S$. E.g.~when $\pi : S \rightarrow B$ is an elliptic surface this gives $\C^*$-actions on the formal neighbourhood of the fibres even though the fibres themselves need not have natural $\C^*$-actions.
\begin{corollary}
Let $\pi : S \rightarrow B$ be a flat morphism of smooth varieties with $\dim(B)=1$ and let $F$ be any fibre. Denote by $\widehat{\C}_0$ the formal neighbourhood of $\{0\} \subset \C$. Then
$$
\widehat{S}_F \cong F \times \widehat{\C}_0.
$$
In particular, the natural scaling action of $\C^*$ on $\widehat{\C}_0$ induces an action of $\C^*$ on $\widehat{S}_F$.
\end{corollary}
\begin{proof}
Let $p \in B$ be a closed point with ideal sheaf $\I_p$. Then $$\O_B / \I_{p}^{r} \cong \C[\![t]\!] / (t)^r$$ and the natural quotient map
$$
\O_B / \I_{p}^{r+1} \longrightarrow \O_B / \I_{p}^{r}
$$
is just the natural quotient map
$$
\C[\![t]\!] / (t)^{r+1} \longrightarrow \C[\![t]\!] / (t)^r.
$$
This has an obvious section 
$$
s : \sum_{i=0}^{r-1} a_i t^i \in \C[\![t]\!] / (t)^r \mapsto \sum_{i=0}^{r-1} a_i t^i \in \C[\![t]\!] / (t)^{r+1}. 
$$
These sections satisfy property (2) of Proposition \ref{formaliso}. Let $\I_F$ be the ideal sheaf of $F \subset X$. Note that $\I_p \cong \O_B(-p)$ and $\I_F \cong \O_S(-F)$ so
$$
\pi^* \I_{p}^r \cong \I_{F}^{r}.
$$
Since $\pi$ is flat 
$$
\pi^* \big( \O_B / \I_{p}^r \big) \cong \pi^*\O_B / \pi^* \I_{p}^r \cong \O_S / \I_{F}^{r}.
$$
Therefore $\pi^* (s)$ are sections of the natural quotient maps
$$
\O_S / \I_{F}^{r+1} \longrightarrow \O_S / \I_{F}^{r}
$$
satisfying property (2). By Proposition \ref{formaliso}
$$
\widehat{S}_F \cong \widehat{N_{F/S}}_F.
$$

Since $\pi$ is flat 
$$
N_{F/S} \cong \pi^* N_{\{p\}/B} \cong \O_F.
$$
We claim this implies
$$
F \times \widehat{\C}_0 \cong \widehat{N_{F/S}}_F.
$$
This follows by showing that for each closed point $p \in F$ we have an isomorphism 
$$
(\O_{X,p} / \I_{p})[\![z]\!] \cong \prod_{i=0}^{\infty} \I_{p}^{i} / \I_{p}^{i+1}.
$$
This can be seen by writing $\I_p = (s)$ for some $s \in \O_{X,p}^{*}$ and sending
$$
\sum_{i=0}^{\infty} a_i z^i \mapsto \Big\{ \sum_{i=0}^{\infty} a_i s^i \Big\}_{i=0}^{\infty}.
$$
The corollary follows.
\end{proof}

A less surprising corollary is the following.
\begin{corollary}
Let $Z \hookrightarrow S$ be a regular embedding of $\C$-schemes of finite type, let $\I$ be the ideal sheaf of $Z$, and let $\L$ be a line bundle on $S$. Let $X = \mathrm{Tot}(\L)$ and denote the formal completion of $S$ resp.~$X$ along $Z$ by $\widehat{S}_Z$, $\widehat{X}_Z$. Suppose
$$
\widehat{S}_Z \cong \widehat{N_{Z/S}}_Z,
$$
and the isomorphism commutes with the structure maps to $\O_Z$. Then 
$$
\widehat{X}_Z \cong \widehat{N_{Z/X}}_Z,
$$
and the isomorphism commutes with the structure maps to $\O_Z$.
\end{corollary}
\begin{proof}
The ideal sheaf of $Z \hookrightarrow X$ is
$$
\J := \I \oplus \L^* \oplus \L^{*2} \oplus \cdots.
$$
Therefore 
\begin{equation} \label{idealJ}
\O_X / \J^r \cong \O_S / \I^r \oplus \L^* / \I^{r-1}\L^* \oplus \cdots \oplus \L^{*(r-1)} / \I \L^{*(r-1)}.
\end{equation}
By Proposition \ref{formaliso} we have sections 
$$
\O_S / \I^r \longrightarrow \O_S / \I^{r+1}
$$
for all $r \geq 1$ satisfying property (2). Using \eqref{idealJ} we easily construct induced sections
$$
\O_X / \J^r \longrightarrow \O_X / \J^{r+1}
$$
satisfying property (2). The corollary follows from Proposition \ref{formaliso}.
\end{proof}


\begin{thebibliography}{MNOP2}
%\bibitem[ACGH]{ACGH} E.~Arbarello, M.~Cornalba, P.~A.~Griffiths, and J.~Harris, \textit{Geometry of algebraic curves}, Volume I, Springer-Verlag (1985).
%\bibitem[AIK]{AIK} A.~Altman, A.~Iarrobino, and S.~Kleiman, \textit{Irreducibility of the compactified Jacobian}, Real and complex singularities (Proc. Ninth Nordic Summer School/NAVF Sympos.~Math., Oslo, 1976) 1--12 (1977).
%\bibitem[Beh]{Beh} K.~Behrend, \textit{Gromov-Witten invariants in algebraic geometry}, Invent.~Math.~127 601--617 (1997). arXiv:alg-geom/9601011v1.
%\bibitem[BF]{BF} K.~Behrend and B.~Fantechi, \textit{The intrinsic normal cone}, Invent.~Math.~128 45--88 (1997). arXiv:alg-geom/9601010v1.
%\bibitem[BGS]{BGS} J.~Brian\c{c}on, M.~Granger, J.-P.~Speder, \textit{Sur le sch\'ema de Hilbert d'une courbe plane}, Ann.~Sci.~de l'~\'Ecole Normale Sup\'erieure 4 14 1--25 (1981).
%\bibitem[BL1]{BL1} J.~Bryan, C.~Leung, \textit{The enumerative geometry of K3 surfaces and modular forms}, J.~Amer.~Math.~Soc.~13 371--410 (2000).
%\bibitem[BL]{BL} J.~Bryan and C.~Leung, \textit{Generating functions for the number of curves on abelian surfaces}, Duke Math.~J.~99 311--328 (1999). arXiv:math/9802125v1.
%\bibitem[Blo]{Blo} S.~Bloch, \textit{Semi-regularity and de Rham cohomology}, Invent.~Math.~17 51--66 (1972).
%\bibitem[BM]{BM} K.~Behrend, Yu.~Manin, \textit{Stacks of stable maps and Gromov--Witten invariants}, Duke Math.~J.~85 1 1--60 (1996).
%\bibitem[BPV]{BPV} W.~Barth, C.~Peters, and A.~van de Ven, \textit{Compact complex surfaces}, Springer-Verlag (1984).
%\bibitem[Bri]{Bri} T.~Bridgeland, \textit{Hall algebras and curve counting}, JAMS 24 969--998 (2011).
%\bibitem[BS]{BS} M.~Beltrametti and A.~J.~Sommese, \textit{Zero cycles and $k$th order embeddings of smooth projective surfaces. With an appendix by Lothar G\"ottsche}, Problems in the theory of surfaces and their classification (Cortona, 1988) Sympos.~Math.~32 33--48 Academic Press (1991).
%\bibitem[Che]{Che} J.~Cheah, \textit{The cohomology of smooth nested Hilbert schemes of points, thesis}, Chicago (1994).
%\bibitem[CK]{CK} H.-l.~Chang and Y.-H.~Kiem, \textit{Poincar\'e invariants are Seiberg-Witten invariants}, to appear in Geom.~and Topol., arXiv:1205.0848.
%\bibitem[DKO]{DKO} M.~D\"urr, A.~Kabanov, and C. Okonek, \textit{Poincar\'e invariants}, Topology 46 225--294 (2007).
%\bibitem[Don]{Don} S.~K.~Donaldson, \textit{Irrationality and the $h$-cobordism conjecture}, J.~Diff.~Geom.~26 141--168 (1987).
%\bibitem[EG]{EG} Edidin and Graham, equivariant GRR.
%\bibitem[EGL]{EGL} G.~Ellingsrud, L.~G\"ottsche, and M.~Lehn, \textit{On the cobordism class of the Hilbert scheme of a surface}, Jour.~Alg.~Geom.~10 81-100 (2001). %arXiv:math/9904095v1.
%\bibitem[FM]{FM} R.~Friedman and J.~W.~Morgan, \textit{Obstruction bundles, semiregularity and Seiberg-Witten invariants}, Comm.~Anal.~Geom.~7 451--495 (1999).
\bibitem[Ful]{Ful} W.~Fulton, \textit{Intersection theory}, Springer-Verlag (1998).
%\bibitem[GH]{GH} Ph.~Griffiths, J.~Harris, \textit{Principles of algebraic geometry}, Wiley Classics Library (1994).
%\bibitem[Got]{Got} L.~G\"ottsche, \textit{A conjectural generating function for numbers of curves on surfaces}, Comm.~Math.~Phys.~196 523-533 (1998).
%\bibitem[GP]{GP} T.~Graber and R.~Pandharipande, \textit{Localization of virtual classes}, Invent.~Math.~135 487--518 (1999). %arXiv:alg-geom/9708001v2.
%\bibitem[GV1]{GV1} R.~Gopakumar and C.~Vafa, \textit{M-theory and topological strings---I}, hep-th/9809187.
%\bibitem[GV2]{GV2} R.~Gopakumar and C.~Vafa, \textit{M-theory and topological strings---II}, hep-th/9812127.
\bibitem[Har]{Har} R.~Hartshorne, \textit{Algebraic geometry}, Springer-Verlag (1977).
%\bibitem[HL]{HL} D.~Huybrechts, M.~Lehn, \textit{The geometry of moduli spaces of sheaves}, Cambridge University Press (2010).
%\bibitem[HTT]{HTT} Y.~Hinohara, K.~Takahashi, H.~Terakawa, \textit{On tensor products of $k$-very ample line bundles}, Proc.~Amer.~Math.~Soc.~133 687--692 (2004).
%\bibitem[HM]{HM} J.~Harris, I.~Morrison, \textit{Moduli of curves}, Springer-Verlag (1998).
%\bibitem[HT]{HT} D.~Huybrechts and R.~P.~Thomas, \textit{Deformation-obstruction theory for complexes via Atiyah and Kodaira-Spencer classes}, Math. Ann. 346 545--569 (2010).
%\bibitem[Iar]{Iar} A.~Iarrobino, \textit{Punctual Hilbert schemes}, Bull.~Amer.~Math.~Soc.~78 819--823 (1972).
%\bibitem[Ill]{Ill} L.~Illusie, \textit{Complexe cotangent et d\'eformations I}, Lecture Notes in Math.~239 Springer-Verlag (1971).
%\bibitem[Jun]{Jun} Jun Li, \textit{A note on enumerating rational curves in a K3 surface}, in ``Geometry and nonlinear partial differential equations'' AMS/IP Studies in Adv.~Math.~29 (2002).
%\bibitem[JS]{JS} D.~Joyce and Y.~Song, \textit{A theory of generalized {D}onaldson-{T}homas invariants}, Memoirs of the AMS (2012). arXiv:0810.5645.
%\bibitem[KL1]{KL1} Y.-H.~Kiem and J.~Li, \textit{Gromov-Witten invariants of varieties with holomorphic 2-forms}, arXiv:0707.2986v1.
%\bibitem[KL2]{KL2} Y.-H.~Kiem and J.~Li, \textit{Localizing virtual cycles by cosections}, JAMS 26 1025--1050 (2013).
%\bibitem[KNS]{KNS} K.~Kodaira, L.~Nirenberg, D.~C.~Spencer, \textit{On the existence of deformations of complex analytic structures}, Ann.~Math.~68 450--459 (1958).
%\bibitem[KS]{KS} K.~Kodaira, D.~C.~Spencer, \textit{A theorem of completeness for complex analytic fibre spaces}, Acta Math.~100 281--294 (1958). 
%\bibitem[KST]{KST} M.~Kool, V.~Shende and R.~P.~Thomas, \textit{A short proof of the G\"ottsche conjecture}, Geom.~Topol.~15 397--406 (2011). arXiv:1010.3211v2.
%\bibitem[Kod]{Kod} K.~Kodaira, \textit{On the structure of compact complex analytic surfaces, I}, Am.~J.~Math.~86 751--798 (1964).
%\bibitem[KT1]{KT1} M.~Kool and R.~P.~Thomas, \textit{Reduced classes and curve counting on surfaces I: theory}, arXiv:1112.3069. 
%\bibitem[KT2]{KT2} M.~Kool and R.~P.~Thomas, \textit{Reduced classes and curve counting on surfaces II: calculations}, arXiv:1112.3070. 
%\bibitem[Kur]{Kur} M.~Kuranishi, \textit{On the locally complete families of complex analytic structures}, Ann.~Math.~75 536--577 (1962).
%\bibitem[Lee]{Lee} J.~Lee, \textit{Family Gromov-Witten invariants for K\"ahler surfaces}, Duke Math.~Jour.~ 123 209--233 (2004).
%\bibitem[LP]{LP} J.~Lee and T.~Parker, \textit{A structure theorem for the Gromov-Witten invariants of K\"ahler surfaces}, J.~Diff.~Geom.~77 483--513 (2007). 
%\bibitem[LT]{LT} J.~Li and G.~Tian, \textit{Virtual moduli cycles and Gromov-Witten invariants of algebraic varieties},  J.~Amer.~Math.~Soc.~11 119--174 (1998). arXiv:alg-geom/9602007v6.
%\bibitem[Man]{Man} M.~Manetti, \textit{Lectures on deformations of complex manifolds}, Rend.~Mat.~Appl.~24 1--183 (2004).
%\bibitem[MNOP1]{MNOP1} D.~Maulik, N.~Nekrasov, A.~Okounkov, and R.~Pandharipande, \textit{Gromov-{W}itten theory and {D}onaldson-{T}homas theory, {I}}, Compos.~Math.~142 1263--1285 (2006). %arXiv:math/0312059v3.
%\bibitem[MNOP2]{MNOP2} D.~Maulik, N.~Nekrasov, A.~Okounkov, and R.~Pandharipande, \textit{Gromov-{W}itten theory and {D}onaldson-{T}homas theory, {II}}, Compos.~Math.~142 1286--1304 (2006). %arXiv:math/0406092v2.
%\bibitem[MOOP]{MOOP} D.~Maulik, A.~Oblomkov, A.~Okounkov, and R.~Pandharipande, \textit{Gromov-{W}itten/{D}onaldson-{T}homas correspondence for toric 3-folds}, Invent.~Math.~186 435--479 (2011).
%\bibitem[Moo]{Moo} J.D.~Moore, \textit{Lectures on Seiberg-Witten invariants}, Lecture Notes in Mathematics 1629, Springer-Verlag (1996).
%\bibitem[MP1]{MP1}  D.~Maulik and R.~Pandharipande, \textit{New calculations in Gromov-Witten theory}, Pure Appl.~Math.~Q.~4, Special Issue: In honor of Fedor Bogomolov, 469--500 (2008). 
%\bibitem[MP2]{MP2} D.~Maulik, R.~Pandharipande, \textit{Gromov-Witten theory and Noether-Lefschetz theory}, arXiv:0705.1653 (2010).
%\bibitem[MPT]{MPT} D.~Maulik, R.~Pandharipande and R.~P.~Thomas, \textit{Curves on K3 surfaces and modular forms}, J.~Topol.~3 937--996 (2010). 
%\bibitem[Mum]{Mum} D.~Mumford, \textit{Lectures on curves on an algebraic surface}, Princeton University Press 1966.
%\bibitem[Pot]{Pot} J.~Le Potier, \textit{Faisceaux semi-stables de dimension 1 sur le plan projectif}, Rev.~Roumaine Math.~Pures Appl.~38 635--678 (1993).
%\bibitem[Pid]{Pid} V.~Ya.~Pidstrigach, \textit{Deformations of instanton surfaces}, Izv.~Akad.~Nauk SSSR Ser.~Mat.~55 318--338 (1991).
%\bibitem[PP1]{PP1} R.~Pandharipande and A.~Pixton, \textit{Gromov-Witten/Pairs descendent correspondence for toric 3-folds}, arXiv:1203.0468.
%\bibitem[PP2]{PP2} R.~Pandharipande and A.~Pixton, \textit{Gromov-Witten/Pairs correspondence for the quintic 3-fold}, arXiv:1206.5490.
%\bibitem[PT1]{PT1} R.~Pandharipande and R.~P.~Thomas, \textit{Curve counting via stable pairs in the derived category}, Invent.~Math.~178 407--447 (2009). %arXiv:0707.2348v3.
%\bibitem[PT2]{PT2} R.~Pandharipande and R.~P.~Thomas, \textit{The 3-fold vertex via stable pairs}, Geom.~and Topol.~13 1835--1876 (2009).
%\bibitem[PT3]{PT3} R.~Pandharipande and R.~P.~Thomas, \textit{Stable pairs and BPS invariants}, J.~Amer.~Math.~Soc.~23 267--297 (2010). %arXiv:0711.3899v3.
%\bibitem[PT4]{PT4} R.~Pandharipande and R.~P.~Thomas, \emph{in preparation}.
%\bibitem[Ran]{Ran} Z.~Ran, \textit{Semiregularity, obstructions and deformations of Hodge classes}, Ann.~Scuola Norm.~Sup.~Pisa Cl.~Sci.~4 28 809--820 (1999).
%\bibitem[Sie]{Sie} B.~Siebert, \textit{Virtual fundamental classes, global normal cones and Fulton's canonical classes}, in: Frobenius manifolds, ed.~K.~Hertling and M.~Marcolli, Aspects Math.~36 341--358, Vieweg (2004).
%\bibitem[Spi]{Spi} H.~Spielberg, \textit{Une formule pour les invariants de Gromov-Witten des vari\'et\'es toriques}, PhD Thesis Universit\'e Louis Pasteur (1999).
%\bibitem[Tau1]{Tau1} C.~H.~Taubes, \textit{The Seiberg-Witten and Gromov invariants}, Math.~Res.~Lett.~2 221--238 (1995).
%\bibitem[Tau2]{Tau2} C.~H.~Taubes, \textit{Gr=SW: counting curves and connections}, J.~Diff.~Geom.~52 453-609 (1999). 
%\bibitem[Tod]{Tod} Y.~Toda, \textit{Curve counting theories via stable objects I.~DT/PT correspondence}, J.~Amer.~Math.~Soc.~23 1119--1157 (2010). 
%\bibitem[Voi1]{Voi1} C.~Voisin, \textit{Hodge theory and complex algebraic geometry I}, Cambridge University Press (2002).
%\bibitem[Voi]{Voi} C.~Voisin, \textit{Hodge loci}, Handbook of moduli (to appear).
%\bibitem[Wit]{Wit} E.~Witten, \textit{Monopoles and four-manifolds}, Math.~Res.~Lett.~1 769--796 (1994). \\
\end{thebibliography}
\end{document}

