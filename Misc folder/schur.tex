\documentclass{amsart}



%\usepackage{diagrams}
%\usepackage{eepic,epic}

\usepackage{amsmath}
\usepackage{amsmath,amsthm,amsfonts}
\usepackage{times}
%\usepackage{amstex}

\usepackage[margin=1in]{geometry}

%\newtheorem{thm}{Theorem}%[section]
\newtheorem{theorem}{Theorem}%[section]
\newenvironment{primedtheorem}[1]
  {\renewcommand{\thetheorem}{\ref{#1}$'$}%
   \addtocounter{theorem}{-1}%
   \begin{theorem}}
  {\end{theorem}}
\newtheorem{lem}[theorem]{Lemma}
\newtheorem{prop}[theorem]{Proposition}
\newtheorem{proposition}[theorem]{Proposition}
\newtheorem{conj}[theorem]{Conjecture}
\newtheorem{cor}[theorem]{Corollary}
\newtheorem{lemma}[theorem]{Lemma}
\newtheorem{corollary}[theorem]{Corollary}
\theoremstyle{definition}
\newtheorem{rem}[theorem]{Remark}

\newtheorem{def-theorem}[theorem]{Definition-Theorem}
\newtheorem{remark}[theorem]{Remark}
\newtheorem{defn}[theorem]{Definition}
\newtheorem{exmpl}[theorem]{Example}

\newcommand{\cnums} {{\mathbb C}}          % complex numbers
\newcommand{\nnums} {{\mathbb N}}		% natural numbers
\newcommand{\rnums} {{\mathbb R}}		% real numbers
\newcommand{\znums} {{\mathbb Z}}		% integers
\newcommand{\qnums} {{\mathbb Q}}		% rationals

\newcommand{\Hom}{\operatorname{Hom}}
\newcommand{\Ker}{\operatorname{Ker}}
\newcommand{\End}{\operatorname{End}}
\newcommand{\Tr}{\operatorname{tr}}
\newcommand{\tr}{\operatorname{tr}}
\newcommand{\Coker}{\operatorname{Coker}}
\newcommand{\im}{\operatorname{Im}}

\renewcommand{\P}{\mathbb{P}}
\newcommand{\M}{\overline{\mathcal{M}}}
\newcommand{\smargin}[1]{\marginpar{\tiny{#1}}}


\begin{document}


\section{Definitions of $P_{\alpha } (Q)$ and $R_{\alpha } (Q)$}

Let $\alpha =(\alpha _{1}\geq  \alpha _{2}\geq \dots )$ be a
partition. We define a Laurent polynomial $P_{\alpha } (Q)$ as follows
\[
P_{\alpha } (Q) = \sum _{k=1}^{\infty } ( Q^{-\alpha _{k}+k-1}-Q^{-\alpha _{k}+k} ).
\]
Note that this is a Laurent \emph{polynomial} since the sum telescopes
for large $k$. Its leading term is $Q^{-\alpha _{1}}$. It can be
expressed in terms of the topological vertex as follows:
\[
P_{\alpha } (Q) = (1-Q)\frac{V_{\alpha ,\square,\emptyset }}{V_{\alpha
,\emptyset ,\emptyset }}.
\]
A closely related quantity shows up in Okounkov's papers. For example
in Okounkov-Pandharipande (math/0204305) page 33, the eigenvalues of
the operator $\mathcal{E}_{0} (z)$ (acting on $|\alpha\rangle $) is
given by $e (\alpha ,z)$ which is equal to $P_{\alpha } (Q)$ upto a
factor of $(Q^{-1/2}-Q^{1/2})$ under the substitution $Q=e^{-z}$.

The second important quantity we consider is $R_{\alpha } (Q)$. We can
define it as follows:
\[
R_{\alpha } (Q) = \sum _{\eta } Q^{|\alpha |-|\eta |} \left(S_{\alpha /\eta } (\mathbf{Q}) \right)^{2}
\]
where $S_{\alpha /\eta }$ is the skew Schur function and $S_{\alpha
/\eta } (\mathbf{Q}) = S_{\alpha /\eta } (1,Q,Q^{2},Q^{3},\dots )$.
In terms of the (PT) vertex, $R_{\alpha }$ is the vertex $V_{\emptyset
,\alpha ,\alpha '}$ normalized to be a series starting with 1.

$R_{\alpha } (Q)$ is a rational function in $Q$. These functions satisfy the following symmetry relation
\[
P_{\alpha } (Q)=P_{\alpha '} (Q^{-1}) \quad \quad R_{\alpha } (Q)=R_{\alpha '} (Q^{-1}) 
\]
where $\alpha '$ is the conjugate partition.

\section{The main conjecture, related propostions, equivalent formulations}

\begin{proposition}
The follow product formulas hold.
\begin{align*}
\sum _{\alpha } u^{|\alpha |} =& \prod _{i=1}^{\infty } (1-u^{i})^{-1}\\
\sum _{\alpha } R_{\alpha } (Q) u^{|\alpha |} =&\prod _{i=1}^{\infty }\left( (1-u^{i})^{-1}\cdot\prod _{m=1}^{\infty } (1-u^{i}Q^{m})^{-m} \right)\\
\sum _{\alpha }P_{\alpha } (Q)u^{|\alpha |}=&\prod _{i=1}^{\infty } (1-u^{i}) (1-Qu^{i})^{-1} (1-Q^{-1}u^{i})^{-1}.
\end{align*}
\end{proposition}
The first formula is elementary and well known, the second formula
follows from the orthogonality properties of skew Schur functions. The
third formula follows easily from Theorem 6.5 of Bloch-Okounkov
(9712009v2). Our main conjecture, extensively checked with Maple (it's
certainly true) is
\begin{conj}
The following formula holds
\[
\sum _{\alpha } P_{\alpha '} (Q) R_{\alpha } (Q ) u^{|\alpha |} = \prod _{i=1}^{\infty }\left((1-Qu^{i})^{-1} (1-Q^{-1}u^{i})^{-1}\prod _{m=1}^{\infty } (1-Q^{m}u^{i})^{-m} \right).
\]
\end{conj}

Using the formulas in the proposition, the above formula is equivalent
to the following weirdly symmetric looking formula
\[
\sum _{\alpha } P_{\alpha '} (Q) R_{\alpha } (Q) u^{|\alpha |}  = \sum _{\alpha ,\mu } P_{\alpha '} (Q) R_{\mu } (Q) u^{|\alpha |+|\mu |}.
\]
Note the presense of the transpose partition $\alpha '$ in the
subscript of $P$ on the left hand sides of the conjecture
formulas. The conjecture is false without it.


   
\end{document}


