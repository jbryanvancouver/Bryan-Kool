\documentclass[12pt]{amsart}

\title{Trace identities for the Topological Vertex}
\author{Jim Bryan, Martijn Kool, Ben Young}
\date{\today}
\address{
Department of Mathematics\\
University of British Columbia \\
Room 121, 1984 Mathematics Road  \\
Vancouver, B.C., Canada V6T 1Z2  
}



%\usepackage{diagrams}
%\usepackage{eepic,epic}

\usepackage{verbatim}
\usepackage{amsmath}
\usepackage{amsmath,amsthm,amsfonts,amssymb}
\usepackage{times}
%\usepackage{amstex}


\newcommand{\cnums} {{\mathbb C}}          % complex numbers
\newcommand{\nnums} {{\mathbb N}}		% natural numbers
\newcommand{\rnums} {{\mathbb R}}		% real numbers
\newcommand{\znums} {{\mathbb Z}}		% integers
\newcommand{\qnums} {{\mathbb Q}}		% rationals

\newcommand{\Vsf}{\mathsf{V}}
\newcommand{\bx}{\square}
\renewcommand{\emptyset}{\varnothing}
\newcommand{\half}{\frac{1}{2}}



\newtheorem{thm}{Theorem}%[section]
\newtheorem{theorem}{Theorem}%[section]
\newtheorem{lem}[thm]{Lemma}
\newtheorem{prop}[thm]{Proposition}
\newtheorem{proposition}[thm]{Proposition}
\newtheorem{conj}[thm]{Conjecture}
\newtheorem{cor}[thm]{Corollary}
\newtheorem{lemma}[theorem]{Lemma}
\newtheorem{corollary}[theorem]{Corollary}
\theoremstyle{definition}
\newtheorem{rem}[thm]{Remark}

\newtheorem{def-thm}[thm]{Definition-Theorem}
\newtheorem{remark}[theorem]{Remark}
\newtheorem{defn}[theorem]{Definition}
\newtheorem{exmpl}[theorem]{Example}


\newcommand{\Hom}{\operatorname{Hom}}
\newcommand{\Ker}{\operatorname{Ker}}
\newcommand{\End}{\operatorname{End}}
\newcommand{\Tr}{\operatorname{tr}}
\newcommand{\tr}{\operatorname{tr}}
\newcommand{\Coker}{\operatorname{Coker}}
\newcommand{\im}{\operatorname{Im}}

\renewcommand{\P}{\mathbb{P}}
\newcommand{\M}{\overline{\mathcal{M}}}
\newcommand{\smargin}[1]{\marginpar{\tiny{#1}}}
\newcommand{\FockSpace}{\Lambda^{\frac{\infty}{2}}V}
\newcommand{\FockSpaceZero}{\Lambda^{\frac{\infty}{2}}_{0}V}
\newcommand{\ZplusHalf}{\znums+{ \half}}
\newcommand{\E}{\mathcal{E}}
\newcommand{\ptotheminusrho}{p^{-\rho}}


\begin{document}

\begin{abstract}
The topological vertex is a universal series which can be regarded as
an object in combinatorics, representation theory, geometry, or
physics. It encodes the combinatorics of 3D partitions, the action of
vertex operators on Fock space, the Donaldson-Thomas theory of toric
Calabi-Yau threefolds, or the open string partition function of
$\cnums^{3}$.

We prove several identities in which a sum over terms involving the
topological vertex is expressed as a closed formula, often a product
of simple terms, closely related to Fourier expansions of Jacobi
forms. We use purely combinatorial and representation theoretic
methods to prove our formulas, but we discuss applications to the
Donaldson-Thomas invariants of elliptically fibered Calabi-Yau
threefolds at the end of the paper. 
\end{abstract}

\maketitle 

%\markboth{???}  {???}
%\renewcommand{\sectionmark}[1]{}


%\tableofcontents
%\pagebreak


\section{Introduction}\label{sec: intro}

The topological vertex $\Vsf_{\lambda \mu \nu}=\Vsf_{\lambda \mu
\nu}(p)$ is a universal formal Laurent series in $p$ depending on a
triple of partitions $(\lambda, \mu, \nu )$. It can be considered as
an object in combinatorics, representation theory, geometry, or
physics. In combinatorics, $\Vsf_{\lambda \mu \nu}$ is the generating
function for the number of 3D partitions with asymptotic legs of type
$(\lambda, \mu, \nu )$ (see Definition~\ref{defn: box counting
vertex}). In representation theory, $\Vsf_{\lambda \mu \nu}$ is given
as the matrix coefficients of a certain vertex operator on Fock space
\cite{Ok-Re-Va}. In geometry, $\Vsf_{\lambda \mu \nu}$ is the basic
building block for computing the Donaldson-Thomas/Gromov-Witten
invariants of toric Calabi-Yau threefolds \cite{MNOP1}; in this
incarnation, it can be realized as the generating function for the
Euler characteristics of certain Hilbert schemes of curves in
$\cnums^{3}$ (see \S~\ref{sec: geometry and applications}). The
topological vertex was first discovered in physics as an open string
partition function in type IIA string theory on $\cnums^{3}$
\cite{AKMV}. An explicit expression for $\Vsf_{\lambda \mu \nu}$ in
terms of Schur functions was given by \cite{Ok-Re-Va} (see
section~\ref{sec: topo vertex and schur fncs}).


In this paper we prove several ``trace identities'' in which a sum
over certain combinations of the vertex is expressed as a closed
formula, often a product of simple terms. The products are closely
related to the Fourier expansions of Jacobi forms. Applications of
these identities are used to compute the Donaldson-Thomas partition
functions of certain Calabi-Yau threefolds in terms of Jacobi forms
\cite{Bryan-Kool,Bryan-K3xE,BOPY}.

\section{Definitions and the main result.}\label{sec: defns and
result}

In this section we give the combinatorial definition of the vertex and
we state our main identities.

\begin{defn}\label{defn: 3D partition asympt to (a,b,c)} Let $(\lambda
,\mu ,\nu )$ be a triple of ordinary partitions. A \emph{3D
partition $\pi $ asymptotic to $(\lambda ,\mu ,\nu )$} is a subset
\[
\pi \subset \left(\znums _{\geq 0} \right)^{3}
\]
satisfying
\begin{enumerate}
\item if any of $(i+1,j,k)$, $(i,j+1,k)$, and $(i,j,k+1)$ is in $\pi
$, then $(i,j,k)$ is also in $\pi $, and
\item
\begin{enumerate}
\item $(j,k)\in \lambda $ if and only if $(i,j,k)\in \pi $ for all $i\gg 0$,
\item $(k,i)\in \mu  $ if and only if $(i,j,k)\in \pi $ for all $j\gg 0$,
\item $(i,j)\in \nu  $ if and only if $(i,j,k)\in \pi $ for all $k\gg 0$.
\end{enumerate}
\end{enumerate}
where we regard ordinary partitions as finite subsets of $\left(\znums
_{\geq 0} \right)^{2}$ via their diagram.
\end{defn}

Intuitively, $\pi $ is a pile of boxes in the positive octant of
3-space.  Condition (1) means that the boxes are stacked stably with
gravity pulling them in the $(-1,-1,-1)$ direction; condition (2)
means that the pile of boxes is infinite along the coordinate axes
with cross-sections asymptotically given by $\lambda $, $\mu $, and
$\nu $.

The subset $\{(i,j,k ): (j,k)\in \lambda \}\subset \pi $ will be
called the \emph{leg} of $\pi $ in the $i$ direction, and the legs in
the $j$ and $k$ directions are defined analogously. Let
\begin{equation*}
\xi _{\pi } (i,j,k) = 1 - \# \text{ of legs of $\pi $ containing }
(i,j,k) .
\end{equation*}

We define the renormalized volume of $\pi $ by
\[
|\pi | = \sum _{(i,j,k)\in \pi } \xi _{\pi } (i,j,k).
\]
Note that $|\pi |$ can be negative.
\begin{defn}\label{defn: box counting vertex}
The topological vertex $\Vsf_{\lambda \mu \nu }$ is defined to be
\[
\Vsf _{\lambda \mu \nu }= \sum _{\pi } p^{|\pi |}
\]
where the sum is taken over all 3D partitions $\pi $ asymptotic to
$(\lambda ,\mu ,\nu )$. We regard $\Vsf _{\lambda \mu \nu }$ as a
formal Laurent series in $p$. Note that $\Vsf _{\lambda \mu \nu }$ is
clearly cyclically symmetric in the indices, and reflection about the
$i=j$ plane yields
\[
\Vsf _{\lambda \mu \nu } = \Vsf _{\mu '\lambda '\nu '}
\]
where $'$ denotes conjugate partition:
\[
\lambda ' = \{(i,j): (j,i)\in \lambda  \}.
\]

\end{defn}
This definition of topological vertex differs from the vertex $C
(\lambda ,\mu ,\nu )$ of the physics literature by a normalization
factor (and we use the variable $p$ instead of $q$). Our $\Vsf
_{\lambda \mu \nu }$ is equal to $P (\lambda ,\mu ,\nu )$ defined by
Okounkov, Reshetikhin, and Vafa \cite[eqn~3.16]{Ok-Re-Va}. They derive
an explicit formula for $\Vsf _{\lambda \mu \nu }=P (\lambda, \mu, \nu
)$ in terms of Schur functions \cite[eqns~3.20 and 3.21]{Ok-Re-Va}.



The \emph{rows} or \emph{parts} of $\lambda $
are the integers $\lambda _j = \min \{i \;|\;(i,j) \not \in \lambda
\}$, for $j \geq 0$. We use the notation
\[
|\lambda | = \sum_{j} \lambda_{j},\quad \| \lambda \| ^{2} =\sum_{j}\lambda_{j}^{2}.
\]
Let $\bx$ denote the partition with a single part of size 1.

We also use the notation
\[
M(p,q) = \prod_{m=1}^{\infty} (1-p^{m}q)^{-m}
\]
and the shorthand $M(p)=M(p,1)$.

We can now state our main result.

\begin{theorem}\label{thm: main formulas}
The following identities hold as formal power series in $q$ whose
coefficients are formal Laurent series in $p$:
\begin{align}
% &\sum_{\lambda} q^{|\lambda |} = \prod_{m=1}^{\infty} (1-q^{m})^{-1}\label{eqn 1}\\
&\sum_{\lambda} q^{|\lambda |} p^{\| \lambda' \| ^{2}} \Vsf_{\lambda'
\lambda \emptyset }=  M(p) \prod_{d=1}^{\infty} (1-q^{d})^{-1}M(p,q^{d})\label{eqn 2}\\
\quad\nonumber \\
&\sum_{\lambda} q^{|\lambda |}\frac{\Vsf_{\lambda
\bx\emptyset}}{\Vsf_{\lambda \emptyset \emptyset}} =
(1-p)^{-1}\prod_{d=1}^{\infty} \frac{(1-q^{d})}{(1-pq^{d})(1-p^{-1}q^{d})}\label{eqn 3}\\
\quad\nonumber \\
&\sum_{\lambda} q^{|\lambda |} p \frac{\Vsf_{\bx \bx 
\lambda}}{\Vsf_{\emptyset \emptyset \lambda}} =
\prod_{m=1}^{\infty}(1-q^{m})^{-1}\cdot \left\{
1+\frac{p}{(1-p)^{2}}+\sum_{d=1}^{\infty}\sum_{k|d}k(p^{k}+p^{-k})q^{q}\right\}\label{eqn 4}\\
\quad&\quad \nonumber \\
\quad \quad \quad \quad &\sum_{\lambda} q^{|\lambda |} p^{\| \lambda' \| ^{2}}
\frac{\Vsf_{\lambda \bx \emptyset}}{\Vsf_{\lambda \emptyset
\emptyset}}\Vsf_{\lambda' \lambda \emptyset} =
(1-p)^{-1}M(p)\prod_{d=1}^{\infty}
\frac{M(p,q^{d})}{(1-pq^{d})(1-p^{-1}q^{d})}\label{eqn 5}
\end{align}
The sums in the left hand sides of the above formulas run over all
partitions. 
\end{theorem}




We call these formulas ``trace formulas'' since the left hand side can
be expressed as the traces of certain operators on Fock space. This
will be made explicit in section~\ref{sec: vertex ops and the pf of
eqn 5}.

% We note that formula~\eqref{eqn 1} is elementary and well known.
We prove Formula~\eqref{eqn 2} in section~\ref{sec: topo vertex and
schur fncs} using the orthogonality properties of skew Schur
functions. Formulas~\eqref{eqn 3} and \eqref{eqn 4} are proved in
section~\ref{sec: applications of the Bloch-Okounkov thm} using a
theorem of Bloch-Okounkov \cite{Bloch-Okounkov}. The most difficult
identity to prove is equation~\eqref{eqn 5} which we do in
section~\ref{sec: vertex ops and the pf of eqn 5}. There we prove that
the left hand side of equation~\eqref{eqn 5} is given as the trace of
a certain product of operators on Fock space. To compute the trace, we
use a trick which involves an ``infinite number'' of permutations of
the operators.

\section{The topological vertex and Schur functions}\label{sec: topo
vertex and schur fncs}

Okounkov-Reshetikhin-Vafa derived a formula for the topological vertex
in terms of skew Schur functions. Translating their formulas
\cite[3.20\& 3.21]{Ok-Re-Va} into our notation, we get:
\begin{equation}\label{eqn: ORV formula for vertex}
\Vsf_{\lambda \mu \nu}(p) = M(p) p^{-\half (\| \lambda \| ^{2}+\| \mu'
\| ^{2}+\| \nu \| ^{2})} s_{\nu '}(\ptotheminusrho ) \sum_{\eta} s_{\lambda
'/\eta}(p^{-\nu -\rho})s_{\mu /\eta}(p^{-\nu '-\rho} ).
\end{equation}

Here, $s_{\alpha /\beta}(x_{1},x_{2},\dots )$ is the skew Schur
function (see for example \cite[\S~5]{MacDonald}) and 
\[
\rho =\left(-\half ,-\frac{3}{2},-\frac{5}{2},\dots  \right)
\]
so that $p^{-\nu -\rho}$ is notation for the variable list
\[
p^{-\nu -\rho} = \left(p^{-\nu_{1} +\half },p^{-\nu_{2} +\frac{3}{2}},\dots   \right).
\]

We prove equation~\eqref{eqn 2} as follows. 
Using equation~\eqref{eqn: ORV formula for vertex} we see 
\[
\Vsf_{\lambda '\lambda \emptyset} = M(p)p^{-\| \lambda' \| ^{2}}
\sum_{\eta} s_{\lambda /\eta}(\ptotheminusrho )^{2}
\]
and so (using orthogonality of skew Schur functions \cite[28(a) pg
94]{MacDonald} in the second line below) we see
\begin{align*}
\sum_{\lambda} q^{|\lambda |} p^{\| \lambda' \| ^{2}} \Vsf_{\lambda
'\lambda \emptyset} &= M(p)\sum_{\lambda ,\eta} q^{|\lambda |} (s_{\lambda /\eta}(p^{\half },p^{\frac{3}{2}},\dots ))^{2}\\
&=M(p)\prod_{d=1}^{\infty} \left((1-q^{d})^{-1}\prod_{j,k=1}^{\infty}(1-q^{d}p^{i-\half +j-\half })^{-1} \right)\\
&= M(p) \prod_{d=1}^{\infty} (1-q^{d})^{-1}\prod_{m=1}^{\infty}(1-q^{d}p^{m})^{-m}\\
&=M(p)\prod_{d=1}^{\infty} (1-q^{d})^{-1}M(p,q^{d}).
\end{align*}

We also use equation~\eqref{eqn: ORV formula for vertex} to derive the
following key formulas:
\begin{lemma}\label{lem: eqns for Vlambdaboxempty/Vlambdaemptyempty
and Vlambdaboxbox/Vlambdaemptyempty} 
The following hold:
\begin{align*}
p^{\half }\,\,  \frac{\Vsf_{\lambda \bx \emptyset}}{\Vsf_{\lambda
\emptyset \emptyset}} &= \sum_{i=1}^{\infty} p^{-\lambda_{i}+i-\half }\\
p\,\,  \frac{\Vsf_{\lambda \bx \bx}}{\Vsf_{\lambda
\emptyset \emptyset}} &= 1-\left(\sum_{i=1}^{\infty}
p^{-\lambda_{i}+i-\half } \right) \left(\sum_{j=1}^{\infty} p^{\lambda_{j}-j+\half } \right)
\end{align*}
\end{lemma}
\proof Applying equation~\eqref{eqn: ORV formula for vertex} to
$\Vsf_{\lambda \bx \emptyset}/\Vsf_{\lambda \emptyset
\emptyset}=\Vsf_{\bx \emptyset \lambda}/\Vsf_{\emptyset \emptyset
\lambda}$ we see that
\[
p^{\half} \frac{\Vsf_{\lambda \bx \emptyset}}{V_{\lambda \emptyset
\emptyset  }} = s_{\bx}(p^{-\lambda -\rho})=s_{\bx}
(p^{-\lambda_{1}+\half },p^{-\lambda_{2}+\frac{3}{2}},\dots ) =
\sum_{i=1}^{\infty} p^{-\lambda_{i}+i-\half }.
\]
Similarly, 
\begin{align*}
p\frac{\Vsf_{\lambda \bx \bx}}{\Vsf_{\lambda \emptyset \emptyset}} =
p\frac{\Vsf_{\bx \bx \lambda}}{\Vsf_{\emptyset \emptyset \lambda }} &=
\sum_{\eta} s_{\bx /\eta}(p^{-\lambda -\rho} )  s_{\bx /\eta}(p^{-\lambda' -\rho} )  \\
&= 1 + s_{\bx}(p^{-\lambda -\rho })s_{\bx}(p^{-\lambda' -\rho }).
\end{align*}

In general we have the following relation (see
\cite[Eqn~(3.10)]{Ok-Re-Va})\footnote{There is a typo in equation~3.10 in
\cite{Ok-Re-Va} --- the exponent on the right hand side should be $-\nu'-\rho$. }
\[
s_{\lambda /\mu}(p^{\nu +\rho}) = (-1)^{|\lambda |-|\mu |}s_{\lambda
'/\mu '}(p^{-\nu '-\rho})
\]
so in particular $s_{\bx}(p^{\nu +\rho})=-s_{\bx}(p^{-\nu '-\rho})$
and thus
\begin{align*}
p\frac{\Vsf_{\lambda \bx \bx}}{\Vsf_{\lambda \emptyset \emptyset}} &=
1 - s_{\bx}(p^{-\lambda -\rho })s_{\bx}(p^{\lambda +\rho })\\
&= 1-\left(\sum_{i=1}^{\infty} p^{-\lambda_{i}+i-\half }
\right)\left(\sum_{j=1}^{\infty} p^{\lambda_{j}-j+\half } \right)
\end{align*}
which proves the lemma.

\section{Applications of a theorem of Bloch-Okounkov}
\label{sec: applications of the Bloch-Okounkov thm}

We summarize a result of Bloch-Okounkov \cite{Bloch-Okounkov} and use
it to prove equations~\eqref{eqn 3} and \eqref{eqn 4}.

We define the following theta function
\[
\Theta (p,q) = \eta (q)^{-3}\sum_{n\in \znums}
(-1)^{n}q^{\half (n+\half )^{2}} p^{n+\half } 
\]
which, by the Jacobi triple product formula is given by
\[
\Theta (p,q) = (p^{\half} -p^{-\half})\prod_{m=1}^{\infty}
\frac{(1-pq^{m})(1-p^{-1}q^{m})}{(1-q^{m})^{2}}. 
\]

We suppress the $q$ from the notation: $\Theta (p) = \Theta (p,q)$, and
we note that
\[
\Theta (p) = -\Theta (p^{-1}).
\]

\begin{theorem}[Bloch-Okounkov \cite{Bloch-Okounkov}]\label{thm: Bloch-Okounkov thm}
Define the $n$ point correlation function by the formula
\[
F(p_{1},\dots ,p_{n}) = \prod_{m=1}^{\infty}(1-q^{m})\, \sum_{\lambda}
q^{\lambda} \prod_{k=1}^{n} \left(\sum_{i=1}^{\infty} p_{k}^{\lambda_{i}-i+\half } \right).
\]
Then
\[
F(p) = \frac{1}{\Theta (p)}
\]
and
\[
F(p_{1},p_{2}) = \frac{1}{\Theta
(p_{1}p_{2})}\left(p_{1}\frac{d}{dp_{1}}\log(\Theta (p_{1}))+ p_{2}\frac{d}{dp_{2}}\log(\Theta (p_{2})) \right).
\]
\end{theorem}

In \cite{Bloch-Okounkov}, formulas for the general $n$ variable
function are given, but we will only need the cases of $n=1$ and
$n=2$.

Using this theorem, we will prove equations~\eqref{eqn 3} and
\eqref{eqn 4} of the main theorem.

\subsection{Proofs of equations~\eqref{eqn 3} and \eqref{eqn
4}}\label{subsec: pfs of eqn 3 and 4} $\, $
    


We apply Lemma~\ref{lem: eqns for Vlambdaboxempty/Vlambdaemptyempty
and Vlambdaboxbox/Vlambdaemptyempty} and Theorem~\ref{thm: Bloch-Okounkov thm}:
\begin{align*}
\sum_{\lambda}(1-p) q^{|\lambda |} \frac{\Vsf_{\lambda \bx
\emptyset}}{\Vsf_{\lambda \emptyset \emptyset}} &=
(p^{-\half}-p^{\half}) \sum_{\lambda} q^{|\lambda |}\sum_{i=1}^{\infty}p^{-\lambda_{i}+i-\half}\\
&=(p^{-\half}-p^{\half}) \prod_{m=1}^{\infty}(1-q^{m})^{-1} F(p^{-1})\\
&=(p^{-\half}-p^{\half}) \prod_{m=1}^{\infty}(1-q^{m})^{-1} \frac{1}{-\Theta (p)}\\
&=\prod_{m=1}^{\infty} \frac{(1-q^{m})}{(1-pq^{m})(1-p^{-1}q^{m})}
\end{align*}
which proves equation~\eqref{eqn 3}.

Again we apply Lemma~\ref{lem: eqns for
Vlambdaboxempty/Vlambdaemptyempty and Vlambdaboxbox/Vlambdaemptyempty}
and Theorem~\ref{thm: Bloch-Okounkov thm}:
\begin{align*}
\sum_{\lambda} q^{|\lambda |}p \frac{\Vsf_{\lambda \bx
\bx}}{\Vsf_{\lambda \emptyset \emptyset}} &= \sum_{\lambda}
q^{|\lambda
|}\left\{1-\left(\sum_{i=1}^{\infty}p^{-\lambda_{i}+i-\half}
\right)\left(\sum_{j=1}^{\infty}p^{\lambda_{j}-j+\half} \right)  \right\}\\
&= \prod_{m=1}^{\infty}(1-q^{m})^{-1} \left(1-F(p,p^{-1}) \right).
\end{align*}

From Theorem~\ref{thm: Bloch-Okounkov thm}, we see that 
\[
F(p,p^{-1}) = \lim_{ (p_{1},p_{2})\to (p,p^{-1}) } \frac{1}{\Theta
(p_{1}p_{2})} \left(p_{1}\frac{d}{dp_{1}}\log(\Theta (p_{1}))+ p_{2}\frac{d}{dp_{2}}\log(\Theta (p_{2})) \right).
\]
To evaluate this limit, we simplify the above expression. A short
computation shows that 
\[
p\frac{d}{dp}\log(\Theta (p)) = \half \frac{p+1}{p-1}
+\sum_{m=1}^{\infty}\sum_{k=1}^{\infty} \left(-p^{k}+p^{-k} \right)
q^{mk}.
\]
Thus
\begin{align*}
F(p,p^{-1}) =& \lim_{\begin{smallmatrix} (p_{1},p_{2})\to \\
(p,p^{-1}) \end{smallmatrix}}
\left((p_{1}p_{2})^{\half} - (p_{1}p_{2})^{-\half} \right)^{-1}
\prod_{m=1}^{\infty}
\frac{(1-q^{m})^{2}}{(1-(p_{1}p_{2})q^{m})(1-(p_{1}p_{2})^{-1}q^{m})} \\
&\quad \quad \quad \quad \cdot \left\{\half \cdot \frac{p_{1}+1}{p_{1}-1}+\half \cdot \frac{p_{2}+1}{p_{2}-1}
+\sum_{m=1}^{\infty}\sum_{k=1}^{\infty}
\left(-p_{1}^{k}-p_{2}^{k}+p_{1}^{-k}+p_{2}^{-k} \right)q^{mk}
\right\} \\
=&  \lim_{\begin{smallmatrix} (p_{1},p_{2})\to \\
(p,p^{-1}) \end{smallmatrix}}
\frac{-(p_{1}p_{2})^{\half}}{1-p_{1}p_{2}} \cdot
\left\{\frac{p_{1}p_{2}-1}{(p_{1}-1)(p_{2}-1)} +
\sum_{m=1}^{\infty}\sum_{k=1}^{\infty} (1-p_{1}^{k}p_{2}^{k})(p_{1}^{-k}+p_{2}^{-k})q^{mk} \right\}\\
= &  \lim_{\begin{smallmatrix} (p_{1},p_{2})\to \\
(p,p^{-1}) \end{smallmatrix}}
\,(p_{1}p_{2})^{\half} \left\{\frac{1}{(1-p_{1})(1-p_{2})} -
\sum_{m=1}^{\infty }\sum_{k=1}^{\infty } \frac{1-(p_{1}p_{2})^{k}}{1-p_{1}p_{2}}(p_{1}^{-k}+p_{2}^{-k})q^{mk} \right\}\\
=& \frac{1}{(1-p)(1-p^{-1})} - \sum_{m=1}^{\infty}\sum_{k=1}^{\infty}
k(p^{k}+p^{-k})q^{mk}. 
\end{align*}
Therefore 
\[
1-F(p,p^{-1}) = 1 +\frac{p}{(1-p)^{2}} + \sum_{d=1}^{\infty}\sum_{k|d}k(p^{k}+p^{-k})q^{d}
\]
which finishes the proof of equation~\eqref{eqn 4}.

\section{Vertex operators and the proof of equation~\eqref{eqn
5}}\label{sec: vertex ops and the pf of eqn 5}


There are several sources for vertex operators and the infinite wedge
formalism. For consistency, we will follow the notation and conventions
of \cite[Appendix~A]{Okounkov-InfWedge}.

Let $V$ be the vector space with basis $\left\{\underline{k}
\right\},$ $k\in \ZplusHalf$. We define \emph{Fock space} $\FockSpace
$ to be the vector space spanned by vectors
\[
v_{S} = \underline{s_{1}}\wedge \underline{s_{2}}\wedge \dots 
\]
where $S=\left\{s_{1}>s_{2}>\dots \right\}\subset \ZplusHalf$ is
any subset such that the sets
\[
S_{+} = S\cap \left(\ZplusHalf \right)_{>0}\quad \text{and} \quad S_{-} =
S^{c}\cap \left(\ZplusHalf \right)_{<0}
\]
are both finite. Let $(\cdot ,\cdot )$ be the inner product on
$\FockSpace$ such that the basis $\left\{v_{S} \right\}$ is
orthonormal. 

For any $k\in \ZplusHalf $ let $\psi_{k}$ be the operator
\[
\psi_{k}(f) = \underline{k}\wedge f
\]
and let $\psi^{*}_{k}$ be its adjoint.

For any partition $\lambda =\{\lambda_{1}\geq \lambda_{2}\geq \dots
\}$, we define the vector
\[
v_{\lambda}  = \underline{(\lambda_{1}-\half )}\wedge
\underline{(\lambda_{2}-\frac{3}{2})}\wedge \dots  
\]

Let $\FockSpaceZero \subset \FockSpace$ be the subspace spanned by the
vectors $\{v_{\lambda} \}$ where $\lambda$ runs over all
partitions. We call this \emph{charge zero Fock space.}

The \emph{energy operator}
\[
H=\sum_{k>0} k\left(\psi_{k}\psi_{k}^{*}+\psi_{-k}^{*}\psi_{-k} \right)
\]
acts on the basis $v_{\lambda}$ by
\[
Hv_{\lambda}  = |\lambda |v_{\lambda }
\]
and so the operator $q^{H}$ acts by
\[
q^{H}v_{\lambda} = q^{|\lambda |}v_{\lambda }
\]
where $q$ is a formal parameter.

For $n\in \znums$, $n\neq 0$ define
\[
\alpha_{n}=\sum_{k} \psi_{k-n}\psi^{*}_{k}
\]
and observe that $\alpha^{*}_{n}=\alpha_{-n}$.

Following \cite{Okounkov-InfWedge}, we define the \emph{vertex
operators} $\Gamma_{\pm}(\mathbf{x})$ which are operators on
$\FockSpaceZero$ over the coefficient ring given by symmetric
functions in an infinite set of variables $\mathbf{x} =
(x_{1},x_{2},x_{3},\dots )$. Let $\mathbf{s}=(s_{1},s_{2},\dots )$
\[
s_{k} =\frac{1}{k}\sum_{i=1}^{\infty} x_{i}^{k}
\]
be the power sum basis for the ring of symmetric functions and
let
\footnote{In \cite{Okounkov-InfWedge}, the argument of
$\Gamma_{\pm}$ is $\mathbf{s}$, and the dependence on the underlying set of
variables $\mathbf{x}$ is left implicit. We prefer to make
$\mathbf{x}$ the explicit argument.}
\[
\Gamma_{\pm}(\mathbf{x}) =
\exp\left(\sum_{n=1}^{\infty}s_{n}\alpha_{\pm n } \right).
\]
Observe that $\Gamma^{*}_{\pm}=\Gamma_{\mp}$.

The matrix coefficients of the vertex operators in the $\{v_{\lambda}
\}$ basis are given by skew Schur functions:
\begin{equation}\label{eqn: matrix coefs of vertex ops are skew schur}
(\Gamma_{-}(\mathbf{x})v_{\mu},v_{\lambda}) =
(v_{\mu},\Gamma_{+}(\mathbf{x})v_{\lambda}) = s_{\lambda
/\mu}(\mathbf{x}).
\end{equation}

Orthogonality of the skew Schur functions then gives rise to the
following commutation equation:
\[
\Gamma_{+}(\mathbf{x})\Gamma_{-}(\mathbf{y}) = \prod_{i,j}
(1-x_{j}y_{j})^{-1} \Gamma_{-}(\mathbf{y})\Gamma_{+}(\mathbf{x}),
\]
in particular
\begin{equation}\label{eqn: Gamma+Gamma- commutation relation}
\Gamma_{+}(u\ptotheminusrho )\Gamma_{-}(v\ptotheminusrho ) =M(p,uv)
\Gamma_{-}(v\ptotheminusrho )\Gamma_{+}(u\ptotheminusrho )
\end{equation}
where recall that $u\ptotheminusrho  = (up^{\frac{1}{2}},up^{\frac{3}{2}},up^{\frac{5}{2}},\dots )$.

We let
\[
\psi (z) = \sum_{i} z^{i}\psi_{i}\quad \text{and}\quad \psi^{*}(w) =
\sum_{j} w^{-j} \psi_{j}^{*}. 
\]

The commutation relations of these operators with the vertex operators
is given by
\begin{align}\label{eqn: commutation of Gamma(x) with psi(z) and psi*(w)}
\Gamma_{\pm}(\mathbf{x})\psi (z) &= \prod_{i=1}^{\infty}
(1-x_{i}z^{\pm 1})^{-1} \,\, \psi (z)\Gamma_{\pm}(\mathbf{x})\\
\Gamma_{\pm}(\mathbf{x})\psi^{*} (w) &= \prod_{i=1}^{\infty}
(1-x_{i}w^{\pm 1}) \,\,\psi^{*}(w)\Gamma_{\pm}(\mathbf{x}).\nonumber
\end{align}


We use operators $\E_{r}$ introduced by Okounkov-Pandharipande in
\cite[\S~2.2.4]{Okounkov-Pandharipande-completed-cycles}.  For $r\in
\znums$, let\footnote{We avoid the use of normal ordering by allowing
coefficients which are Laurent series in $p^{\half}$.}
\[
\E_{r}(p) = \sum_{k\in \ZplusHalf} p^{-k+\frac{r}{2}}\,\, \psi_{k-r}\psi^{*}_{k}.
\]
Our variable $p$ is related to the variable $z$ in
\cite{Okounkov-Pandharipande-completed-cycles} by $p=e^{-z}.$


From \cite[Eqns~2.9 and 0.18]{Okounkov-Pandharipande-completed-cycles}, we
see that $\E_{0}$ is the diagonal operator given by
\begin{equation}\label{eqn: formula for the operator E0}
\E _0 (p) v_{\lambda} = \left(\sum_{i=1}^{\infty}
p^{-\lambda_{i}+i-\half} \right)v_{\lambda}. 
\end{equation}

We define
\[
\E (a,p) = \sum_{r\in \znums} a^{-r} \E_{r}(p)
\]
where $a$ is a formal parameter. A short computation shows that 
\[
\E (a,p) = \psi (ap^{\half})\psi^{*} (ap^{-\half}).
\]
From equation~\eqref{eqn: commutation of Gamma(x) with psi(z) and
psi*(w)} we get
\[
\Gamma_{\pm}(\mathbf{x})\E (a,p) =\prod_{i=1}^{\infty}\frac{(1-a^{\mp
1}p^{\mp \half}x_{i})}{(1-a^{\mp
1}p^{\pm \half}x_{i})}\,\, \E (a,p)\Gamma_{\pm}(\mathbf{x}). 
\]

For
$\mathbf{x}=u\ptotheminusrho =(up^{\frac{1}{2}},up^{\frac{3}{2}},up^{\frac{5}{2}},\dots )
$ the above simplifies to 
\begin{align}\label{eqn: commutation of Gamma+- with E(a,p)}
\Gamma_{+}(u\ptotheminusrho ) \E (a,p) &= (1-au) \E (a,p) \Gamma_{+}(u\ptotheminusrho )\\
\E (a,p) \Gamma_{-}(u\ptotheminusrho ) &= (1-a^{-1}u) \Gamma_{-}(u\ptotheminusrho )
\E (a,p). \nonumber
\end{align}
Finally, it follows from equation~\eqref{eqn: matrix coefs of vertex
ops are skew schur} that
\begin{equation}\label{eqn: commutation of Gamma and q^{H}}
\Gamma_{\pm}(\mathbf{x})q^{H} = q^{H} \Gamma_{\pm}(q^{\pm 1}\mathbf{x}). 
\end{equation}


We now write the left hand side of equation~\eqref{eqn 5} in the main
theorem as a trace of operators on charge zero Fock space.
\begin{lemma}\label{lem: eqn 5 written as a trace}
\begin{align*}
\sum_{\lambda} q^{|\lambda |} p^{\| \lambda' \| ^{2}} \frac{\Vsf_{\lambda
\bx \emptyset}}{\Vsf_{\lambda \emptyset \emptyset}} \Vsf_{\lambda
'\lambda \emptyset}& = p^{-\half} \operatorname{Tr} \left(\E
_0(p)\Gamma_{+}(\ptotheminusrho )\Gamma_{-}(\ptotheminusrho )q^{H} \right)\\
& = p^{-\half} \operatorname{Coeff}_{a^{0}}\left\{\operatorname{Tr} \left(\E(a,p)\Gamma_{+}(\ptotheminusrho )\Gamma_{-}(\ptotheminusrho )q^{H} \right) \right\}.
\end{align*}
\end{lemma}
\proof The following computation uses, in order, the commutation
relation for $\Gamma_{+}$ and $\Gamma_{-}$ (equation~\eqref{eqn:
Gamma+Gamma- commutation relation}), the definition of trace, the
formula for $\E_{0} (p)$ (equation~\eqref{eqn: formula for the
operator E0}), Lemma~\ref{lem: eqns for
Vlambdaboxempty/Vlambdaemptyempty and
Vlambdaboxbox/Vlambdaemptyempty}, equation~\eqref{eqn: matrix coefs of
vertex ops are skew schur}, and finally Lemma~\ref{lem: eqns for
Vlambdaboxempty/Vlambdaemptyempty and Vlambdaboxbox/Vlambdaemptyempty}
again:
\begin{align*}
\quad &\,\, \quad p^{-\half} \tr \left(\E_{0}(p)\Gamma_{+}(\ptotheminusrho )\Gamma_{-}(\ptotheminusrho )q^{H} \right)\\
& = p^{-\half} M(p) \tr \left(\E_{0}(p)\Gamma_{-}(\ptotheminusrho )\Gamma_{+}(\ptotheminusrho )q^{H} \right)\\
&=p^{-\half}M(p)\sum_{\lambda} \left(v_{\lambda}, \E_{0}(p)\Gamma_{-}(\ptotheminusrho )\Gamma_{+}(\ptotheminusrho )q^{H} v_{\lambda}\right)\\
&=p^{-\half } \sum_{\lambda} q^{|\lambda |} \left(\sum_{i=1}^{\infty}
p^{-\lambda_{i}+i-\half} \right) \left(v_{\lambda}, \Gamma_{-}(\ptotheminusrho )\Gamma_{+}(\ptotheminusrho )v_{\lambda} \right)\\
&=M(p)\sum_{\lambda}q^{|\lambda |} \frac{\Vsf_{\lambda \bx
\emptyset}}{\Vsf_{\lambda \emptyset \emptyset}} \left(\Gamma_{+}(\ptotheminusrho )V_{\lambda },\Gamma_{+}(\ptotheminusrho )V_{\lambda } \right)\\
&= \sum_{\lambda} q^{|\lambda |} \frac{\Vsf_{\lambda \bx
\emptyset}}{\Vsf_{\lambda \emptyset \emptyset}} M(p)\sum_{\eta} \left(s_{\lambda /\eta}(\ptotheminusrho ) \right)^{2}\\
&= \sum_{\lambda} q^{|\lambda |} \frac{\Vsf_{\lambda \bx
\emptyset}}{\Vsf_{\lambda \emptyset \emptyset}} p^{\| \lambda' \| ^{2}}
V_{\lambda '\lambda \emptyset} .
\end{align*}
\qed 

While the operator $\E_{0}(p)$ does not have good commutation
relations with the vertex operators, the operator $\E (a,p)$
does. Hence we first replace $\E_{0}(p)$ with the more general $\E
(a,p)$, compute the trace, and then specialize to the $a^{0}$
coefficient.
\begin{lemma}\label{lem: trace of E Gamma+Gamma-q^{H}}
\begin{multline*}
\tr \left(\E (a,p)\Gamma_{+}(\ptotheminusrho )\Gamma_{-}(\ptotheminusrho )q^{H}
\right) = \\
\frac{1}{p^{-\half}-p^{\half}}M(p)\prod_{m=1}^{\infty}
\frac{(1-q^{m}a)(1-q^{m-1}a^{-1})(1-q^{m})
M(p,q^{m})}{(1-pq^{m})(1-p^{-1}q^{m})}.
\end{multline*}
\end{lemma}
\proof Our strategy is the following. We use the cyclic invariance of
trace along with the commutation relations for $\Gamma_{+}$ to move
the operator $\Gamma_{+}$ past the other operators cyclically to the
right until the operators are back to their original positions, but
with new arguments. We perform this operation a countable number of
times, eventually making the $\Gamma_{+}$ operator
disappear\footnote{The third author thanks Guillaume Chapuy and Sylvie
Corteel for teaching him this trick at a conference lunch in
2014. Bouttier, Chapuy, and Corteel used the trick in the paper
\cite{Bouttier-Chapuy-Corteel} in the proof of theorem 12
therein.}. We then employ the same strategy moving $\Gamma_{-}$
cyclically to the left a countable number of times until it disappears
and we are left with a term which we can evaluate with the
Okounkov-Bloch theorem (theorem~\ref{thm: Bloch-Okounkov thm}).

We first cyclically commute the operator $\Gamma_{+}$ to the right
using equations~\eqref{eqn: Gamma+Gamma- commutation relation},
\eqref{eqn: commutation of Gamma+- with E(a,p)}, and \eqref{eqn:
commutation of Gamma and q^{H}}:

\begin{align*}
\quad &\quad \quad \quad \,\,\, \tr (\E (a,p)\Gamma_{+}(\ptotheminusrho)\Gamma_{-}(\ptotheminusrho
)q^{H})\\
& = M(p)\tr (\E (a,p)\Gamma_{-}(\ptotheminusrho)\Gamma_{+}(\ptotheminusrho
)q^{H})\\
& = M(p)\tr (\E (a,p)\Gamma_{-}(\ptotheminusrho)q^{H}\Gamma_{+}(q\ptotheminusrho
))\\
& = M(p)\tr (\Gamma_{+}(q\ptotheminusrho
)\E (a,p)\Gamma_{-}(\ptotheminusrho)q^{H})\\
& = M(p)(1-qa)\tr (\E (a,p)\Gamma_{+}(q\ptotheminusrho) \Gamma_{-}(\ptotheminusrho)q^{H}).
\end{align*}

Cyclically commuting $\Gamma_{+}$ to the right a second time we get:
\begin{multline*}
 \tr (\E (a,p)\Gamma_{+}(\ptotheminusrho)\Gamma_{-}(\ptotheminusrho
)q^{H}) = \\
M(p)(1-qa)M(p,q)(1-q^{2}a) \tr (\E (a,p)\Gamma_{+}(q^{2}\ptotheminusrho)\Gamma_{-}(\ptotheminusrho
)q^{H}).
\end{multline*}

After performing $N$ iterations of this strategy, we arrive at
\begin{multline*}
 \tr (\E (a,p)\Gamma_{+}(\ptotheminusrho)\Gamma_{-}(\ptotheminusrho
)q^{H}) = \\
\prod_{d=1}^{N}M(p,q^{d-1})(1-q^{d}a) \tr (\E (a,p)\Gamma_{+}(q^{N}\ptotheminusrho)\Gamma_{-}(\ptotheminusrho
)q^{H}).
\end{multline*}
It follows from Equation~\eqref{eqn: matrix coefs of vertex ops are
skew schur} that 
\[
\Gamma_{\pm}(q^{N}\ptotheminusrho )\equiv \operatorname{Id} \mod q^{N}.
\]
So the above two equations imply that the equation
\begin{multline*}
 \tr (\E (a,p)\Gamma_{+}(\ptotheminusrho)\Gamma_{-}(\ptotheminusrho
)q^{H}) = \\
\prod_{d=1}^{\infty }M(p,q^{d-1})(1-q^{d}a) \tr (\E (a,p)\Gamma_{-}(\ptotheminusrho
)q^{H})
\end{multline*}
holds to all orders in $q$ and is hence true as a formal power series
in $q$.

We now apply the same strategy commuting $\Gamma_{-}$ to the left:
\begin{align*}
\tr (\E (a,p)\Gamma_{-}(\ptotheminusrho )q^{H}) &=(1-a^{-1})\tr (\E (a,p)\Gamma_{-}(q\ptotheminusrho )q^{H}) \\
&=(1-a^{-1})(1-a^{-1}q)\tr (\E (a,p)\Gamma_{-}(q^{2}\ptotheminusrho )q^{H}) \\
&=\dots \\
&=\prod_{d=1}^{\infty}(1-a^{-1}q^{d-1})\tr (\E (a,p)q^{H})
\end{align*}
and so we have proved
\begin{multline*}
 \tr (\E (a,p)\Gamma_{+}(\ptotheminusrho)\Gamma_{-}(\ptotheminusrho
)q^{H}) = \\
\prod_{d=1}^{\infty }M(p,q^{d-1})(1-q^{d}a)(1-a^{-1}q^{d-1})\tr (\E (a,p)q^{H}).
\end{multline*}


From the definition of $\E_{r}(p)$ we see that its matrix entries are
all off-diagonal if $r\neq 0$. Therefore
\begin{align*}
\tr (\E (a,p) q^{H})&=\tr (\E_{0}(p)q^{H})\\
&= \sum_{\lambda} q^{|\lambda |}\sum_{i=1}^{\infty}p^{-\lambda +i-\half}\\
&=(p^{-\half}-p^{\half})^{-1}\prod_{m=1}^{\infty} \frac{(1-q^{m})}{(1-pq^{m})(1-p^{-1}q^{m})}
\end{align*}
where the last equality follows from the computation in the proof of
equation~\eqref{eqn 3} in \S~\ref{subsec: pfs of eqn 3 and
4}. Combining this with the previous computations finishes the proof
of the lemma. \qed


Combining Lemmas~\ref{lem: eqn 5 written as a trace} and \ref{lem:
trace of E Gamma+Gamma-q^{H}}, we get
\begin{multline*}
\sum_{\lambda} q^{|\lambda |} p^{\| \lambda' \|  ^{2}} \frac{\Vsf_{\lambda
\bx \emptyset}}{\Vsf_{\lambda \emptyset \emptyset}} \Vsf_{\lambda
'\lambda \emptyset} =\frac{1}{1-p} M(p)\prod_{m=1}^{\infty}
\frac{M(p,q^{m})}{(1-pq^{m})(1-p^{-1}q^{m})}\\
 \cdot \operatorname{Coeff}_{a^{0}}\left\{\prod_{m=1}^{\infty}
(1-q^{m}a)(1-q^{m}a^{-1})(1-q^{m}) \right\}.
\end{multline*}

By the Jacobi triple product identity, we have
\[
\prod_{m=1}^{\infty} (1-q^{m}a)(1-q^{m}a^{-1})(1-q^{m}) =
\sum_{n=-\infty}^{\infty} q^{\binom{n}{2}} (-a)^{n}
\]
whose $a^{0}$ coefficient is 1. Plugging into the previous equation we
finish the proof of equation~\eqref{eqn 5}.

\section{Geometry and Applications}\label{sec: geometry and applications}

\smallskip

$\Vsf_{\lambda \mu \nu}(p)$ as generating function for Euler
characteristics of Hilbert schemes on $\cnums^{3}$.

\smallskip

short description of local contributions to DT invariants in
threefolds with elliptic fibrations. \begin{enumerate}
\item multiple of smooth fiber
\item multiple of nodal fiber
\item multiple of smooth fiber attached to section
\item multiple of nodal fiber attached to section
\item multiple of smooth fiber attached to node
\end{enumerate}

\smallskip
connection of formulas with Jacobi forms. 



\bibliography{/Users/jbryan/jbryan/resources/mainbiblio}
\bibliographystyle{plain}

\end{document}

