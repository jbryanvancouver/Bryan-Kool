\documentclass{amsart}

\title{Note to referee}
\author{Jim Bryan}



%\usepackage{diagrams}
%\usepackage{eepic,epic}

\usepackage{amsmath}
\usepackage{amsmath,amsthm,amsfonts}
\usepackage{times}
%\usepackage{amstex}

\newcommand{\cnums} {{\mathbb C}}          % complex numbers
\newcommand{\nnums} {{\mathbb N}}		% natural numbers
\newcommand{\rnums} {{\mathbb R}}		% real numbers
\newcommand{\znums} {{\mathbb Z}}		% integers
\newcommand{\qnums} {{\mathbb Q}}		% rationals

\newcommand{\Hom}{\operatorname{Hom}}
\newcommand{\Ker}{\operatorname{Ker}}
\newcommand{\End}{\operatorname{End}}
\newcommand{\Tr}{\operatorname{tr}}
\newcommand{\tr}{\operatorname{tr}}
\newcommand{\Coker}{\operatorname{Coker}}
\newcommand{\im}{\operatorname{Im}}

\renewcommand{\P}{\mathbb{P}}
\newcommand{\M}{\overline{\mathcal{M}}}
\newcommand{\smargin}[1]{\marginpar{\tiny{#1}}}


\begin{document}


\maketitle 

Dear referee.

Thank you for your question. All the series in this paper, with the
exception of $M(p)=M(p,1)$ are in fact in $\znums (p)[[q]]$, that is,
the coefficients of $q^{N}$ are the Laurent expansions of rational
functions in $p$. This follows from the fact that 
\[
s_{\lambda /\eta}(p^{-\nu -\rho })
\]
is the Laurent expansion of a rational function in $p^{1/2}$ for any
partitions $\lambda , \eta , \nu$.

We seem to only be using this fact explicitly for the case of
\[
s_{\square}(p^{-\lambda -\rho }) = \sum_{i=1}^{\infty} p^{-\lambda_{i}+i
- 1/2}
\]
where the assertion is self-evident since the tail of the series is a
geometric progression.

It is definitely worth making this clarification and I will revise the
paper to make this point more clearly. In fact, after dividing both
sides of equations (1) and (4) by $M(p)$, the main theorem could be
stated as an equality of formal series in $q$ whose coefficients are
rational functions in $p$.


   
\end{document}


