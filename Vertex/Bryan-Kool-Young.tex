\documentclass{amsart}

\title{A topological vertex identity and the Katz-Klemm-Vafa formula}
\author{Jim Bryan, Martijn Kool and Benjamin Young}
\date{\today}
\address{
Department of Mathematics\\
University of British Columbia \\
Room 121, 1984 Mathematics Road  \\
Vancouver, B.C., Canada V6T 1Z2  
}


%\usepackage{diagrams}
%\usepackage{eepic,epic}

\usepackage{amsmath}
\usepackage{amsmath,amsthm,amsfonts}
\usepackage{amssymb}
\usepackage{times}
\usepackage[all]{xy}
%\usepackage{amstex}


%\newtheorem{thm}{Theorem}%[section]
\newtheorem{theorem}{Theorem}%[section]
\newtheorem{proposition}[theorem]{Proposition}
\newtheorem{conjecture}[theorem]{Conjecture}
\newtheorem{lemma}[theorem]{Lemma}
\newtheorem{corollary}[theorem]{Corollary}
\theoremstyle{definition}

\newtheorem{def-theorem}[theorem]{Definition-Theorem}
\newtheorem{remark}[theorem]{Remark}
\newtheorem{definition}[theorem]{Definition}
\newtheorem{example}[theorem]{Example}


\newcommand{\CC} {\mathbb{C}}          % complex numbers
\newcommand{\NN} {\mathbb{N}}		% natural numbers
\newcommand{\RR} {\mathbb{R}}		% real numbers
\newcommand{\ZZ} {\mathbb{Z}}		% integers
\newcommand{\QQ} {\mathbb{Q}}		% rationals
\newcommand{\PP} {\mathbb{P}}
\renewcommand{\AA} {\mathbb{A}}
\newcommand{\LL} {\mathbb{L}}
\newcommand{\FF} {\mathbb{F}}
\renewcommand{\O}{\mathcal{O}}


\newcommand{\rt}[1]{\stackrel{#1\,}{\rightarrow}}
\newcommand{\Rt}[1]{\stackrel{#1\,}{\longrightarrow}}
\newcommand\To{\longrightarrow}
\newcommand\into{\hookrightarrow}
\newcommand\Into{\ensuremath{\lhook\joinrel\relbar\joinrel\rightarrow}}
\newcommand\INTO{\ar@{^{(}->}[r]}
\newcommand\acts{\curvearrowright}


\newcommand{\Hom}{\operatorname{Hom}}
\newcommand{\Ker}{\operatorname{Ker}}
\newcommand{\End}{\operatorname{End}}
\newcommand{\GL}{\operatorname{GL}}
\newcommand{\Tr}{\operatorname{tr}}
\newcommand{\tr}{\operatorname{tr}}
\newcommand{\Coker}{\operatorname{Coker}}
\newcommand{\im}{\operatorname{Im}}
\newcommand{\M}{\overline{\mathcal{M}}}
\newcommand{\smargin}[1]{\marginpar{\tiny{#1}}}
\newcommand{\Sym}{\operatorname{Sym}}
\newcommand{\Coh}{\operatorname{Coh}}


\begin{document}

\begin{abstract}
Motivated by a new calculation of the Katz-Klemm-Vafa formula (primitive case), the first two authors conjectured a product formula for a certain generating function involving the topological vertex. In this paper we prove this formula using the infinite wedge formalism. The method is by writing the generating function as a trace and then combining standard commutation relations of vertex operators with cyclicity of trace. This trick was picked up by the third author from a paper of J.~Bouttier, G.~Chapuy and S.~Corteel. The techniques of this paper are useful for calculating generating functions of Donaldson-Thomas or stable pair invariants in numerous geometric settings.
\end{abstract}

\maketitle 

%\markboth{???}  {???}
%\renewcommand{\sectionmark}[1]{}


%\tableofcontents
%\pagebreak


\section{Introduction}

\noindent \textbf{Acknowledgements.} Paul Johnson, PIMS, ...


\section{Review of the infinite wedge formalism}

We start with a brief review of the infinite wedge formalism, various operators and their commutation relations of vertex operators appearing in the work of A.~Okounkov, R.~Pandharipande \cite{OP} and Okounov, N.~Reshetikhin \cite{OR2}. See also \cite{Kac} and \cite{You}. This section is transcribed from these references. We include it in order to establish our sign conventions and for the readers convenience.

Let $V$ be the complex vector space spanned by $\underline{k}$, where $k \in \ZZ + \frac{1}{2}$. By definition, the infinite wedge space $\Lambda^{\frac{\infty}{2}} V$ is the complex vector space spanned by vectors
$$
v_S := \underline{s_1} \wedge \underline{s_2} \wedge \cdots
$$
where $S = \{s_1 > s_2 > \cdots\} \subset \ZZ + \frac{1}{2}$ for which both 
$$
S_+ = S \setminus \Big(\ZZ_{\leq 0} - \frac{1}{2} \Big), \ S_- = \Big(\ZZ_{\leq 0} - \frac{1}{2} \Big) \setminus S
$$ 
are finite. The subspace spanned $v_S$ for which $|S_+| = |S_-|$ is known as the zero charge space and denoted by $\Lambda^{\frac{\infty}{2}}_{0} V$. The collection of subset $S = \{s_1 > s_2 > \cdots\} \subset \ZZ + \frac{1}{2}$ for which $|S_+| = |S_-|$ is in natural bijection with the collection of plane partitions $\lambda = \{\lambda_1 \geq \lambda_2 \geq \cdots\} \subset \ZZ_{\geq 0}$ via the mapping \cite[2.1.3]{OR}
$$
\lambda \mapsto \mathfrak{S}(\lambda) = \Big\{ \lambda_i - i + \frac{1}{2} \Big\}_i \subset \ZZ + \frac{1}{2}.
$$
These are known as modified Frobenius coordinates. We denote partitions by $\lambda, \mu, \nu, \eta,  \ldots$ and define 
$$|\lambda\rangle := v_{\mathfrak{S}(\lambda)}.$$
In particular, the vacuum vector is given by
$$
|\varnothing \rangle := \underline{-\frac{1}{2}} \wedge \underline{-\frac{3}{2}} \wedge \cdots.
$$
Denote by $\langle \cdot | \cdot \rangle$ the complex inner product for which
$$
\langle \lambda | \mu \rangle = \delta_{\lambda\mu},
$$
where $\delta_{\lambda\mu}$ is the Kronecker delta. 

For each $k \in \ZZ + \frac{1}{2}$ one defines the operator (on $\Lambda^{\frac{\infty}{2}} V$)
$$
\psi_k := \underline{k} \wedge \cdot
$$
and its adjoint is denoted by $\psi_{k}^{*}$. These operators satisfy the anti-commutation relations
\begin{align*}
\psi_k \psi_l + \psi_l \psi_k = \psi_{k}^{*} \psi_{l}^{*} + \psi_{l}^{*} \psi_{k}^{*} = 0, \\ 
\psi_k \psi_{l}^{*} + \psi_{l}^{*} \psi_{k} = \delta_{kl}.  
\end{align*}
These operators can be combined to
$$
\psi(a) := \sum_{k \in \ZZ + \frac{1}{2}} \psi_k a^k, \ \psi^*(a) := \sum_{k \in \ZZ + \frac{1}{2}} \psi_{k}^{*} a^{-k}. 
$$
Next consider the operators 
$$
\alpha_n := \sum_{k \in \ZZ + \frac{1}{2}} \psi_{k+n} \psi_{k}^{*}, \ n \in \ZZ.
$$
These satisfy the Heisenberg commutation relations $[\alpha_n,\alpha_m] = -n \delta_{n,-m}$ and
$$
[\alpha_n, \psi(a)] = a^{-n} \psi(a), \ [\alpha_n, \psi^*(a)] =  - a^{-n} \psi^*(a). 
$$

We are interested in the vertex operators
$$
\Gamma_{\pm}(q) := \exp \Big( \sum_{n \geq 1} \frac{q^n}{n!} \alpha_\pm \Big).
$$
These acts on $\Lambda^{\frac{\infty}{2}}_{0} V$ as follows
\begin{align*}
\Gamma_-(q) |\mu \rangle &= \sum_{\lambda \succ \mu} q^{|\lambda| - |\mu|} |\lambda \rangle, \\ 
\Gamma_+(q) | \lambda \rangle &= \sum_{\lambda \succ \mu} q^{|\lambda| - |\mu|} |\mu \rangle.
\end{align*}
Here $|\lambda|:=\sum_i \lambda_i$ is the size of the partition and $\lambda \succ \mu$ means $\lambda$ interlaces $\mu$, i.e.
$$
\lambda_1 \geq \mu_1 \geq \lambda_2 \geq \mu_2 \geq \cdots.
$$
Equivalently, the skew diagram $\lambda \setminus \mu$ is a disjoint union of horizontal strips (see \cite{You} for details). Therefore we think of $\Gamma_-(q) |\mu\rangle$ as adding horizontal strips from $\mu$ and $\Gamma_+(q) |\lambda\rangle$ as removing horizontal strips from $\lambda$. 

We will use the following commutation relations from Okounkov and Reshetikhin \cite{OR2}
\begin{align*}
\Gamma_+(a) \psi(b) &= (1-a b^{-1})^{-1} \psi(b) \Gamma_+(a), \\
\Gamma_-(a) \psi(b) &= (1-a b)^{-1} \psi(b) \Gamma_-(a), \\
\Gamma_+(a) \psi^*(b) &= (1-a b^{-1}) \psi^*(b) \Gamma_+(a), \\
\Gamma_-(a) \psi^*(b) &= (1-a b) \psi^*(b) \Gamma_-(a)
\end{align*}
\begin{align*}
\Gamma_+(a)\Gamma_-(b) = (1-ab) \Gamma_-(b)\Gamma_+(a)
\end{align*}


\section{Operators and commutation relations}

Define our bold face operators with their commutation relations derived from previous section. Reference to Bloch-Okounkov. Possibly derivation MacMahon and 2 leg DT=PT as warm-up examples?

\section{Calculation}

The disconnected series (known identity) first using the trace trick. Our new formula second.


     
\bibliography{mainbiblio}
\bibliographystyle{plain}

\end{document}

