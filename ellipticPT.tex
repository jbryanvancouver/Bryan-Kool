\documentclass[12pt]{amsart}
%\documentclass[12pt]{article}

\usepackage[english]{babel}

%\usepackage[pdftex,paper=a4paper,portrait=true,textwidth=450pt,textheight=675pt,tmargin=3cm,marginratio=1:1]{geometry}
\usepackage[pdftex,textwidth=400pt,marginratio=1:1]{geometry}

\usepackage{amsfonts}

\usepackage[dvips]{graphics}

\usepackage[colorinlistoftodos]{todonotes}

\usepackage{amsmath}

\usepackage{amsthm}

\usepackage{amssymb}

\usepackage{bbm}

\usepackage{cancel}

\usepackage{color}

\usepackage[curve]{xypic}

%\usepackage{stmaryrd}

%\input xy

%\xyoption{all}

\newtheorem{theorem}{Theorem}[section]

\newtheorem{corollary}[theorem]{Corollary}

\newtheorem{proposition}[theorem]{Proposition}

\newtheorem{lemma}[theorem]{Lemma}

\newtheorem{conjecture}[theorem]{Conjecture}

\newtheorem{assumption}[theorem]{Assumption}

\theoremstyle{definition}

\newtheorem{definition}[theorem]{Definition}

\newtheorem{example}[theorem]{Example}

\newtheorem{remark}[theorem]{Remark}

\newtheorem*{acknowledgements}{Acknowledgements}

\theoremstyle{property}

\newtheorem{property}[theorem]{Property}

\makeatletter
\newcommand{\contraction}[5][1ex]{%
  \mathchoice
    {\contraction@\displaystyle{#2}{#3}{#4}{#5}{#1}}%
    {\contraction@\textstyle{#2}{#3}{#4}{#5}{#1}}%
    {\contraction@\scriptstyle{#2}{#3}{#4}{#5}{#1}}%
    {\contraction@\scriptscriptstyle{#2}{#3}{#4}{#5}{#1}}}%
\newcommand{\contraction@}[6]{%
  \setbox0=\hbox{$#1#2$}%
  \setbox2=\hbox{$#1#3$}%
  \setbox4=\hbox{$#1#4$}%
  \setbox6=\hbox{$#1#5$}%
  \dimen0=\wd2%
  \advance\dimen0 by \wd6%
  \divide\dimen0 by 2%
  \advance\dimen0 by \wd4%
  \vbox{%
    \hbox to 0pt{%
      \kern \wd0%
      \kern 0.5\wd2%
      \contraction@@{\dimen0}{#6}%
      \hss}%
    \vskip 0.5ex%  how far above the line starts
    \vskip\ht2}}
\newcommand{\contracted}[5][1ex]{%
  \contraction[#1]{#2}{#3}{#4}{#5}\ensuremath{#2#3#4#5}}
\newcommand{\contraction@@}[3][0.05em]{%
% the 1st parameter (explicitely inserted) is the width % of the contraction line
  \hbox{%
    \vrule width #1 height 0pt depth #3%
    \vrule width #2 height 0pt depth #1%
    \vrule width #1 height 0pt depth #3%
    \relax}}
\makeatother

%%Richard's macros%%
\DeclareFontFamily{OT1}{rsfs}{}
\DeclareFontShape{OT1}{rsfs}{n}{it}{<-> rsfs10}{}
\DeclareMathAlphabet{\curly}{OT1}{rsfs}{n}{it}
\newcommand\A{\mathcal A}
\newcommand\I{\curly I}
\renewcommand\k{\mathsf k}
\newcommand\Hk{H_{\mathsf k}}
\renewcommand\L{\mathcal L}
\newcommand\LL{\mathbb L}
\renewcommand\S{\mathcal S}
\renewcommand\O{\mathcal O}
\newcommand\PP{\mathbb P}

\newcommand\cP{\mathcal P}
\newcommand\cW{\mathcal W}


\newcommand\Pb{\mathcal P_\beta}

\newcommand\Db{\overline{\!D}}
\newcommand\mdot{{\scriptscriptstyle\bullet}}


\newcommand\D{\mathcal D}

\newcommand\cE{\mathcal E}
\newcommand\F{\mathcal F}
\newcommand\cA{\mathcal A}

\newcommand\cK{\mathcal K}


\newcommand\E{\mathbb E}

\newcommand\n{\mathbf{n}}
\newcommand\p{\mathbf{p}}
\newcommand\q{\mathbf{q}}
\newcommand\bfd{\mathbf{d}}

\newcommand\bslambda{\boldsymbol{\lambda}}

\newcommand\evv{\mathrm{ev}}
\newcommand\pt{\mathrm{pt}}
\newcommand\vir{\mathrm{vir}}
\newcommand\loc{\mathrm{loc}}
\newcommand\AJ{\mathrm{AJ}}
\newcommand\GW{\mathrm{GW}}
\newcommand\td{\mathrm{td}}
\newcommand\red{\mathrm{red}}
\newcommand\dd{\mathrm{d}}
\newcommand\Supp{\mathrm{Supp}}
\newcommand\sing{\mathrm{sing}}


\newcommand\ualpha{\underline \alpha}

\newcommand\Mb{\,\overline{\!M}}
\newcommand\Xb{\,\overline{\!X}}
\newcommand\MMb{\,\overline{\!\mathcal M}}
\newcommand\XXb{\,\overline{\!\mathcal X}}
\newcommand\C{\mathbb C}

\newcommand\cC{\mathcal C}

\newcommand\FF{\mathbb F}
\newcommand\GG{\mathbb G}


\newcommand\II{\mathbb I}
\newcommand\Q{\mathbb Q}
\newcommand\cQ{\mathcal Q}

\newcommand\Pg{\mathcal P_\gamma}

\newcommand\R{\mathbb R}
\newcommand\ccR{\mathcal R}


\newcommand\Z{\mathbb Z}

\newcommand\cZ{\mathcal Z}

\newcommand\m{\mathfrak m}
\renewcommand\t{\mathfrak t}

\newcommand{\rt}[1]{\stackrel{#1\,}{\rightarrow}}
\newcommand{\Rt}[1]{\stackrel{#1\,}{\longrightarrow}}
\newcommand\To{\longrightarrow}
\newcommand\into{\hookrightarrow}
\newcommand\Into{\ensuremath{\lhook\joinrel\relbar\joinrel\rightarrow}}
\newcommand\INTO{\ar@{^{(}->}[r]}
\newcommand\acts{\curvearrowright}
\newcommand\so{\ \ \Longrightarrow\ }
\newcommand\res{\arrowvert^{}_}
\newcommand\ip{\lrcorner\,}
\newcommand\comp{{\,}_{{}^\circ}}
\renewcommand\_{^{}_}
\newcommand\take{\backslash}
\newcommand{\mat}[4]{\left(\begin{array}{cc} \!\!#1 & #2\!\! \\ \!\!#3 &
#4\!\!\end{array}\right)}
\newcommand\bull{{\scriptscriptstyle\bullet}}
\newcommand\udot{^\bull}
\newcommand\dbar{\overline\partial}
\newcommand\rk{\operatorname{rk}}
\newcommand\Alb{\operatorname{Alb}}
\newcommand\tr{\operatorname{tr}}
\newcommand\coker{\operatorname{coker}}
\newcommand\im{\operatorname{im}}
\newcommand\Td{\operatorname{Td}}
\newcommand\ch{\operatorname{ch}}
\newcommand\ev{\operatorname{ev}}
\renewcommand\div{\operatorname{div}}
\newcommand\id{\operatorname{id}}
\newcommand\vol{\operatorname{vol}}
\renewcommand\Im{\operatorname{Im}}
\newcommand\Hom{\operatorname{Hom}}
\renewcommand\hom{\curly H\!om}
\newcommand\Ext{\operatorname{Ext}}
\newcommand\ext{\curly Ext}
\newcommand\At{\operatorname{At}}
\newcommand\Aut{\operatorname{Aut}}
\newcommand\ob{\operatorname{ob}}
\newcommand\Ob{\operatorname{Ob}}
\newcommand\Bl{\operatorname{Bl}}
\newcommand\Pic{\operatorname{Pic}}
\newcommand\Proj{\operatorname{Proj}\,}
\newcommand\Spec{\operatorname{Spec}\,}
\newcommand\Hilb{\operatorname{Hilb}}
\renewcommand\Re{\operatorname{Re}}
\newcommand\Sym{\operatorname{Sym}}
\newcommand\beq[1]{\begin{equation}\label{#1}}
\newcommand\eeq{\end{equation}}
\newcommand\beqa{\begin{eqnarray*}}
\newcommand\eeqa{\end{eqnarray*}}
\newcommand\foo\{foo}

\setcounter{secnumdepth}{2}
\DeclareRobustCommand{\SkipTocEntry}[4]{}
% \makeatletter
% \newcommand\@dotsep{4.5}
% \def\@tocline#1#2#3#4#5#6#7{\relax
%   \ifnum #1>\c@tocdepth % then omit
%   \else
%     \par \addpenalty\@secpenalty\addvspace{#2}%
%     \begingroup \hyphenpenalty\@M
%     \@ifempty{#4}{%
%       \@tempdima\csname r@tocindent\number#1\endcsname\relax
%     }{%
%       \@tempdima#4\relax
%     }%
%     \parindent\z@ \leftskip#3\relax \advance\leftskip\@tempdima\relax
%     \rightskip\@pnumwidth plus1em \parfillskip-\@pnumwidth
%     #5\leavevmode\hskip-\@tempdima #6\relax
%     \leaders\hbox{$\m@th
%       \mkern \@dotsep mu\hbox{.}\mkern \@dotsep mu$}\hfill
%     \hbox to\@pnumwidth{\@tocpagenum{#7}}\par
%     \nobreak
%     \endgroup
%   \fi}
% \makeatother 

\begin{document}
\title{BPS state counts of local elliptic surfaces via formal geometry and the topological vertex}
\author[J.~Bryan, M.~Kool]{Jim Bryan and Martijn Kool \vspace{-5mm}}
\maketitle

\begin{abstract}
We compute the (connected) stable pair invariants of $X = \mathrm{Tot}(K_S)$ where $S$ is an elliptic surface with section and at worst 1-nodal singular fibres. The calculation includes thickenings to all orders in both the surface and fibre direction. We use a new method combining motivic arguments and torus localization. 

We stratify the moduli space according to underlying reduced support $C_{\red}$ of the stable pair and compute the contribution of each $C_{\red}$ individually. The contribution of $C_{\red}$ can be split up into a part coming from the nodes of $C_{\red}$ and the complement of the nodes $C_{\red}^{\circ} \subset C_{\red}$. The formal neighbourhood of $C_{\red}^{\circ}$ in $X$ is isomorphic to a formal neighbourhood of $C_{\red}^{\circ}$ inside its normal bundle. This gives us lots of $\C^{*}$-actions. 

Localization with respect to the torus actions leads to a vertex calculation which can be performed explicitly. As special cases we find a new proof of the Katz--Klemm--Vafa formula in the primitive case (independent of Kawai--Yoshioka's formula) and the BPS spectrum of the local rational elliptic surface.
\end{abstract}
\thispagestyle{empty}

\tableofcontents


\section{Introduction}

Let $X$ be a smooth projective 3-fold, $\chi \in \Z$, and $\beta \in H_2(X)$ a curve class. Denote by $P_\chi(X,\beta)$ the moduli space of stable pairs $I^\mdot = [\O_X \rightarrow \F]$ on $X$ for which $\chi(\F) = \chi$, and the scheme theoretic support of $F$ has curve class $\beta$. The moduli space $P_\chi(X,\beta)$ is an instance of Le Potier's more general moduli spaces of stable pairs \cite{LeP}. 
%and has a universal stable pair
%\[
%\II^\mdot = [\O_{P_\chi(X,\beta) \times X} \rightarrow \FF].
%\]
The deformation-obstruction theory of stable pairs \emph{does not} provide a perfect obstruction theory for $P_\chi(X,\beta)$. R.~P.~Thomas and R.~Pandharipande realize $P_\chi(X,\beta)$ as a component of the moduli space of complexes in $D^b(X)$ with trivial determinant. Viewed as a moduli space of complexes $P_\chi(X,\beta)$ \emph{does} have a perfect obstruction theory \cite{PT1}. When $X$ is in addition Calabi-Yau, this perfect obstruction theory is symmetric and the stable pair invariants of $X$ are defined as the degree of the virtual cycle
\[
P_{\chi,\beta}(X) := \int_{[P_\chi(X,\beta)]^{\vir}} 1.
\]
By a theorem of K.~Behrend \cite{Beh}
\[
\int_{[P_\chi(X,\beta)]^{\vir}} 1 = \int_{P_\chi(X,\beta)} \nu_B \ \dd e,
\]
where $\nu_B : P_\chi(X,\beta) \rightarrow \Z$ is Behrend's constructible function and $e(\cdot)$ denotes topological Euler characteristic. 

In this paper $\pi : S \rightarrow B$ denotes an elliptic surface. This means $S$ is a smooth surface, $B$ a smooth curve of genus $g(B)$, and $\pi$ a holomorphic map with general fibre a connected smooth genus 1 curve \cite{Mir}. We make two assumptions:\todo{Can more be said about such surfaces? I don't thin we need $B \cong \PP^1$?}
\begin{itemize}
\item $\pi$ has a section $B \hookrightarrow S$,
\item all singular fibres of $\pi$ are of Kodaira type $I_1$, i.e.~rational 1-nodal curves.
\end{itemize}
We are interested in the case $X = \mathrm{Tot}(K_S)$ and $\beta = B + d F$, where $B$ is the class of the section and $F$ is the class of the fibre. Since $X$ is a non-compact Calabi-Yau 3-fold we require curves of $P_\chi(X,\beta)$ to have proper support. Non-compactness of $P_\chi(X,\beta)$ also means we do not have a virtual cycle, so one should \emph{define} stable pair invariants in this setting either\todo{How are both approaches related?} by Graber-Pandharipande's localization formula \cite{GP} or by integration of $\nu_B$ over $P_\chi(X,\beta)$. We choose the latter approach.\todo{EULER CHAR FOR THIS VERSION FOR NOW!}

Consider the (disconnected) generating function
\begin{align}
\begin{split} \label{discon}
Z^{\mdot P}(q,y) &:= \sum_{d \geq 0} \sum_{\chi} P_{\chi, B+dF}(X) q^{\chi} y^d, \\
P_{\chi, B+dF}(X) &:= e(P_{\chi}(X,B+dF)).
\end{split}
\end{align}
The connected generating function is defined as \cite{PT1}
\begin{align}
\begin{split} \label{con}
Z^{P}(q,y) &:= \frac{\sum_{d \geq 0} \sum_{\chi} P_{\chi,B+dF}(X) q^{\chi} y^d}{\sum_{d \geq 0} \sum_{\chi} P_{\chi,dF}(X) q^{\chi} y^d}, \\
P_{\chi,dF}(X) &:= e(P_{\chi}(X,dF)),
\end{split}
\end{align}
where $P_{\chi,0}(X) = 1$ for all $\chi$. Our main result is the following.
\begin{theorem} \label{mainthm}
Let $X = \mathrm{Tot}(K_S)$ where $S \rightarrow B$ is an elliptic surface with section $B$ of genus $g(B)$ and $N$ 1-nodal fibres. Then
\begin{align*}
Z^{P}(q,y) = \bigg( \frac{q}{(1-q)^2} \bigg)^{1-g(B)} \prod_{i=1}^{\infty} \frac{1}{(1-y^i)^{N - 2e(B)} (1 - q y^i)^{e(B)} (1 - q^{-1} y^i)^{e(B)}}.
\end{align*}
\end{theorem}
The proof is divided into five movements: \\

\noindent \emph{Stratifation}, \emph{Restriction}, \emph{Formalization}, \emph{Localization}, \emph{Finale (Schur)}. \\

\todo{Applications for this paper: stable pair version of KKV in the primitive case independent of KY, gen fun for rational elliptic surface. Future applications: elliptically fibres CY3's, refinement and comparison to refined KKV, ...}



\section{Stratification}

Let $\beta$ be Poincar\'e dual to $B+dF$. The projections
\[
\varpi : X \longrightarrow S \longrightarrow B
\]
induce a push-forward map
\[
P_\chi(X,\beta) \longrightarrow \Sym^d(B), \ I^\mdot = [\O_X \rightarrow \F] \mapsto \varpi_* \F.
\]
We denote the fibre of $S \rightarrow B$ over $p \in B$ by $F_p$. Let
$$
\p := \sum_{i=1}^m d_i p_i  \subset B
$$
be an effective divisor with all $d_i > 0$ and $\sum_{i=1}^{m} d_i = d$. Consider the reduced curve
$$
C_{\p} := \bigcup_{i=1}^m F_{p_i} \subset S \subset X,
$$
where $S \subset X$ is the zero-section. The fibre of $\varpi_*$ over $\p$ is
$$
P_\chi(X, \p) := \big\{ I^\mdot = [\O_X \rightarrow \F] \in P_\chi(X,\beta) \ : \ \varpi_* \F = \p \big\},
$$
i.e.~the locally closed subset of stable pairs $I^\mdot = [\O_X \rightarrow \F] \in P_\chi(X,\beta)$ for which $\F$ has set theoretic support $C_{\p}$ and multiplicity $d_i$ along $F_{p_i}$ for all $i$. We are interested in the stratification
\[
P_\chi(X,\beta) = \coprod_{\p \in \Sym^d(B)} P_\chi(X,\p).
\]
\begin{lemma} \label{lem1}
\[
e(P_\chi(X,\beta)) = \int_{\p \in \Sym^d(B)} e(P_\chi(X,\p)) \ \dd e.
\]
\end{lemma}
\begin{proof}
\cite{MacP}.
\end{proof}


\section{Restriction} \label{secres}

By Lemma \ref{lem1} we are reduced to computing $e(P_\chi(X,\p))$ for any 
$$
\p = \sum_{i=1}^m d_i p_i \in \Sym^d(B).
$$ 
Let $q \in C_{\p}$ be one of the \emph{nodal} singularities (either a node in a singular fibre or an intersection point of a fibre with the section). We denote by $\widehat{X}_q$ the formal neighbourhood of $\{q\} \subset X$ and by $X \setminus q$ the complement of $\{q\} \subset X$. Let\todo{A little more care in the def of this moduli space is needed since $X \setminus q$ is non-compact AND the supports of the stable pairs are non-compact.}
$$
P_{\chi}(X \setminus q, \p)
$$
be the moduli space of stable pairs $I^\mdot = [\O_{X \setminus q} \rightarrow \F]$ such that $\chi(\F) = \chi$, $\F$ has set theoretic support $C_{\p} \setminus q$, $\F$ has multiplicity 1 along $B \setminus q$, and $\F$ has multiplicity $d_i$ along $F_{p_i} \setminus q$ for all $i$. Moreover let\todo{This might need a little more care too since stable pair theory is not yet defined for formal schemes. }  
$$
P_{\chi}(\widehat{X}_q,\p)
$$
be the moduli spaces of stable pairs $I^\mdot = [\O_{\widehat{X}_q} \rightarrow \F]$ such that $\chi(\F) = \chi$, $\F$ has set theoretic support $\widehat{C_{\p}}$, $\F$ has multiplicity 1 along $\widehat{B}$, and $\F$ has multiplicity $d_i$ along $\widehat{F_{p_i}}$ for all $i$. Here $\widehat{C_{\p}}$, $\widehat{B}$, $\widehat{F_{p_i}}$ denote the lifts\footnote{Let $\widehat{X}_{Z}$ be the formal completion of any scheme along a closed subset $Z$. If $\cE$ is a coherent sheaf on $X$ then one can define a lift $\cE^{\Delta}$ to $\widehat{X}_{Z}$ \cite{Har}. In the case $\cE = \I \subset \O_X$ is an ideal sheaf, this provides an ideal sheaf $\I^{\Delta} \subset \O_{\widehat{X}_Z}$ \cite{Har}.} of $C_{\p}$, $B$, $F_{p_i}$ to $\widehat{X}_q$. We are interested in the injective morphism induced by restriction
\begin{equation} \label{embd1}
P_\chi(X,\p) \hookrightarrow \coprod_{\chi = \chi_1 + \chi_2} P_{\chi_1}(X \setminus q,\p) \times P_{\chi_2}(\widehat{X}_q,\p).
\end{equation}
The image of this morphism can be characterized as follows. Let\todo{May not exist. In general work in stalk? See Jim's e-mail on 25.6.2014.} 
$$
U = \Spec \C[x,y] \subset S
$$
be an open affine neighbourhood of $q$ over which $X = \mathrm{Tot}(K_S)$ trivializes with fibre coordinate $z$. Then $\widehat{X}_q$ is the reduced point $q$ with sheaf of rings
$$
\O_{\widehat{X}_q} \cong \widehat{\O}_{X,q} \cong \C[\![x,y,z]\!].
$$
Suppose the coordinates are chosen such that $C_{\p}$ is defined by $xy = z = 0$. Define open subsets
$$
V = \{ x \neq 0 \} \subset U, \ W = \{ y \neq 0 \} \subset U.
$$
\begin{lemma} \label{lem2}
An element
$$
([s_1 : \O_{X \setminus q} \rightarrow \F_1], [s_2 : \O_{\widehat{X}_q} \rightarrow \F_2] )
$$
lies in the image of the embedding \eqref{embd1} if and only if the Cohen-Macaulay support curves $C_{\F_1}$, $C_{\F_2}$ underlying both stable pairs glue i.e.
\begin{align*}
&\Gamma(\widehat{X}_q, \I_{C_{\F_2}}) \otimes_{\C[\![x,y,z]\!]} \C[\![x^{\pm},y,z]\!] \cong \widehat{\Gamma}(V \times \C, \I_{C_{\F_1}}|_{V \times \C}), \\
&\Gamma(\widehat{X}_q, \I_{C_{\F_2}})) \otimes_{\C[\![x,y,z]\!]} \C[\![x,y^{\pm},z]\!] \cong \widehat{\Gamma}(W \times \C, \I_{C_{\F_1}}|_{W \times \C}), 
\end{align*}
%\begin{align*}
%&\Gamma(\widehat{X}_q, \F_2) \otimes_{\C[\![x,y,z]\!]} \C[\![x^{\pm},y,z]\!] \cong \widehat{\Gamma}(V \times \C, \F_1|_{V \times \C}) \\
%&\Gamma(\widehat{X}_q, \F_2) \otimes_{\C[\![x,y,z]\!]} \C[\![x,y^{\pm},z]\!] \cong \widehat{\Gamma}(W \times \C, \F_1|_{W \times \C}) \\
%&\Gamma(\widehat{X}_q, s_2) \otimes_{\C[\![x,y,z]\!]} \C[\![x^{\pm},y,z]\!] \cong \widehat{\Gamma}(V \times \C, s_1|_{V \times \C}) \\
%&\Gamma(\widehat{X}_q, s_2) \otimes_{\C[\![x,y,z]\!]} \C[\![x,y^{\pm},z]\!] \cong \widehat{\Gamma}(W \times \C, s_1|_{W \times \C}),
%\end{align*}
where $\Gamma(\cdot)$ denotes the global section functor, $\widehat{(\cdot)}$ is the formal completion of the module $(\cdot)$, and $\I_{C_{\F_1}}$, $\I_{C_{\F_2}}$ are ideal sheaves.
\end{lemma}
\begin{proof}
Perhaps Ben-Bassat--Temkin's \cite{BT} abstract setup (or a stable pairs version) reduces to this when $Z $ (in their notation) is just a point.\todo{See Jim's fpqc e-mail on 25.6.2014.} \ Note: life is not too bad because only the support curve has to glue. This is because the section of a stable pair is an isomorphism outside a 0-dim subscheme.
\end{proof}

We want to apply the above construction not just for one point $q$. Let $q_1, \ldots, q_n \in C_{\p}$ be all nodes. For notational simplicity we write $$X^\circ := X \setminus \{q_1, \ldots, q_n\}.$$ We embed 
\[
P_\chi(X,\p) \hookrightarrow \coprod_{\chi = \chi' + \chi_1 +  \cdots + \chi_n}  P_{\chi'}(X^{\circ},\p) \times \prod_{j=1}^{n} P_{\chi_j}(\widehat{X}_{q_j},\p).
\]
The image is characterized by gluing conditions as in Lemma \ref{lem2} at each of the nodes $q_j$.


\section{Formalization}

In the previous section we characterized the image of $P_\chi(X,\p)$ under restriction to special points and their complements 
\[
P_\chi(X,\p) \hookrightarrow \coprod_{\chi = \chi' + \chi_1 +  \cdots + \chi_n}  P_{\chi'}(X^{\circ},\p) \times \prod_{i=1}^{n} P_{\chi_j}(\widehat{X}_{q_j},\p)
\]
In this section we relate $P_{\chi}(X^{\circ},\p)$ to moduli spaces of stable pairs on the (punctured) fibres/section inside their normal bundle.

Recall that 
$$
\p := \sum_{i=1}^m d_i p_i \in \Sym^d(B), \ C_{\p} := \bigcup_{i=1}^m F_{p_i}, 
$$ 
and $q_1, \ldots, q_n$ are all nodes of $C_{\p}$. We have an inclusion 
$$
P_{\chi}(X^\circ,\p) \subset P_{\chi}(X^\circ,\beta),
$$
where $P_{\chi}(X^\circ,\beta)$ denotes the moduli space of stable pairs $I^\mdot = [\O_{X^\circ} \rightarrow \F]$ on $X^\circ$ such that $\chi(\F) = \chi$ and the closure of the scheme theoretic support of $\F$ in $X$ is proper with class $\beta$. We can make a formal completion of the former space along the latter
$$
\widehat{P}_{\chi}(X^\circ,\beta)_{P_{\chi}(X^\circ,\p)}.
$$
Obviously the underlying topological space is unchanged so
$$
e(\widehat{P}_{\chi}(X^\circ,\beta)_{P_{\chi}(X^\circ,\p)}) = e(P_{\chi}(X^\circ,\p)).
$$
Passing to the formal completion allows us to consider stable pairs on the formal completion of $X^\circ$ along $C_{\p}^{\circ} := C_{\p} \setminus \{q_1, \ldots, q_n\}$. This formal completion is denoted by
$$
\widehat{X^\circ}_{C_{\p}^{\circ}}.
$$ 
\begin{lemma} \label{lem3}
There exists a canonical isomorphism
\[
\widehat{P}_{\chi}(X^\circ,\beta)_{P_{\chi}(X^\circ,\p)} \cong P_{\chi}(\widehat{X^\circ}_{C_{\p}^{\circ}}, \p),
\]
where $P_{\chi}(\widehat{X^\circ}_{C_{\p}^{\circ}}, \p)$ is the moduli space of stable pairs $I^{\mdot} = [\O \rightarrow \F]$ on $\widehat{X^\circ}_{C_{\p}^{\circ}}$ such that $\chi(\F) = \chi$, $\F$ has multiplicity 1 along $\widehat{B^\circ}$, and $\F$ has multiplicity $d_i$ along $\widehat{F_{p_i}^{\circ}}$ for all $i$. Here $\widehat{B^\circ}$, $\widehat{F_{p_i}^{\circ}}$ denote the lifts of $B^\circ$, $F_{p_i}^{\circ} := F_{p_i} \setminus \{q_1, \ldots, q_n\}$ to $\widehat{X^\circ}_{C_{\p}^{\circ}}$. 
\end{lemma}
\begin{proof}
Jim's idea of categorical limits. This should be formal.
\end{proof}
Let us take a closer look at the formal scheme $\widehat{X^\circ}_{C_{\p}^\circ}$. 
%Denote the smooth fibres of $C_{\p}$ by $F_{p_1}$, $\ldots$, $F_{p_k}$ and the nodal fibres by $F_{p_{k+1}}$, $\ldots$, $F_{p_m}$. Denote the section of $C_{\p}$ by $E$. 
Removing the nodes points $q_1, \ldots, q_n$ we obtain smooth curves $B^{\circ}$, $F_{p_i}^{\circ}$ and
$$
C_{\p}^{\circ} \cong B^{\circ} \sqcup F_{p_1}^{\circ} \sqcup \cdots \sqcup F_{p_m}^{\circ}.
$$
This isomorphism also holds at the level of formal schemes.
\begin{lemma} \label{lem4}
There exists a canonical isomorphism
$$
\widehat{X^\circ}_{C_{\p}^\circ} \cong \widehat{X^{\circ}}_{B^{\circ}} \sqcup \widehat{X^\circ}_{F_{p_1}^{\circ}} \sqcup \cdots \sqcup \widehat{X^\circ}_{F_{p_m}^{\circ}},
$$
where $\widehat{X^{\circ}}_{B^{\circ}}$, $\widehat{X^\circ}_{F_{p_i}^{\circ}}$ are the formal completions of $X$ along $B^\circ$, $F_{p_i}^{\circ}$. 
\end{lemma}
\begin{proof}
Disjoint union commutes with formal completion\todo{IS THIS REALLY TRUE? Sounds plausible.}.
\end{proof}

This lemma allows us to pass to the normal bundles of $B^\circ \subset X^\circ$, $F_{p_i}^{\circ} \subset X^\circ$.
\begin{lemma} \label{lem5}
There exists natural isomorphisms
$$
\widehat{X^{\circ}}_{B^{\circ}} \cong \widehat{N_{B^{\circ} / X^\circ}}_{B^\circ}, \ \widehat{X^{\circ}}_{F_{p_i}^{\circ}} \cong \widehat{N_{F_{p_i}^{\circ} / X^\circ}}_{F_{p_i}^\circ}, 
$$
where $\widehat{N_{B^{\circ} / X^\circ}}_{B^\circ}$, $\widehat{N_{F_{p_i}^{\circ} / X^\circ}}_{F_{p_i}^{\circ}}$ are the formal completions of the normal bundles $N_{B^\circ / X^\circ}$, $N_{F_{p_i}^{\circ} / X^\circ}$ along their zero sections $B^\circ$, $F_{p_i}^{\circ}$.
\end{lemma}
\begin{proof}
For the fibres we proved this rigorously using sections of $\O / \I^{r+1} \rightarrow \O / \I^{r}$ pulled back from the base $B$. This requires flatness of $\pi$. For the section we use Davesh's argument.
\end{proof}

Lemmas \ref{lem3}, \ref{lem4}, \ref{lem5} allow us write
$$
\widehat{P}_{\chi}(X^\circ,\beta)_{P_{\chi}(X^\circ,\p)} \cong \coprod_{\chi = \chi' + \chi_1 + \cdots + \chi_{m}} P_{\chi'}(\widehat{N_{B^{\circ} / X^\circ}}_{B^\circ}, \p) \times \prod_{i=1}^{m} P_{\chi_i}(\widehat{N_{F_{p_i}^{\circ} / X^\circ}}_{F_{p_i}^{\circ}},\p),
$$
where $P_{\chi}(\widehat{N_{F_{p_i}^{\circ} / X^\circ}}_{F_{p_i}^{\circ}}, \p)$ is the moduli space of stable pairs $I^\mdot = [\O \rightarrow \F]$ on $\widehat{N_{F_{p_i}^{\circ} / X^\circ}}_{F_{p_i}^{\circ}}$ with $\chi(\F) = \chi$ and $\F$ has set theoretic support $\widehat{F_{p_i}^{\circ}}$ with multiplicity $d_i$. Here $\widehat{F_{p_i}^{\circ}}$ denotes the lift of $F_{p_i}^\circ$ to $\widehat{N_{F_{p_i}^{\circ} / X^\circ}}_{F_{p_i}^{\circ}}$. Similar for $P_{\chi}(\widehat{N_{B^{\circ} / X^\circ}}_{B^{\circ}},\p)$ where the multiplicity along $\widehat{B^\circ}$ is required to be one.

Finally we want to ``undo'' the formal completion on the normal bundles by using categorical limits as in Lemma \ref{lem3}. We denote by 
$$
P_{\chi}(N_{F_{p_i}^{\circ} / X^{\circ}}, \p) \subset P_{\chi}(N_{F_{p_i}^\circ / X^\circ}, d_i F_{p_i}^{\circ})
$$
moduli spaces of stable pairs $I^\mdot = [\O \rightarrow \F]$ on $N_{B^\circ / X^\circ}$ with $\chi(\F) = \chi$. The first has $\F$ with set theoretic support $F_{p_i}^{\circ}$ and multiplicity $d_i$. The second has $\F$ such that the closure of its set theoretic support in $N_{F_{p_i} / X}$ is proper with class $d_i F_{p_i}$. Similarly we consider 
$$
P_{\chi}(N_{B^{\circ} / X^{\circ}}, \p) \subset P_{\chi}(N_{B^\circ / X^\circ}, B^\circ).
$$
The argument presented in the proof of Lemma \ref{lem3} gives
\begin{align*}
P_{\chi}(\widehat{N_{B^{\circ} / X^\circ}}_{B^\circ}, \p) &\cong \widehat{P}_{\chi}(N_{B^\circ / X^\circ}, B^\circ)_{P_{\chi}(N_{B^{\circ} / X^\circ}, \p)}  \\
P_{\chi}(\widehat{N_{F_{p_i}^{\circ} / X^\circ}}_{F_{p_i}^{\circ}}, \p) &\cong \widehat{P}_{\chi}(N_{F_{p_i}^{\circ} / X^\circ}, d_i F_{p_i}^{\circ})_{P_{\chi}(N_{F_{p_i}^{\circ} / X^\circ}, \p)}.
\end{align*}
Combining all arguments of this section gives the following result.
\begin{proposition}
We have natural isomorphisms
\begin{align*}
\widehat{P}_{\chi}(X^\circ,\beta)_{P_{\chi}(X^\circ,\p)} \cong \coprod_{\chi = \chi' + \chi_1 + \cdots + \chi_m} &\widehat{P}_{\chi'}(N_{B^\circ / X^\circ}, B^\circ)_{P_{\chi}(N_{B^{\circ} / X^\circ}, \p)} \\
&\times \prod_{i=1}^{m} \widehat{P}_{\chi_i}(N_{F_{p_i}^{\circ} / X^\circ}, d_i F_{p_i}^{\circ})_{P_{\chi}(N_{F_{p_i}^{\circ} / X^\circ}, \p)}.
\end{align*}
In particular on the underlying topological space we have a homeomorphism
\[
P_{\chi}(X^\circ,\p) \approx \coprod_{\chi = \chi' + \chi_1 + \cdots + \chi_m} P_{\chi}(N_{B^{\circ} / X^\circ}, \p) \times \prod_{i=1}^{m} P_{\chi}(N_{F_{p_i}^{\circ} / X^\circ}, \p).
\]
\end{proposition}
\begin{proof}
Combination of the above.
\end{proof}


\section{Localization}


\subsection{Localization I}

In the previous two sections we constructed an embedding
\begin{equation} \label{embd2}
P_\chi(X,\p) \hookrightarrow \coprod_{\chi = \chi' + \chi_1 +  \cdots + \chi_n}  P_{\chi'}(X^{\circ},\p) \times \prod_{j=1}^{n} P_{\chi_j}(\widehat{X}_{q_i},\p)
\end{equation}
and homeomorphisms
\begin{equation} \label{homeo}
P_{\chi}(X^\circ,\p) \approx \coprod_{\chi = \chi' + \chi_1 + \cdots + \chi_m} P_{\chi}(N_{B^{\circ} / X^\circ}, \p) \times \prod_{i=1}^{m} P_{\chi}(N_{F_{p_i}^{\circ} / X^\circ}, \p).
\end{equation}
Each normal bundle has a natural $\C^{*2}$-action given by scaling the fibres. The action of $\C^{*2}$ on $P_{\chi}(N_{B^{\circ} / X^\circ}, \p)$ is trivial\footnote{This action is transverse to the section and our stable pairs have multiplicity 1 along $B$.} so we ignore it. Therefore $\C^{*2m}$ acts naturally on $P_{\chi}(X^\circ,\p)$ by \eqref{homeo}. 

Since each $\widehat{X}_{q_j}$ is just the reduced point $q_j$ with structure sheaf
\[
\O_{\widehat{X}_{q_j}} \cong \widehat{\O}_{X,q_j} \cong \C [\![x,y,z]\!],
\]
we have $\C^{*3}$ acting on this space by $(s_1,s_2,s_3 ) \cdot (x,y,z) = (s_1 x,s_2 y, s_3 z)$. In total we get an action of $\C^{*(2m+3n)}$ on the RHS of \eqref{embd2}. However $P_\chi(X,\p)$ is not invariant under this full torus.
\begin{lemma}
Define the a $2m$-dimensional subtorus $T \subset \C^{*(2m+3n)}$ by the following equations. For any nodal fibre $F_{p_i}$ with node $q_j$ let $(t_{1}^{(i)}, t_{2}^{(i)})$ be the coordinates of $\C^{*2}$ acting on $N_{F_{p_i}^{\circ} / X^\circ}$ and let $(s_{1}^{(j)},s_{2}^{(j)},s_{3}^{(j)})$ be the coordinates of $\C^{*3}$ acting on $\widehat{X}_{q_j}$, then
\[
s_{1}^{(j)}=s_{2}^{(j)}=t_{1}^{(i)}, \ s_{3}^{(j)} = t_{2}^{(i)}. 
\]
For any (not necessarily nodal) fibre $F_{p_i}$ and $\{q_j\} = F_{p_i} \cap B$ let $(t_{1}^{(i)}, t_{2}^{(i)})$ be the coordinates of $\C^{*2}$ acting on $N_{F_{p_i}^{\circ} / X^\circ}$ and let $(s_{1}^{(j)},s_{2}^{(j)},s_{3}^{(j)})$ be the coordinates of $\C^{*3}$ acting on $\widehat{X}_{q_j}$, then
\[
s_{1}^{(j)}=1, \ s_{2}^{(j)}=t_{1}^{(i)}, \ s_{3}^{(j)} = t_{2}^{(i)}. 
\]
Then $T$ leaves $P_\chi(X,\p)$ invariant.
\end{lemma}
\begin{proof}
Use the gluing conditions of Lemma \ref{lem2}. This does require passing through several isomorphisms which could be tricky.\todo{How tedious will this be...?}
\end{proof}

Since $e(P_\chi(X,\p)) = e(P_\chi(X,\p)^T)$ we are reduced to understanding the fixed point locus $P_\chi(X,\p)^T$. Let 
\[
([s : \O_{X^\circ} \rightarrow \cE], \{ [s_j : \O_{\widehat{X}_{q_j}} \rightarrow \F_j] \}_{j=1}^{n} ) \in \coprod_{\chi = \chi' + \chi_1 +  \cdots + \chi_n}  P_{\chi'}(X^{\circ},\p) \times \prod_{j=1}^{n} P_{\chi_j}(\widehat{X}_{q_j},\p).
\]
This element lies in $P_\chi(X,\p)$ if and only if the underlying Cohen-Macaulay curves $C_{\cE}$, $C_{\F_j}$ glue as described in Lemma \ref{lem2}. This element is in addition $T$-fixed if and only if each of the restrictions
\begin{align*}
&\Gamma(\widehat{X}_{q_j}, \I_{C_{\F_j}}) \otimes_{\C[\![x,y,z]\!]} \C[\![x^{\pm},y,z]\!] \cong \widehat{\Gamma}(V \times \C, \I_{C_{\cE}}|_{V \times \C}) \\
&\Gamma(\widehat{X}_{q_j}, \I_{C_{\F_j}})) \otimes_{\C[\![x,y,z]\!]} \C[\![x,y^{\pm},z]\!] \cong \widehat{\Gamma}(W \times \C, \I_{C_{\cE}}|_{W \times \C}) 
\end{align*}
is given by a monomial ideal in two variables, i.e.~a (2-dimensional) partition. For each node which is the intersection point of a (not necessarily nodal) fibre $F_{p_i}$ with the zero section, this amounts to specifying a partition $\lambda_i$ of $d_i$ in the fibre direction. The partition in the section direction is $(1)$, because the multiplicity of $C_{\cE}$ along $B$ is 1. For each node of a nodal fibre $F_{p_i}$ the cross-section of the Cohen-Macaulay support curve has to be given by the same partitions $\lambda_i$. Altogether we have fixed partitions $\bslambda = \{\lambda_i \vdash d_i\}_{i=1}^{m}$. Denote by 
$$
P_{\chi}(X^{\circ},\p)_{\bslambda} \subset P_{\chi}(X^{\circ},\p), \  P_{\chi}(\widehat{X}_{q_j},\p)_{\bslambda} \subset P_{\chi}(\widehat{X}_{q_j},\p)
$$
the locally closed subsets for which the underlying Cohen-Macaulay curves have restrictions described by partitions $\bslambda$ as above. We arrive at the following conclusion.
\begin{lemma} \label{lem6}
The embedding \eqref{embd2} induces a bijective morphism
$$
P_\chi(X,\p)^T \cong \coprod_{\chi = \chi' + \chi_1 +  \cdots + \chi_n} \coprod_{\bslambda = \{\lambda_i \vdash d_i\}_{i=1}^{m}}  P_{\chi'}(X^{\circ},\p)_{\bslambda} \times \prod_{j=1}^{n} P_{\chi_j}(\widehat{X}_{q_j},\p)_{\bslambda},
$$
where $T$ is the torus of Lemma \ref{lem5}.
\end{lemma}
\begin{proof}
Easy from the above.
\end{proof}


\subsection{Localization II}

In this subsection we focus attention on $e(P_{\chi}(\widehat{X}_{q_j},\p)_{\bslambda})$ for any $\bslambda = \{\lambda_i \vdash d_i\}_{i=1}^{m}$. On each moduli space $P_{\chi}(\widehat{X}_{q_j},\p)$ we have a $\C^{*3}$-action as described in the previous subsection. This action leaves 
$$
P_{\chi}(\widehat{X}_{q_j},\p)_{\bslambda} \subset P_{\chi}(\widehat{X}_{q_j},\p)
$$
invariant. The fixed point locus $P_{\chi}(\widehat{X}_{q_j},\p)_{\bslambda}^{\C^{*3}}$ consists of \emph{isolated} fixed points which can be counted using the vertex/edge formalism for stable pairs developed by R.~Pandharipande and R.~P.~Thomas \cite{PT2}. Note that the fixed loci indeed consist of isolated reduced points since one leg is always empty \cite{PT2}. There are two cases: \\

\noindent \textbf{Case 1:} $q_j$ is a node of a nodal fibre $F_{p_i}$. In this case the legs of the elements of $P_{\chi}(\widehat{X}_{q_j},\p)_{\bslambda}^{\C^{*3}}$ are fixed by the partitions $(\lambda_i, \lambda_{i}^{t}, \varnothing)$ where $(\cdot)^{t}$ denotes the dual partition and we use the ordering convention of \cite{ORV}. The generating function is given by the stable pairs vertex\todo{\cite{PT2, ORV} are signed Euler chars, whereas for the moment we are doing ordinary Euler chars. $W_{\lambda,\mu,\nu}(q)$ are understood in this way for now.}
\begin{equation} \label{formalgenfun1}
W_{\lambda_i, \lambda_{i}^{t}, \varnothing}(q) = \sum_{\chi} e(P_{\chi}(\widehat{X}_{q_j},\p)_{\bslambda}^{\C^{*3}}) q^{\chi} =  \sum_{ \cQ \in P_{\chi}(\widehat{X}_{q_j},\p)_{\bslambda}^{\C^{*3}}} w(\cQ) q^{l(\cQ) + 2 |\lambda_i|},
\end{equation}
where we use the notation of \cite{PT2}. \\

\noindent \textbf{Case 2:} $q_j$ is a node arising from $B$ intersecting a fibre $F_{p_i}$. In this case the legs of the elements of $P_{\chi}(\widehat{X}_{q_j},\p)_{\bslambda}^{\C^{*3}}$ are fixed by the partitions $(\lambda_i, (1), \varnothing)$. The generating function is given by the stable pairs vertex
\begin{equation} \label{formalgenfun2}
W_{\lambda_i, (1), \varnothing}(q) = \sum_{\chi} e(P_{\chi}(\widehat{X}_{q_j},\p)_{\bslambda}^{\C^{*3}}) q^{\chi} = \sum_{ \cQ \in P_{\chi}(\widehat{X}_{q_j},\p)_{\bslambda}^{\C^{*3}}} w(\cQ) q^{l(\cQ) + |\lambda_i| + 1}.
\end{equation}


\subsection{Punctured curves}

In this subsection we consider $e(P_{\chi}(X^{\circ},\p)_{\bslambda})$ for any $\bslambda = \{\lambda_i \vdash d_i\}_{i=1}^{m}$. Recall the homeomorphism \eqref{homeo} and define locally closed subsets
$$
P_{\chi}(N_{B^{\circ} / X^\circ}, \p)_{\bslambda} \subset P_{\chi}(N_{B^{\circ} / X^\circ}, \p), \ P_{\chi}(N_{F_{p_i}^{\circ} / X^\circ}, \p)_{\lambda} \subset P_{\chi}(N_{F_{p_i}^{\circ} / X^\circ}, \p)
$$
with specified ``cross-sections'' $\bslambda$ of the underlying Cohen-Macaulay curves. Since the Cohen-Macaulay curves underlying the stable pairs in $P_{\chi}(N_{B^{\circ} / X^\circ}, \p)$ have multiplicity 1, this space is just a Hilbert scheme of points on $B^\circ$ \cite{PT3}
\[
P_{\chi}(N_{B^{\circ} / X^\circ}, \p) \cong \Hilb^n(B^{\circ})
\]
where
$$
\chi = 1 - g(B) + n.
$$
Therefore
\begin{equation} \label{secgenfun}
\sum_{\chi} e(P_{\chi}(N_{B^{\circ} / X^\circ}, \p)) q^{\chi} = q^{1-g(B)} \sum_{n = 0}^{\infty} e(\Hilb^n(B^{\circ})) q^n = \frac{q^{1-g(B)}}{(1-q)^{e(B^{\circ})}}.
\end{equation}
The curves $F_{p_i}^{\circ}$ coming from a nodal fibre are punctured $\PP^1$'s
$$
F_{p_i}^{\circ} \cong \PP^1 \setminus \{3 \ pts\} \cong \C^{*} \setminus pt.
$$
The curves $F_{p_i}^{\circ}$ coming from a smooth fibre are smooth elliptic curves $E$ with one puncture. Moreover all normal bundles are in fact \emph{trivial}. Indeed for any fibre $F$ of the elliptic surface $\pi : S \rightarrow B$ we have
$$
N_{F / X} \cong N_{F/S} \oplus N_{S/X}|_{F} \cong \O_{F}(F) \oplus K_{S}|_{F} \cong \O_{F} \oplus \O_{F}.
$$ 
The last isomorphism follows from $F^2 = 0$ and the formula for the canonical divisor of an elliptic fibration \cite{Mir}
$$
K_S = \pi^* D,
$$
where $D$ is a divisor of degree $\chi(\O_S) - \chi(\O_B)$ on $B$. Therefore
$$
N_{F_{p_i}^{\circ} / X^\circ} \cong F_{p_i}^{\circ} \times \C^2 \cong \Big\{ \begin{array}{cc} (\C^{*} \setminus pt) \times \C^2 & \mathrm{if } \ F_{p_i} \ \mathrm{is \ nodal} \\ (E \setminus pt) \times \C^2 & \mathrm{if } \ F_{p_i} \ \mathrm{is \ smooth.} \end{array}
$$
The generating functions of the trivial rank 2 bundles over $F_{p_i} \setminus q_j \cong \C^*$ (when $F_{p_i}$ is nodal with node $q_j$) and $F_{p_i} \cong E$ (when $F_{p_i}$ is smooth) are easy. Indeed in the former case case $\C^*$ acts (freely) on itself by multiplication and in the latter case $E$ acts (freely) on itself by addition. These actions lift to free actions on the moduli spaces. We obtain the following result.
\begin{lemma} \label{lem7}
The following equalities hold
\begin{align*}
&\sum_{\chi} e(P_{\chi}(\C^* \times \C^2,\p)_{\bslambda} q^{\chi} = 1, \\
&\sum_{\chi} e(P_{\chi}(E \times \C^2,\p)_{\bslambda} q^{\chi} = 1.
\end{align*}
\end{lemma} 
\begin{proof}
Easy using freeness of the action and $e(\C^*) = e(E) = 0$.
\end{proof}

The required generating functions can be computed by using the restriction argument of Section \ref{secres} once more. Let $C$ be any smooth curve and consider the 3-fold $C \times \C^2$. Let $p \in C$ and consider the embedding
$$
P_{\chi}(C \times \C^2, d) \hookrightarrow \coprod_{\chi = \chi_1 + \chi_2} P_{\chi_1}((C \setminus p) \times \C^2, d) \times P_{\chi_2}(\widehat{C}_p \times \C^2, d), 
$$
where $d$ denotes the degree of the curve class\footnote{The precise definition of these moduli spaces is as in Section \ref{secres}: we assume the underlying reduced supports of the stable pairs in each moduli space are $C$, $C \setminus p$, $\widehat{C}$ respectively and $d$ denotes the multiplicity of the underlying Cohen-Macaulay supports along these curves.}. The torus $T = \C^{2*}$ is acting on both spaces by scaling of the factors of $\C^2$ and the fixed loci are indexed by partitions $\lambda \vdash d$ as earlier in this section. Again we use the notation $(\cdot)_\lambda$ to indicate that the ``cross-section'' of the underlying Cohen-Macaulay support curve has been fixed to be the monomial ideal corresponding to $\lambda$. We obtain a bijective morphism
$$
P_{\chi}(C \times \C^2, d)_\lambda \cong \coprod_{\chi = \chi_1 + \chi_2} P_{\chi_1}((C \setminus p) \times \C^2, d)_\lambda \times P_{\chi_2}(\widehat{C}_p \times \C^2, d)_\lambda. 
$$
Summing over all $\chi$ gives the following lemma.
\begin{lemma} \label{lem8}
\begin{align*}
\sum_{\chi} e(P_{\chi}(C \times \C^2, d)_\lambda) q^{\chi} = W_{\lambda,\varnothing,\varnothing}(q) \cdot \sum_{\chi} e(P_{\chi}(\widehat{C}_p \times \C^2, d)) q^{\chi},
\end{align*}
\end{lemma}
\begin{proof}
To obtain the stable pair vertex use a $\C^{*3}$-action on $\widehat{C}_p \times \C^2$ as in the previous subsection.
\end{proof}

Putting everything together we obtain the desired generating function.

\begin{proposition} \label{punctgenfun}
For each fibre $F_{p_i}$ (nodal or not) we have
$$
\sum_{\chi} e(P_{\chi}(N_{F_{p_i}^{\circ} / X^\circ}, \p)_{\lambda}) q^\chi = \frac{1}{W_{\lambda,\varnothing,\varnothing}(q)}.
$$
\end{proposition}
\begin{proof}
Combine Lemmas \ref{lem7}, \ref{lem8}.
\end{proof}


\section{Finale (Schur)}

We calculate the disconnected generating function \eqref{discon} first. The connected generating function \eqref{con} then follows easily. Denote by $B^{\circ} \subset B$ the locus of smooth fibres and by $B^{\sing} \subset B$ the locus of singular fibres. Let $\mathrm{Conf}^{i}(B^\circ)$ be the configuration space of $i$ unordered points on $B^\circ$ and let $N := |B^{\sing}|$. Lemma \ref{lem1} implies
\begin{align*}
Z^{P \mdot}(q,y) &= \sum_\chi \sum_{i = 0}^{\infty} \sum_{i'=0}^{N} \sum_{d_1, \ldots, d_i \geq 0} \sum_{d_{1}', \ldots, d_{i'}' \geq 0} y^{\sum_{a=1}^{i} d_a + \sum_{a=1}^{i'} d_{a}'} \cdot e(\mathrm{Conf}^{i}(B^\circ)) \cdot \binom{N}{i'} \times \\
&\qquad e\big(P_\chi\big(X,\sum_{a=1}^{i} d_a p_a + \sum_{a=1}^{i'} d_{a}' p_{a}'\big)\big) \\
&=\sum_\chi \sum_{i = 0}^{\infty} \sum_{i'=0}^{N} \sum_{d_1, \ldots, d_i \geq 0} \sum_{d_{1}', \ldots, d_{i'}' \geq 0} y^{\sum_{a=1}^{i} d_a + \sum_{a=1}^{i'} d_{a}'} \cdot \binom{e(B) - N}{i} \cdot \binom{N}{i'} \times \\
&\qquad e\big(P_\chi\big(X,\sum_{a=1}^{i} d_a p_a + \sum_{a=1}^{i'} d_{a}' p_{a}'\big)\big),
\end{align*}
where $p_1, \ldots, p_i$ are any choice of distinct points on $B^\circ$, $p_{1}', \ldots, p_{i'}'$ are any choice of distinct points among $B^{\sing}$, and
\begin{equation} \label{binom}
\binom{n}{k} := (-1)^k \binom{k-n-1}{k},
\end{equation} 
for $n<0$. We abbreviate $\p := \sum_{a=1}^{i} d_a p_a$, $\p' := \sum_{a=1}^{i'} d_{a}' p_{a}'$, $\bfd := \sum_{a=1}^{i} d_a$, and $\bfd' := \sum_{a=1}^{i'} d_{a}'$. Lemma \ref{lem6} gives
\begin{align*}
&\sum_{i = 0}^{\infty} \sum_{i'=0}^{N} \sum_{d_1, \ldots, d_i \geq 0} \sum_{d_{1}', \ldots, d_{i'}' \geq 0} \sum_{\chi} \sum_{\chi_1, \ldots, \chi_i} \sum_{\chi_{1}', \ldots, \chi_{i'}'} \sum_{\bslambda = \{\lambda_a \vdash d_a\}_{a=1}^{i}} \sum_{\bslambda' = \{\lambda_{a}' \vdash d_{a}'\}_{a=1}^{i'}} y^{\bfd + \bfd'} \cdot  \binom{e(B) - N}{i} \cdot \binom{N}{i'} \times \\
&e(P_{\chi}(X \setminus \{q_1, \ldots, q_{i}, q_{1}', \ldots, q_{i'}', r_1, \ldots, r_{i'}\} ,\p + \p')_{\bslambda, \bslambda'}) \times \\
&\prod_{a=1}^{i} e(P_{\chi_a}(\widehat{X}_{q_a},\p+\p')_{\bslambda}) \cdot \prod_{a=1}^{i'} e(P_{\chi_{a}'}(\widehat{X}_{q_{a}'},\p+\p')_{\bslambda'}) \cdot \prod_{a=1}^{i'} e(P_{\chi_{a}'}(\widehat{X}_{r_{a}'},\p+\p')_{\bslambda'}),
\end{align*}
where $q_1, \ldots, q_i$ denote the nodes arising from $F_{p_1}, \ldots, F_{p_i}$ intersecting the zero section, $q_{1}', \ldots, q_{i'}'$ denote the nodes arising from $F_{p_{1}'}, \ldots, F_{p_{i'}'}$ intersecting the zero section, and $r_1, \ldots, r_{i'}$ are the internal nodes of $F_{p_{1}'}, \ldots, F_{p_{i'}'}$. The sums $\sum_{\chi} \sum_{\chi_1, \ldots, \chi_i} \sum_{\chi_{1}', \ldots, \chi_{i'}'} \cdots$ can be done using equations \eqref{formalgenfun1}, \eqref{formalgenfun2}, and \eqref{secgenfun} 
\begin{align*}
&\sum_{i = 0}^{\infty} \sum_{i'=0}^{N} \sum_{d_1, \ldots, d_i \geq 0} \sum_{d_{1}', \ldots, d_{i'}' \geq 0} \sum_{\bslambda = \{\lambda_a \vdash d_a\}_{a=1}^{i}} \sum_{\bslambda' = \{\lambda_{a}' \vdash d_{a}'\}_{a=1}^{i'}} y^{\bfd + \bfd'} \cdot \binom{e(B) - N}{i} \cdot \binom{N}{i'} \times \\
&\frac{q^{1-g(B)}}{(1-q)^{e(B) - i - i'}} \cdot \prod_{a=1}^{i} \frac{W_{\lambda_{a}, (1), \varnothing}(q)}{W_{\lambda_a,\varnothing,\varnothing}(q)} \cdot \prod_{a=1}^{i'} \frac{W_{\lambda_{a}', \lambda_{a}^{ \prime t}, \varnothing}(q) W_{\lambda_{a}', (1), \varnothing}(q)}{W_{\lambda_{a}',\varnothing,\varnothing}(q)} \\
&= \frac{q^{1-g(B)}}{(1-q)^{e(B)}} \sum_{i = 0}^{\infty} \sum_{i'=0}^{N} \binom{e(B) - N}{i} \cdot \binom{N}{i'} \cdot \Bigg( (1-q) \sum_{\lambda} \frac{W_{\lambda, (1), \varnothing}(q)}{W_{\lambda,\varnothing,\varnothing}(q)} y^{|\lambda|} \Bigg)^{i} \times \\
&\Bigg( (1-q) \sum_{\lambda} \frac{W_{\lambda, \lambda^{t}, \varnothing}(q) W_{\lambda, (1), \varnothing}(q)}{W_{\lambda,\varnothing,\varnothing}(q)} y^{|\lambda|}\Bigg)^{i'}.
\end{align*}
With our convention for binomial coefficients \eqref{binom}, Newton's binomial theorem and the geometric series can be combined in one formula
\[
(1+x)^n = \sum_{k =0}^{n} \binom{n}{k} x^k, \ \mathrm{for \ all \ } n \in \Z.
\]
Performing the sums $\sum_{i = 0}^{\infty} \sum_{j=0}^{N} \cdots$ yields
\[
\bigg( \frac{q}{(1-q)^2} \bigg)^{1-g(B)} \Bigg( (1-q) \sum_{\lambda} \frac{W_{\lambda, (1), \varnothing}(q)}{W_{\lambda,\varnothing,\varnothing}(q)} y^{|\lambda|} \Bigg)^{e(B) - N} \cdot \Bigg( (1-q) \sum_{\lambda} \frac{W_{\lambda, \lambda^{t}, \varnothing}(q) W_{\lambda, (1), \varnothing}(q)}{W_{\lambda,\varnothing,\varnothing}(q)} y^{|\lambda|} \Bigg)^{N}.
\]
Similarly (but easier) one calculates the generating function $\sum_{d \geq 0} \sum_{\chi} P_{\chi,dF}(X) q^{\chi} y^d$
\[
\Bigg( \sum_{\lambda} y^{|\lambda|} \Bigg)^{e(B) - N} \cdot \Bigg( \sum_{\lambda} W_{\lambda, \lambda^{t}, \varnothing}(q) y^{|\lambda|} \Bigg)^{N}.
\]
We arrive at the following proposition.
\begin{proposition} \label{mainprop}
The connected generating series $Z^{P}(q,y)$ for stable pairs of $X = \mathrm{Tot}(K_S)$ of an elliptic surface $S \rightarrow B$ with section of genus $g(B)$ and $N$ 1-nodal fibres is given by
\begin{align*}
\bigg( \frac{q}{(1-q)^2} \bigg)^{1-g(B)} \Bigg( \frac{(1-q) \sum_{\lambda} \frac{W_{\lambda, (1), \varnothing}(q)}{W_{\lambda,\varnothing,\varnothing}(q)} y^{|\lambda|}}{\sum_{\lambda} y^{|\lambda|}} \Bigg)^{e(B) - N} \cdot \Bigg( \frac{(1-q) \sum_{\lambda} \frac{W_{\lambda, \lambda^{t}, \varnothing}(q) W_{\lambda, (1), \varnothing}(q)}{W_{\lambda,\varnothing,\varnothing}(q)} y^{|\lambda|}}{\sum_{\lambda} W_{\lambda, \lambda^{t}, \varnothing}(q) y^{|\lambda|}} \Bigg)^{N},
\end{align*}
where $W_{\lambda, \mu, \nu}(q)$ is the stable pairs vertex of \cite{PT2}.
\end{proposition}

The various generating functions of vertices appearing in this proposition can be computed. Obviously
\[
\sum_\lambda y^{|\lambda|} = \prod_{i=1}^{\infty} (1-y^i)^{-1}.
\]
More interesting is the following lemma.
\begin{lemma} \label{lem9}
The following identity holds\todo{I have not checked whether the overall $q^{...}$ factors work out. Is the power $y^{i-1}$ in RHS correct?}
\[
\sum_{\lambda} W_{\lambda, \lambda^{t}, \varnothing}(q) y^{|\lambda|} = \prod_{i=1}^{\infty} \Big( (1-y^i) \prod_{j = 1}^{\infty} (1-y^{i-1} q^j)^j \Big)^{-1}.
\]
\end{lemma}
\begin{proof}
\cite{ORV} and \cite{MacD} or exercise in \cite{Sta}.
\end{proof}
Less trivial is the following lemma.
\begin{lemma} \label{lem10}
The following identity holds
\[
(1-q) \sum_{\lambda} \frac{W_{\lambda, (1), \varnothing}(q)}{W_{\lambda,\varnothing,\varnothing}(q)} y^{|\lambda|} = \prod_{i=1}^{\infty} \frac{1-y^i}{(1-q y^i)(1-q^{-1} y^i)} .
\]
\end{lemma}
\begin{proof}
First apply \cite{ORV}. The remaining sum appears in \cite{BO} as pointed out by P.~Johnson answering a MathOverflow question. 
\end{proof}
The hardest is the following lemma\todo{STUCK ON THIS. Do we need help from Andrei, Paul, or Ben?}.
\begin{lemma} \label{lem11}
The following identity holds
\[
(1-q) \sum_{\lambda} \frac{W_{\lambda, \lambda^t, \varnothing}(q) W_{\lambda, (1), \varnothing}(q)}{W_{\lambda,\varnothing,\varnothing}(q)} y^{|\lambda|} = \Bigg( \prod_{i=1}^{\infty} \frac{1-y^i}{(1-q y^i)(1-q^{-1} y^i)} \Bigg) \cdot \Bigg( \prod_{i=1}^{\infty} \Big( (1-y^i) \prod_{j = 1}^{\infty} (1-y^{i-1} q^j)^j \Big)^{-1} \Bigg).
\]
\end{lemma}
\begin{proof}
?????
\end{proof}
We obtain a proof of the theorem in the introduction.
\begin{proof} [Proof of Theorem \ref{mainthm}]
Combine Proposition \ref{mainprop} and Lemmas \ref{lem9}, \ref{lem10}, and \ref{lem11}. 
\end{proof}

\newpage


\begin{thebibliography}{MNOP2}
%\bibitem[ACGH]{ACGH} E.~Arbarello, M.~Cornalba, P.~A.~Griffiths, and J.~Harris, \textit{Geometry of algebraic curves}, Volume I, Springer-Verlag (1985).
%\bibitem[AIK]{AIK} A.~Altman, A.~Iarrobino, and S.~Kleiman, \textit{Irreducibility of the compactified Jacobian}, Real and complex singularities (Proc. Ninth Nordic Summer School/NAVF Sympos.~Math., Oslo, 1976) 1--12 (1977).
%\bibitem[Beh]{Beh} K.~Behrend, \textit{Gromov-Witten invariants in algebraic geometry}, Invent.~Math.~127 601--617 (1997). arXiv:alg-geom/9601011v1.
\bibitem[Beh]{Beh} K.~Behrend
%\bibitem[BF]{BF} K.~Behrend and B.~Fantechi, \textit{The intrinsic normal cone}, Invent.~Math.~128 45--88 (1997). arXiv:alg-geom/9601010v1.
%\bibitem[BGS]{BGS} J.~Brian\c{c}on, M.~Granger, J.-P.~Speder, \textit{Sur le sch\'ema de Hilbert d'une courbe plane}, Ann.~Sci.~de l'~\'Ecole Normale Sup\'erieure 4 14 1--25 (1981).
%\bibitem[BL1]{BL1} J.~Bryan, C.~Leung, \textit{The enumerative geometry of K3 surfaces and modular forms}, J.~Amer.~Math.~Soc.~13 371--410 (2000).
%\bibitem[BL]{BL} J.~Bryan and C.~Leung, \textit{Generating functions for the number of curves on abelian surfaces}, Duke Math.~J.~99 311--328 (1999). arXiv:math/9802125v1.
%\bibitem[Blo]{Blo} S.~Bloch, \textit{Semi-regularity and de Rham cohomology}, Invent.~Math.~17 51--66 (1972).
%\bibitem[BM]{BM} K.~Behrend, Yu.~Manin, \textit{Stacks of stable maps and Gromov--Witten invariants}, Duke Math.~J.~85 1 1--60 (1996).
\bibitem[BO]{BO} S.~Bloch and A.~Okounkov
%\bibitem[BPV]{BPV} W.~Barth, C.~Peters, and A.~van de Ven, \textit{Compact complex surfaces}, Springer-Verlag (1984).
%\bibitem[Bri]{Bri} T.~Bridgeland, \textit{Hall algebras and curve counting}, JAMS 24 969--998 (2011).
%\bibitem[BS]{BS} M.~Beltrametti and A.~J.~Sommese, \textit{Zero cycles and $k$th order embeddings of smooth projective surfaces. With an appendix by Lothar G\"ottsche}, Problems in the theory of surfaces and their classification (Cortona, 1988) Sympos.~Math.~32 33--48 Academic Press (1991).
\bibitem[BT]{BT} O.~Ben-Bassat and M.~Temkin
%\bibitem[Che]{Che} J.~Cheah, \textit{The cohomology of smooth nested Hilbert schemes of points, thesis}, Chicago (1994).
%\bibitem[CK]{CK} H.-l.~Chang and Y.-H.~Kiem, \textit{Poincar\'e invariants are Seiberg-Witten invariants}, to appear in Geom.~and Topol., arXiv:1205.0848.
%\bibitem[DKO]{DKO} M.~D\"urr, A.~Kabanov, and C. Okonek, \textit{Poincar\'e invariants}, Topology 46 225--294 (2007).
%\bibitem[Don]{Don} S.~K.~Donaldson, \textit{Irrationality and the $h$-cobordism conjecture}, J.~Diff.~Geom.~26 141--168 (1987).
%\bibitem[EG]{EG} Edidin and Graham, equivariant GRR.
%\bibitem[EGL]{EGL} G.~Ellingsrud, L.~G\"ottsche, and M.~Lehn, \textit{On the cobordism class of the Hilbert scheme of a surface}, Jour.~Alg.~Geom.~10 81-100 (2001). %arXiv:math/9904095v1.
%\bibitem[FM]{FM} R.~Friedman and J.~W.~Morgan, \textit{Obstruction bundles, semiregularity and Seiberg-Witten invariants}, Comm.~Anal.~Geom.~7 451--495 (1999).
%\bibitem[Ful]{Ful} W.~Fulton, \textit{Intersection theory}, Springer-Verlag (1998).
%\bibitem[GH]{GH} Ph.~Griffiths, J.~Harris, \textit{Principles of algebraic geometry}, Wiley Classics Library (1994).
%\bibitem[Got]{Got} L.~G\"ottsche, \textit{A conjectural generating function for numbers of curves on surfaces}, Comm.~Math.~Phys.~196 523-533 (1998).
\bibitem[GP]{GP} T.~Graber and R.~Pandharipande, \textit{Localization of virtual classes}, Invent.~Math.~135 487--518 (1999). %arXiv:alg-geom/9708001v2.
%\bibitem[GV1]{GV1} R.~Gopakumar and C.~Vafa, \textit{M-theory and topological strings---I}, hep-th/9809187.
%\bibitem[GV2]{GV2} R.~Gopakumar and C.~Vafa, \textit{M-theory and topological strings---II}, hep-th/9812127.
\bibitem[Har]{Har} R.~Hartshorne, \textit{Algebraic geometry}, Springer-Verlag (1977).
%\bibitem[HL]{HL} D.~Huybrechts, M.~Lehn, \textit{The geometry of moduli spaces of sheaves}, Cambridge University Press (2010).
%\bibitem[HTT]{HTT} Y.~Hinohara, K.~Takahashi, H.~Terakawa, \textit{On tensor products of $k$-very ample line bundles}, Proc.~Amer.~Math.~Soc.~133 687--692 (2004).
%\bibitem[HM]{HM} J.~Harris, I.~Morrison, \textit{Moduli of curves}, Springer-Verlag (1998).
%\bibitem[HT]{HT} D.~Huybrechts and R.~P.~Thomas, \textit{Deformation-obstruction theory for complexes via Atiyah and Kodaira-Spencer classes}, Math. Ann. 346 545--569 (2010).
%\bibitem[Iar]{Iar} A.~Iarrobino, \textit{Punctual Hilbert schemes}, Bull.~Amer.~Math.~Soc.~78 819--823 (1972).
%\bibitem[Ill]{Ill} L.~Illusie, \textit{Complexe cotangent et d\'eformations I}, Lecture Notes in Math.~239 Springer-Verlag (1971).
%\bibitem[Jun]{Jun} Jun Li, \textit{A note on enumerating rational curves in a K3 surface}, in ``Geometry and nonlinear partial differential equations'' AMS/IP Studies in Adv.~Math.~29 (2002).
%\bibitem[JS]{JS} D.~Joyce and Y.~Song, \textit{A theory of generalized {D}onaldson-{T}homas invariants}, Memoirs of the AMS (2012). arXiv:0810.5645.
%\bibitem[KL1]{KL1} Y.-H.~Kiem and J.~Li, \textit{Gromov-Witten invariants of varieties with holomorphic 2-forms}, arXiv:0707.2986v1.
%\bibitem[KL2]{KL2} Y.-H.~Kiem and J.~Li, \textit{Localizing virtual cycles by cosections}, JAMS 26 1025--1050 (2013).
%\bibitem[KNS]{KNS} K.~Kodaira, L.~Nirenberg, D.~C.~Spencer, \textit{On the existence of deformations of complex analytic structures}, Ann.~Math.~68 450--459 (1958).
%\bibitem[KS]{KS} K.~Kodaira, D.~C.~Spencer, \textit{A theorem of completeness for complex analytic fibre spaces}, Acta Math.~100 281--294 (1958). 
%\bibitem[KST]{KST} M.~Kool, V.~Shende and R.~P.~Thomas, \textit{A short proof of the G\"ottsche conjecture}, Geom.~Topol.~15 397--406 (2011). arXiv:1010.3211v2.
%\bibitem[Kod]{Kod} K.~Kodaira, \textit{On the structure of compact complex analytic surfaces, I}, Am.~J.~Math.~86 751--798 (1964).
%\bibitem[KT1]{KT1} M.~Kool and R.~P.~Thomas, \textit{Reduced classes and curve counting on surfaces I: theory}, arXiv:1112.3069. 
%\bibitem[KT2]{KT2} M.~Kool and R.~P.~Thomas, \textit{Reduced classes and curve counting on surfaces II: calculations}, arXiv:1112.3070. 
%\bibitem[Kur]{Kur} M.~Kuranishi, \textit{On the locally complete families of complex analytic structures}, Ann.~Math.~75 536--577 (1962).
%\bibitem[Lee]{Lee} J.~Lee, \textit{Family Gromov-Witten invariants for K\"ahler surfaces}, Duke Math.~Jour.~ 123 209--233 (2004).
\bibitem[LeP]{LeP} J.~Le Potier
%\bibitem[LP]{LP} J.~Lee and T.~Parker, \textit{A structure theorem for the Gromov-Witten invariants of K\"ahler surfaces}, J.~Diff.~Geom.~77 483--513 (2007). 
%\bibitem[LT]{LT} J.~Li and G.~Tian, \textit{Virtual moduli cycles and Gromov-Witten invariants of algebraic varieties},  J.~Amer.~Math.~Soc.~11 119--174 (1998). arXiv:alg-geom/9602007v6.
\bibitem[MacD]{MacD} I.~G.~MacDonald
\bibitem[MacP]{MacP} R.~D.~MacPherson
%\bibitem[Man]{Man} M.~Manetti, \textit{Lectures on deformations of complex manifolds}, Rend.~Mat.~Appl.~24 1--183 (2004).
\bibitem[Mir]{Mir} R.~Miranda
%\bibitem[MNOP1]{MNOP1} D.~Maulik, N.~Nekrasov, A.~Okounkov, and R.~Pandharipande, \textit{Gromov-{W}itten theory and {D}onaldson-{T}homas theory, {I}}, Compos.~Math.~142 1263--1285 (2006). %arXiv:math/0312059v3.
%\bibitem[MNOP2]{MNOP2} D.~Maulik, N.~Nekrasov, A.~Okounkov, and R.~Pandharipande, \textit{Gromov-{W}itten theory and {D}onaldson-{T}homas theory, {II}}, Compos.~Math.~142 1286--1304 (2006). %arXiv:math/0406092v2.
%\bibitem[MOOP]{MOOP} D.~Maulik, A.~Oblomkov, A.~Okounkov, and R.~Pandharipande, \textit{Gromov-{W}itten/{D}onaldson-{T}homas correspondence for toric 3-folds}, Invent.~Math.~186 435--479 (2011).
%\bibitem[Moo]{Moo} J.D.~Moore, \textit{Lectures on Seiberg-Witten invariants}, Lecture Notes in Mathematics 1629, Springer-Verlag (1996).
%\bibitem[MP1]{MP1}  D.~Maulik and R.~Pandharipande, \textit{New calculations in Gromov-Witten theory}, Pure Appl.~Math.~Q.~4, Special Issue: In honor of Fedor Bogomolov, 469--500 (2008). 
%\bibitem[MP2]{MP2} D.~Maulik, R.~Pandharipande, \textit{Gromov-Witten theory and Noether-Lefschetz theory}, arXiv:0705.1653 (2010).
%\bibitem[MPT]{MPT} D.~Maulik, R.~Pandharipande and R.~P.~Thomas, \textit{Curves on K3 surfaces and modular forms}, J.~Topol.~3 937--996 (2010). 
%\bibitem[Mum]{Mum} D.~Mumford, \textit{Lectures on curves on an algebraic surface}, Princeton University Press 1966.
\bibitem[ORV]{ORV} A.~Okounkov, N.~Reshetikhin, and C.~Vafa
%\bibitem[Pot]{Pot} J.~Le Potier, \textit{Faisceaux semi-stables de dimension 1 sur le plan projectif}, Rev.~Roumaine Math.~Pures Appl.~38 635--678 (1993).
%\bibitem[Pid]{Pid} V.~Ya.~Pidstrigach, \textit{Deformations of instanton surfaces}, Izv.~Akad.~Nauk SSSR Ser.~Mat.~55 318--338 (1991).
%\bibitem[PP1]{PP1} R.~Pandharipande and A.~Pixton, \textit{Gromov-Witten/Pairs descendent correspondence for toric 3-folds}, arXiv:1203.0468.
%\bibitem[PP2]{PP2} R.~Pandharipande and A.~Pixton, \textit{Gromov-Witten/Pairs correspondence for the quintic 3-fold}, arXiv:1206.5490.
\bibitem[PT1]{PT1} R.~Pandharipande and R.~P.~Thomas, \textit{Curve counting via stable pairs in the derived category}, Invent.~Math.~178 407--447 (2009). %arXiv:0707.2348v3.
\bibitem[PT2]{PT2} R.~Pandharipande and R.~P.~Thomas, \textit{The 3-fold vertex via stable pairs}, Geom.~and Topol.~13 1835--1876 (2009).
\bibitem[PT3]{PT3} R.~Pandharipande and R.~P.~Thomas, \textit{Stable pairs and BPS invariants}, J.~Amer.~Math.~Soc.~23 267--297 (2010). %arXiv:0711.3899v3.
%\bibitem[PT4]{PT4} R.~Pandharipande and R.~P.~Thomas, \emph{in preparation}.
%\bibitem[Ran]{Ran} Z.~Ran, \textit{Semiregularity, obstructions and deformations of Hodge classes}, Ann.~Scuola Norm.~Sup.~Pisa Cl.~Sci.~4 28 809--820 (1999).
%\bibitem[Sie]{Sie} B.~Siebert, \textit{Virtual fundamental classes, global normal cones and Fulton's canonical classes}, in: Frobenius manifolds, ed.~K.~Hertling and M.~Marcolli, Aspects Math.~36 341--358, Vieweg (2004).
%\bibitem[Spi]{Spi} H.~Spielberg, \textit{Une formule pour les invariants de Gromov-Witten des vari\'et\'es toriques}, PhD Thesis Universit\'e Louis Pasteur (1999).
\bibitem[Sta]{Sta} R.~P.~Stanley II
%\bibitem[Tau1]{Tau1} C.~H.~Taubes, \textit{The Seiberg-Witten and Gromov invariants}, Math.~Res.~Lett.~2 221--238 (1995).
%\bibitem[Tau2]{Tau2} C.~H.~Taubes, \textit{Gr=SW: counting curves and connections}, J.~Diff.~Geom.~52 453-609 (1999). 
%\bibitem[Tod]{Tod} Y.~Toda, \textit{Curve counting theories via stable objects I.~DT/PT correspondence}, J.~Amer.~Math.~Soc.~23 1119--1157 (2010). 
%\bibitem[Voi1]{Voi1} C.~Voisin, \textit{Hodge theory and complex algebraic geometry I}, Cambridge University Press (2002).
%\bibitem[Voi]{Voi} C.~Voisin, \textit{Hodge loci}, Handbook of moduli (to appear).
%\bibitem[Wit]{Wit} E.~Witten, \textit{Monopoles and four-manifolds}, Math.~Res.~Lett.~1 769--796 (1994). \\
\end{thebibliography}
\end{document}


